\section{Approach Workflow}
% 我们的方法目的是帮助节点链接图的创作者自动生成辅助性语句,帮助节点链接图的目标观众理解节点链接图所要表达的内容。
We present \ApproachName~to generate presentation-aided statements automatically for node-link diagrams creators. The generated statements are aimed to facilitate the comprehension of the created node-link diagrams for target audiences.
% 为了生成能够帮助观众理解的语句,我们需要通过分析节点链接图的构建过程,来寻找观众会对节点链接图产生疑惑的原因。
To generate such presentation-aided statements, we need to decompose the creation process of node-link diagrams to explore the reasons of the audiences' confusion.
% 一个节点链接图的创作过程普遍会有以下X个步骤(其中第2步和第3步的顺序可以调换)[这里需要引一些文献]
% 1. 数据预处理;一般而言,真实世界收集到的源数据常常是表格型的多属性数据,而输入到节点链接图创作的图数据,往往是在源数据的基础上进行预处理,通过在源数据上定义实体之间的关系(链接),最终获得输入到节点链接图的图数据(包含节点和链接)
% 2. 计算图布局;节点链接图需要解决的一个关键问题是如何在二维空间中放置节点。链接则往往被绘制为节点之间的连接线,其位置往往由其两个端点决定而无需计算。基于布局算法产生的节点位置,节点链接图才能进行绘制。
% 3. 设计可视编码;创作者还会把他想表达的数据特征编码到节点链接图上,比如利用节点的颜色来编码节点的类别等。
% 虽然在很多工作中,还支持交互式的修改图布局。但这并不在本文的讨论中,本文只讨论为静态的节点链接图生成辅助性语句。
The creation of node-link diagrams commonly includes the following steps~\cite{DBLP:journals/cgf/SpritzerBDFF15, tvcg/RomatAP21}:
\begin{compactenum}[\textbf{Step} 1.]
    \item Data wrangling. 
    \item Layout computing.
    \item Visual encoding.
\end{compactenum}