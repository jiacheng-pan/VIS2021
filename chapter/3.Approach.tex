\section{\ApproachName}
% 我们提出了xxx,通过抽取信息并填充模板的方式来描述节点链接图,它大概分为三个步骤~\REf{}:1. xxx; xxx
We introduce \ApproachName, which describes node-link diagrams by extracting information and filling templates.
Figure~\ref{fig:workflow} shows the pipeline our our approach with several example descriptions to demonstrate the generated results.
\ApproachName consists of three steps:
\begin{compactenum}
\item \textbf{Interpreting linking conditions} to describe how relationships between nodes are constructed.
\item \textbf{Interpreting visual encodings} by extracting visual mappings among data entities, attributes, and visual channels.
\item \textbf{Interpreting layout meanings} by detecting which kinds of layout is employed.
\end{compactenum}


\begin{figure}
    \centering
    \includegraphics[width=1\columnwidth]{figures/workflow.eps}
    \caption{Caption}
    \label{fig:workflow}
\end{figure}

\subsection{Interpreting Linking Conditions}
% 从表格型数据创建图数据时,最重要的就是建立节点之间的关系。
% 观众从节点链接图中所能获知的信息仅仅是某两个节点之间存在联系,但却不了解联系的内在含义。
% 一些论文已经对可能建立链接的情况进行了定义,
Several techniques~\cite{DBLP:journals/ivs/LiuNS14, DBLP:journals/ivs/HeerP14, DBLP:journals/tvcg/SrinivasanPEB18} about graph wrangling identify link construction as the crucial process and propose several linking conditions.
Thus our technique describes linking conditions to interpret the graph wrangling step.
Ploceus~\cite{DBLP:journals/ivs/LiuNS14} and Orion~\cite{DBLP:journals/ivs/HeerP14} infer potential linking conditions by first constructing a linking graph and then searching valid linking paths. They are aimed at constructing links among multiple data tables by analyzing primary keys and foreign keys.
Graphiti~\cite{DBLP:journals/tvcg/SrinivasanPEB18} identifies potential linking conditions of a homogeneous graph by comparing different attributes.
% 如果多个表合并成一个表,前两者总结的条件可以被Graphiti提出的规则所覆盖
Because multiple tables can be merged into one data table with primary keys and foreign keys, Graphiti can cover the linking conditions identifying rules proposed by Ploceus and Orion. %? 这里mayby要用一张图解释一下
We reorganize the conditions proposed by these techniques into four categories:
\begin{compactenum}[\textbf{C}1]
    \item Two nodes share a same value of a same attribute, e.g., linking two movies is published in the same year.
    \item Two nodes share at least one same common value of a same list attribute, e.g., linking two movies with one or more same actors.
    \item Two nodes share significantly close value of a same attribute where the significance is defined by the normalized difference.
    \item Two nodes share two values in the same bin of a same attribute where the bins are defined as quartiles.
\end{compactenum}
Hear, we regard sharing same topics of textual attributes as same as sharing same values of list attributes (Condition 2), because topics are often presented as a list attribute.

% 我们尝试从它们的反方向进行思考,也就是,当我们获取到了所有的链接的时候,推测这些链接是如何被构建的
Those works infer the potential linking conditions before all links are constructed. 
Whereas, our method runs in the opposite direction where all links have been constructed already.
% 针对某一条链接,可能有不止一种构建条件,我们取所有链接的构建条件的交集作为整个graph的构建条件。
We traverse all links to detect potential conditions for each link.
The intersection of all links' conditions are regarded as the entire graph's linking condition.

% 为了能够将这些condition填充到文本模板中,我们对condition的输出进行了formalize
At last, we formalize the condition as three aspects, and textualize the linking condition of the graph by filling templates. 
One condition can be defined as:
\begin{equation}
    linking\text{ }condition := ( type, attribute, value )
\end{equation}
Four templates corresponds to four conditions according the $type$:
\begin{compactenum}[\textbf{C}1]
    \item \textit{``Two nodes are connected if their attributes} \texttt{\$attribute} \textit{are with a same value\$\{: }\texttt{\$value} \textit{?\}''}.
    \item \textit{``Two nodes are connected if their attributes} \texttt{\$attribute} \textit{share an intersection\$\{: }\texttt{\$value} \textit{?\}''}.
    \item \textit{``Two nodes are connected if their attribute} \texttt{\$attribute} \textit{are close, maybe with a difference less than } \texttt{\$value}\textit{''}.
    \item \textit{``Two nodes are connected if their attribute} \texttt{\$attribute} \textit{are within the same \$\{}\texttt{\$value}\textit{ ?\} bin''}.
\end{compactenum}

Where \texttt{\$attribute} is the name of the attribute, \texttt{\$value} is the value of the attribute, and \$\{?\} means the included part is optional. 
For example, for Condition 2, two movies are connected because they share same actors Alice and Bob. 
The condition is represented as: (``Condition 2'', ``actors'', [``Alice'', ``Bob'']). 
Its corresponding statement is: \textit{``Two nodes are connected if their attributes actors share an intersection: [``Alice'', ``Bob'']''}. 
And there is an implication relation between conditions.
Some conditions are stronger than others.
For example, the condition (``Condition 2'', ``actors'', [``Alice'', ``Bob'']) is stronger than (``Condition 2'', ``actors'', [``Alice'']) and (``Condition 2'', ``actors'', [arbitrary value]).
All the three conditions are regarded as individual conditions.

If a linking condition exists on both nodes with links and nodes without any link, it is regarded as a false condition and should be filtered out.
We describe our filtering strategy in~\ref{alg:conditions}.
After that, conditions are sorted, so that the highest ranked condition will be regard as the most possible condition.
First, all conditions are sorted by their frequency.
% 相同频率的条件,如果他们的type和attribute相同,那么根据
If two conditions with the same frequency have the same $type$ and $attribute$, they can be compared by their degree.
More stronger conditions are ranked higher than others.


% 一条边可能包含多个条件,不同边的条件之间可能有交集,可能没有交集。
% 我们需要一个策略来得到一个关于节点之间如何构建出关系的结论。
%! 解释一下算法
% 首先,对于所有没有边的节点对,找出所有可能的构建条件。
% 其次,对于某条边,它的构建条件和上述的构建条件之间找出差集。
% 最后,所有边的条件的合集属于边的构建条件;其中 频率最高的将被拿出来。



\begin{algorithm}[!t]
    \renewcommand\arraystretch{1.2}
    \caption{ Conditions filtering }
    \label{alg:conditions}
    \begin{algorithmic}[1]
        \Require
            $G=(V=\{v_1, v_2, ..., v_n\}, E=\{e_1, e_2, ..., e_n\})$: a graph;
        \Ensure
            $C$: conditions
        \State Init conditions $C=\varnothing$, false conditions $FC=\varnothing$
        \For {each node pair $(v_i, v_j)$}
            \If {$(v_i, v_j)$ is not a link}
                \State $C_{ij} \gets$ all conditions that can link $(v_i, v_j)$
                \State merge $FC$ with $C_{ij}$
            \EndIf
        \EndFor
        \For {each link $e_k=(v_i, v_j)$}
            \State $C_{ij} \gets$ all conditions that can construct link $e_i$
            \For {each condition $c$ in $C_{ij}$}
                \If {$c$ in $FC$}
                    \State remove $c$ from $C_{ij}$
                \EndIf
            \EndFor
            \State merge $C$ with $C_{ij}$
        \EndFor
        \State \Return $C$
    \end{algorithmic}
\end{algorithm}


\subsection{Interpreting Visual Encodings}
\subsubsection{Background}
% 为了在节点链接图中展示节点、链接的属性,常常将这些属性编码为节点/链接的视觉通道。
To visualize attributes of nodes and links in node-link diagrams, creators often encode them with visual channels on nodes and links.
For the node-link diagram example in Figure~\ref{fig:VisualEncodings}, a node contains a categorical attribute (\textit{x}) and a numerical list attribute (\textit{y}), we encode the numerical list attribute with two rectangles' height and encode the categorical attribute with two rectangles' fill color and the background rounded rectangle's stroke color.
We denote graphical elements (e.g., \texttt{<circle>}, \texttt{<rect>}, \texttt{<ellipse>}, etc.) in node-link diagrams as \textit{visual primitives} and their style attributes such as \texttt{cx}, \texttt{cy}, \texttt{width}, and \texttt{height} are denoted as \textit{visual channels}.
Nodes and links contained in the underlying graph are denoted as \textit{data entities} and each data entity consists of several \textit{attributes}.

\begin{figure}
    \centering
    \includegraphics[width=1\columnwidth]{figures/VisualEncodings.eps}
    \caption{An example shows the workflow of our visual encoding extracting technique. A node-link diagram consists of three nodes and three links (upper right corner). Node primitives are extracted and the \textbf {data binding} step binds them to different nodes. Then primitives with same role across different nodes are aligned into the same role class in the \textbf{primitives aligning} step. Mappings among different role calsses, visual channels, and attributes are detected by the \textbf{encoding mapping} step.}
    \label{fig:VisualEncodings}
\end{figure}

% 我们的工作通过分析源代码,从中提取数据实体()是如何被编码为视觉通道的()
We formulate the problem of describing visual encodings in node-link diagrams by three questions:
\begin{compactenum}[\textbf{Q}1]
    \item \textit{What primitives does a node/link consists of in the diagram?} For example, in Figure~\ref{fig:VisualEncodings}, a node is composed of three rectangles. \label{qstn:composition}
    
    \item \textit{What attributes do primitives and their visual channels encode?} For example, in Figure~\ref{fig:VisualEncodings}, the height of the left rectangle encodes the first item of the attribute $y$. If the shape (\texttt{tagName}) of a primitive encodes some attribute, we should classify and discuss when primitives are encoded into different shapes (e.g., when the node is encoded into a \texttt{<circle>}, the radius encodes its degree, and when the node is encoded into a \texttt{<rect>}, the width and the height encode its degree). \label{qstn:encodings}
    
    \item \textit{Whether there is a certain type of correlation (positive, negative, or categorical) between attributes and visual channels?} e.g., The greater the degree of the node, the greater the radius of the circle.\label{qstn:correlation}
\end{compactenum}
Thus, extracting mappings between entities and primitives and correlations between attributes and visual channels are necessary (Figure~\ref{fig:PrimitiveAligning} (a)).


\begin{figure}
    \centering
    \includegraphics[width=1\columnwidth]{figures/PrimitiveAligning.eps}
    \caption{(a) the target of our visual encoding detection technique; and (b) the effect of the data binding step and the primitives aligning step.}
    \label{fig:PrimitiveAligning}
\end{figure}

% xxxx 等人的工作为我们提供了一个良好的思路,但他们的工作存在一些限制
A tool proposed by Harper and Agrawala~\cite{DBLP:conf/uist/HarperA14} provides a creative perspective for visual encoding extraction.
Although their tool can be extended to support simple node-link diagrams (where each data entity is encoded into only one primitive), it still has several limitations:
\begin{compactenum}
% 1. 其需要创作者使用d3的数据绑定,才能发挥__data__的作用;使用其他工具,或者未将数据绑定到元素上时则无法使用该方法;
\item The data binding feature of D3 is required within the tool such that the ``\texttt{\_\_data\_\_}'' attribute of visual primitives can be acquired by it. 
It cannot deal with SVG-format visualizations without data bound to visual primitives.

% 2. 其只能检验属性和视觉通道之间是否存在线性映射或者类别性映射。
\item Only the linear mapping and the categorical mapping are supported, it cannot maintain situations where visual mappings are complex.
\end{compactenum}

% 我们针对节点链接图的场景,提出了一个数据绑定的策略,通过不断调整输入的数据,检查输出的变化,从而获取数据到svg元素之间的映射关系以及属性到视觉通道之间的映射方式,以解决以上两个问题。
For node-link diagrams, we introduce a new technique to solve such limitations.
The main idea of our technique is regarding the source code as a black box.
Our technique obtains mappings between data entities and visual primitives by detecting changes of the output SVG after modifying the input graph data.
It overcomes the limitations by three steps (Figure~\ref{fig:VisualEncodings}):
(1) \textbf{Data binding} binds primitives into different data entities.
(2) \textbf{Primitives aligning} aligns primitives according to their \textit{roles}. 
(3) \textbf{Encoding mapping} detects correlations between attributes, primitives, and their visual channels.

\subsubsection{Data Binding} \label{sec:databinding}
To detect mappings between data entities and visual primitives, 
we modify attribute values of data entities and record corresponding visual primitive changes.
Mappings between the entity and the changed primitives are hence constructed.
However, directly modifying attributes may deform the attribute distribution, which can influence visual mappings, so that primitives related to other data entities may also be changed.
For example, modifying attributes may broaden the attribute range and thus changes linear mappings defined by the attribute range.
We prevent it by only swapping attributes of two entities rather than modifying them, such that no new data is introduced and the distribution is preserved.
We take nodes for example.
Each node will be swapped with all the other nodes to make sure all primitives belongs to it are detected.
Primitives changed by each swapping correspond to two swapped nodes (Figure~\ref{fig:DataBinding} (b) and (c)).
For example, after swapping nodes B and C in Figure~\ref{fig:DataBinding} (b), primitives 1 to 4 are changed.
All these changed primitives correspond to nodes B and C, because only B and C are swapped.
After swapping the node B with the node D in Figure~\ref{fig:DataBinding} (c), primitives 2 and 3 are changed twice (Figure~\ref{fig:DataBinding} (b) and (c)), thus they correspond to node B because only node B are changed twice.
To ensure all primitives of node B are detected, we swap it with all the other nodes.
After swapping all nodes, the entire node-primitive mappings are constructed (Figure~\ref{fig:DataBinding} (d)), thus node entities are bound to their primitives.
Mappings between Links with their primitives are constructed in the same.
Because swapping two nodes may influence their related links, nodes' related primitives can contain links' primitives,
we remove links' primitives from the node-primitive mappings.

\begin{figure}
    \centering
    \includegraphics[width=1\columnwidth]{figures/DataBinding.eps}
    \caption{Data binding is achieved by swapping attributes of data entities. (a) Visual mappings between original visual primitives and data entities are unknown. (b) After swapping attributes of entities B and C, the appearance of primitives 1 to 4 is changed. Thus, entities B and C correspond to primitives 1 to 4. (c) After swapping attributes of entities A and B, the appearance of primitives 2, 3, 7, and 8 is changed. Thus, entities A and B correspond to these primitives. (d) After swapping B with A and C, primitives 2 and 3 are changed twice, we can map the entity B to primitives 2 and 3.}
    \label{fig:DataBinding}
\end{figure}

\subsubsection{Primitives Aligning}
% Different primitives are aligned according to their roles (the vertical direction in Figure~\ref{fig:PrimitiveAligning} (a)).
The data binding step only binds visual primitives into different data entities (the horizontal direction in Figure~\ref{fig:PrimitiveAligning} (b)).
The role of different primitives within one data entity are undecided.
The role of a primitive can be defined as its affects of visualizing data attributes.
For the example in Figure~\ref{fig:VisualEncodings}, although both of the two inner rectangles are affected by the attribute \textit{y}, their affects are differently for the entire node (the left one encodes the first item in attribute \textit{y} and the right one encodes the second).
Whereas, affects of left rectangles across different nodes are same (they all encode the first item of $y$).
They should be aligned to clarify their roles (the vertical direction in Figure~\ref{fig:PrimitiveAligning} (b)).
% The \textit{role} can be regarded as a function that takes data attributes as input and assign values for visual primitive channels.
Primitives across different data entities are regarded as the same role if their encoding schemes are the same.
% Primitives within the same role should appear exactly the same when their own nodes have exactly same attributes.
We test the consistency of encoding scheme between two primitives by checking whether two primitives appear consistently with same attributes.
Because we totally swap all the attributes of two entities, one entity before swapping is same to the other one after swapping.
We can examine the consistency of primitives along with the swapping step.
For example, in Figure~\ref{fig:DataBinding} (b), after swapping entities B and C,primitive 1 appear same to primitive 2 before swapping (Figure~\ref{fig:DataBinding} (a)). Thus, primitive 1 and 2 can be aligned into the same role.
It clarifies the binding between data entities and visual primitives, which is conductive to the subsequent steps.
% 这样做的意义:能够使得数据和可视化之间的绑定更加清晰,有利于后续步骤的进行。


\subsubsection{Encoding mapping}\label{sec:encodingmapping}
The previous two steps classify primitives according to two dimension: entity and role.
We are able to answer \textbf{Q\ref{qstn:composition}}.
% 到此为止,我们已经已经知道每个实体是由哪些不同角色的元素组成,足够回答第一个问题
However, correlations between visual channels and data attributes are not determined.
% 我们继续以节点为例。
We continue to take nodes for example.
% 为了检验一个属性究竟被编码在了哪些视觉通道上,我们通过shuffle所有节点的某个属性,观察视觉通道发生的变化。
We map one attribute to its related visual channels by first shuffling values of the attribute of all nodes and then observing the changes of visual channels of primitives.
% 变化可以被定义为,某个属性引起某些元素的某些视觉通道的变化。
One change can be defined as the difference of a certain \textit{visual channel} belongs a certain \textit{primitive} caused an \textit{attribute}, which can be formalized as:
\begin{equation}
    change := \{ attribute, primitive, channel \}
\end{equation}
% 然后我们根据不同的primitive角色,对这些发生的变化进行合并。
We merge those changes according to the role of primitive.
It generates more comprehensive mappings between attributes and visual channels for primitives of different roles.
\textbf{Q\ref{qstn:encodings}} is solved.


To solve \textbf{Q\ref{qstn:correlation}}, we test correlations between attributes and visual channels.
Numerical and categorical attributes are supported.
And all the visual channels are regarded as numerical.
However, numerical data can be used as categorical data if there are only a few classes of values.
For example, natural numbers are usually used as categorical attributes such as labels, groups, and classes.
Thus, for the numerical data, we should compare the number of unique values and the number of their entries to determine whether it is numerical or categorical.
We set up a parameter $\alpha$, where the number of unique attribute values is less than $\alpha \%$ of the number of data entities, it is regarded as categorical.
We generate different correlations for different cases:
\begin{compactitem}
    \item The channel is categorical. The correlation is recorded and described by the channel's categories. For each category of the channel, we record value range of the attribute. % 我们记录了当视觉通道表达为这一类值的时候,属性的取值范围。
    If the value ranges of the attribute are intersected across different channel categories, we discard the correlation because it is ambiguous.
    \item Both the channel and the attribute numerical. We compute the Pearson's Correlation Coefficient and test whether the correlation is positive, negative, or uncorrelated with the generated p-value of the significance test. We take the attribute as the impact factor of the layout when the absolute value of the coefficient are larger than $0.5$ and the $p$-value of the significance test is less than $0.05$. 
    Both two parameters can be adjusted freely.
\end{compactitem}


\subsubsection {Template-based Textualization}
Nodes are described at first, followed by descriptions of links. 
% 为了标记模板中用于填充信息的部分,我们对这些位置进行了高亮。
To mark the placeholders of templates that are used to fill in the information, we highlight these vacancies.

% 首先,我们需要描述数据实体被可视化为哪些视觉元素
We answer \textbf{Q\ref{qstn:composition}} with the primitives that compose a node.
Because the shape (\texttt{tagName}) of primitives may encode some attributes, we first describe the number of primitives:
\textit{``Each node consists of \textbf{\color{white}\hl{ 4 }} different primitives''}. 
Then, primitives of different roles are categorized and discussed. 
For each primitive, there are several visual channels encoding attributes.
We describe its shape first: \textit{``The \textbf{\color{white}\hl{ first }} primitive is a {\color{text-highlight}\texttt{<rect>}}''}.
If the shape is used to encode some attribute, the template turns to: \textit{``For the \textbf{\color{text-highlight}second} primitive, its \texttt{tagName} is varying among multiple shapes: \textbf{\color{text-highlight}\texttt{<rect>}, \texttt{<ellipse>}, and \texttt{<circle>}}''}.

For the case where the shape encodes nothing, we describe visual channel correlations one by one.
We first declare which attribute is encoded by which channel: \textit{``Its \textbf{\color{text-highlight}fill} encodes the attribute \textbf{\color{text-highlight}group}''} (to solve \textbf{Q\ref{qstn:encodings}}).
We describe their correlation to solve \textbf{Q\ref{qstn:correlation}}.
If the correlation is categorical-to-categorical, we describe different categories respectively: \textit{``When the value of the attribute \textbf{\color{text-highlight}group} is \textbf{\color{text-highlight}2}, its \textbf{\color{text-highlight}fill} turns to \textbf{\color{text-highlight}darkorange(\#ff7f0e)}.''} (To describe a color, we find the most similar color from a pre-defined set of colors with 146 different names).
If the correlation is numerical-to-numerical, we describe whether the channel will increase or decrease with the increase of the attribute value: \textit{``The greater the attribute \textbf{\color{text-highlight}value} is, the \textbf{\color{text-highlight}greater} its \textbf{\color{text-highlight}stroke-width} is.''}.

And if the shape (\texttt{tagName}) of a primitive encodes some attributes, we first describe common channels shared by different shapes (e.g., fill, stroke, stroke-width, etc.): \textit{``The \textbf{\color{text-highlight}fill} of all these shapes encodes the attribute \textbf{\color{text-highlight}group}''}.
Different shapes have different visual channels, we describe them separately: \textit{``When it turns to \textbf{\color{text-highlight}\texttt{<circle>}}, its \textbf{\color{text-highlight}r} encodes the attribute \textbf{\color{text-highlight}degree}''}.

Examples of generated descriptions are referred in Figure~\ref{}.

\subsection{Interpreting Layout Meanings}
Layouts of attributed graphs have been categorized into attribute-based layouts~\cite{} and topology-based layouts~\cite{} by Nobre et al.~\cite{DBLP:journals/cgf/NobreMSL19}.
Our technique focus on describing whether the layout is an attribute-based layout or a topology-based layout.
The core is to determine the type of the layout.
It contains two steps: 

\textbf{1) Capturing the position of each node;} To capture the position of each node, we compute a bounding box for primitives detected in Section~\ref{sec:databinding}.
The bounding box of a node is the smallest rectangle that contains all primitives of it.
We take the centroid of the bounding box as the position of the node.

\textbf{2) Determining the type of the layout.} 
The attribute-base category of layouts are defined as algorithms that mapping node attributes to the two dimensions (x and y) of the Cartesian coordinate. 
For each attribute of the node, we test whether it relates to the layout.
The test is similar to the correlation test in Section~\ref{sec:encodingmapping}.
Descriptions are also template-based: \textit{``The \textbf{\color{text-highlight}x}-coordinate encodes the attributes \textbf{\color{text-highlight}votes}''}.
If the test does not suggest any attribute as the impact factor, we begin to test whether it is topology-based.
For topology-based layouts, we assume that the euclidean distance between two nodes is influenced by their graph geodesic distance.
The Pearson correlation coefficient can also be used to test the correlation.
We assume that with the conditions that the Pearson correlation coefficient is larger than $0.5$ and $p$-value is smaller to $0.05$, we can infer the layout is a topology-based layout.

Thus we first perform a pre-study to validate our hypothesis.
% 找N个不同的数据集(跨不同领域),找几种不同种类的拓扑based的layout;检验他们的pearson值,p值等等;比较它们在随机布局下的结果;
We prepare 15 different graph datasets from~\cite{DBLP:journals/tvcg/ZhuCHHLZ21}.
The number of their nodes varies from 72 to 1083 and the number of their links varies from 75 to 8677.
They are collected from diverse areas such as literature, music, research, social media, etc..
We layout them with 5 different kinds of topology-based layouts: FM$^3$~\cite{hachul2004drawing}, Fruchterman-Reingold spring layout (F.R.)~\cite{DBLP:journals/spe/FruchtermanR91}, Stress Majorization (S.M.)~\cite{DBLP:conf/gd/GansnerKN04
}, Pivot MDS (PMDS)~\cite{DBLP:conf/gd/BrandesP06}, and Radial Tree layout (R.T.)~\cite{DBLP:conf/infovis/Jankun-KellyM03}.
% 我们还检验了随机布局是否也符合我们的假设
We also test whether the random layout conforms to our assumption.
We highlight cells where Pearson correlation coefficient is smaller than $0.5$ in Table~\ref{tab:pearson-correlation} and where $p$-value is larger than $0.05$ in Table~\ref{tab:pearson-pvalue}.
They suggest there is no correlation between the euclidean distance and the graph geodesic distance.

% 这些基于拓扑的布局产生的节点位置,能够使节点间的欧式距离反应它们的拓扑距离。
Two tables suggest in most cases, the Euclidean distance between two nodes reflects their graph geodesic distance with topology-based layouts.
% 在大部分情况下,使用pearson 相关系数>0.5且p-value<0.05来检验是否该布局为拓扑布局是可行的。
Thus, it is feasible to check whether the layout is a topological layout with both two conditions established (the Pearson correlation coefficient$>0.5$ and $p$-value$<0.05$).

If the layout is topology-based, we only describe: \textit{``The layout is a topology-based layout, which means the greater the graph geodesic distance between two nodes is, the greater the euclidean distance between them is''}.
When the layout does not conform to the two categories, we describe that: \textit{``The layout is neither an attribute-based layout nor a topology-based layout.''}.

\begin{table}[ht]
\caption{\label{tab:pearson-correlation}Pearson correlation value(TODO).}

\setlength{\tabcolsep}{1.5mm}{

% \resizebox{\textwidth}{15mm}{
\begin{tabular}{lllllll}
\hline
Dataset       & FM$^3$ & F.R.           & S.M.  & PMDS  & R.T.           & Random          \\ \hline
dwt\_72       & 0.894                 & 0.552          & 0.94  & 0.897 & 0.714          & \textbf{-0.003} \\
lesmis        & 0.804                 & 0.728          & 0.815 & 0.779 & 0.504          & \textbf{-0.03}  \\
bcsstk09      & 0.967                 & 0.610          & 0.971 & 0.947 & 0.754          & \textbf{-0.008} \\
cage8         & 0.658                 & 0.678          & 0.725 & 0.701 & \textbf{0.307} & \textbf{-0.002} \\
can\_96       & 0.835                 & 0.824          & 0.855 & 0.869 & \textbf{0.426} & \textbf{0.013}  \\
dwt\_1005     & 0.969                 & 0.561          & 0.974 & 0.972 & 0.623          & \textbf{0.007}  \\
dwt\_419      & 0.979                 & \textbf{0.391} & 0.987 & 0.987 & 0.775          & \textbf{-0.004} \\
grid17        & 0.968                 & 0.551          & 0.976 & 0.964 & 0.738          & \textbf{-0.007} \\
jazz          & 0.826                 & 0.821          & 0.813 & 0.786 & \textbf{0.297} & \textbf{0.004}  \\
mesh3e1       & 0.99                  & 0.502          & 0.996 & 0.995 & \textbf{0.465} & \textbf{0.002}  \\
netscience    & 0.901                 & 0.573          & 0.93  & 0.919 & 0.752          & \textbf{-0.018} \\
price\_1000   & 0.783                 & \textbf{0.376} & 0.808 & 0.786 & 0.727          & \textbf{-0.007} \\
rajat11       & 0.753                 & 0.670          & 0.868 & 0.847 & 0.618          & \textbf{-0.064} \\
soc-wiki-Vote & 0.798                 & 0.775          & 0.821 & 0.757 & \textbf{0.316} & \textbf{-0.01}  \\
visbrazil     & 0.937                 & 0.828          & 0.941 & 0.958 & 0.758          & \textbf{0.01}   \\ \hline
\end{tabular}}
\end{table}

\begin{table}[ht]
\caption{\label{tab:pearson-pvalue}Pearson p-value(TODO).}

\setlength{\tabcolsep}{1.9mm}{

\begin{tabular}{lllllll}
\hline
Dataset       & FM$^3$ & F.R. & S.M. & PMDS & R.T. & Random         \\ \hline
dwt\_72       & 0                     & 0    & 0    & 0    & 0    & \textbf{0.842} \\
lesmis        & 0                     & 0    & 0    & 0    & 0    & 0.021          \\
bcsstk09      & 0                     & 0    & 0    & 0    & 0    & 0              \\
cage8         & 0                     & 0    & 0    & 0    & 0    & 0.027          \\
can\_96       & 0                     & 0    & 0    & 0    & 0    & \textbf{0.203} \\
dwt\_1005     & 0                     & 0    & 0    & 0    & 0    & 0              \\
dwt\_419      & 0                     & 0    & 0    & 0    & 0    & \textbf{0.111} \\
grid17        & 0                     & 0    & 0    & 0    & 0    & \textbf{0.060} \\
jazz          & 0                     & 0    & 0    & 0    & 0    & \textbf{0.421} \\
mesh3e1       & 0                     & 0    & 0    & 0    & 0    & \textbf{0.604} \\
netscience    & 0                     & 0    & 0    & 0    & 0    & 0              \\
price\_1000   & 0                     & 0    & 0    & 0    & 0    & 0              \\
rajat11       & 0                     & 0    & 0    & 0    & 0    & 0              \\
soc-wiki-Vote & 0                     & 0    & 0    & 0    & 0    & 0              \\
visbrazil     & 0                     & 0    & 0    & 0    & 0    & 0.025          \\ \hline
\end{tabular}}

\end{table}
