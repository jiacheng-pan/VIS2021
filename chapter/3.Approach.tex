\section{Approach Workflow}
% 我们的方法目的是帮助节点链接图的创作者自动生成辅助性语句,帮助节点链接图的目标观众理解节点链接图所要表达的内容。
We present \ApproachName~to generate presentation-aided statements automatically for node-link diagrams creators. The generated statements are aimed to facilitate the comprehension of the created node-link diagrams for target audiences.
% 为了生成能够帮助观众理解的语句,我们需要通过分析节点链接图的构建过程,来寻找观众会对节点链接图产生疑惑的原因。
To generate such presentation-aided statements, we need to decompose the creation process of node-link diagrams to explore the reasons for the audiences' confusion.
% 一个节点链接图的创作过程普遍会有以下X个步骤(其中第2步和第3步的顺序可以调换)[这里需要引一些文献]
% 1. 数据预处理;一般而言,真实世界收集到的源数据常常是表格型的多属性数据,而输入到节点链接图创作的图数据,往往是在源数据的基础上进行预处理,通过在源数据上定义实体之间的关系(链接),最终获得输入到节点链接图的图数据(包含节点和链接)
% 2. 设计可视编码;创作者还会把他想表达的数据特征编码到节点链接图上,比如利用节点的颜色来编码节点的类别等。
% 3. 计算图布局;节点链接图需要解决的一个关键问题是如何在二维空间中放置节点。链接则往往被绘制为节点之间的连接线,其位置往往由其两个端点决定。基于布局算法产生的节点位置,节点链接图才能进行绘制。
The creation of node-link diagrams commonly includes the following steps (the order of step 2 and step 3 can be reversed)~\cite{DBLP:journals/cgf/SpritzerBDFF15, tvcg/RomatAP21}:
\begin{compactenum}[\textbf{Step} 1.]
    \item \textbf{Graph wrangling}. Original data collected from the real world are usually tabular, while the graph data input into node-link diagrams are often with specific and explicit nodes and links. They are usually wrangled from tabular data by defining relationships between entities~\cite{DBLP:journals/tvcg/SrinivasanPEB18, DBLP:conf/ieeevast/BigelowNML19}.
    \item \textbf{Layout computing}. One key step of visualizing node-link diagrams is how to place nodes in a two-dimensional space. And links are placed according to two nodes each link connected.
    \item \textbf{Visual encoding}. To present data features of the underlying graph data, creators will encode several attributes they want to express with different visual channels. For example, the fill color of one node can be used to encode the category of it.
\end{compactenum}

% 因为最终展示给用户的仅仅为节点链接图,很多过程信息被隐藏起来。我们通过分析节点链接图创建过程中被隐藏的信息,用文本的形式显式揭露这些用户未知的信息,达到帮助理解的目的。这些隐藏的信息被我们总结为以下四类:%? warning: 这个总结暂时没有背书,可能权威性不是很高,可能可以尝试做一些interview
% 1. 链接的构建条件被隐藏。
% 2. 节点链接图的编码方案被隐藏。
% 3. 布局算法的具体实施方案被隐藏。
% 4. 布局产生的visual clutter导致拓扑结构被隐藏。

% 于是我们从这四个角度出发,提出了四个understanding,解决以上四种被隐藏的信息,分别是:understanding relationship, understanding visual encodings, understanding layout, understanding visual clutter. 我们的方法采用图数据和创作者的节点链接图创作代码作为输入,最后输出关于这四个understanding的辅助性解释文本。

\subsection{Understanding Relationship}
\subsection{Understanding Visual Encodings}
%! 首先说明目标是什么
% 为了能够向观众解释节点链接图的编码是什么,我们希望通过分析创作者使用的图数据,创作者编写的代码,得到他使用的编码方案。我们通过三个层次来叙述节点链接图的编码方案:
% 1. 节点/链接的组成元素分别是什么?
% 2. 视觉通道的决定性因素(属性)是什么?
% 3. 视觉通道和其决定性因素的相关性如何?(正相关/负相关/类别相关..)


%! Deconstructing and Restyling D3 Visualizations 的一些不足之处
%! 1. __data__的依赖
%! 2. 映射关系的检验
%TODO: 可能还存在一些不足,在实现的时候可以被发现
% Deconstructing and Restyling D3 Visualizations 对于解决上述问题提出了一些很有见解的解决方法。但它也有一些不足之处。
% 1. 其需要创作者使用d3的数据绑定,才能发挥__data__的作用;使用其他工具,或者未将数据绑定到元素上时则无法使用该方法;
% 2. 其检验视觉通道和属性之间的关联,是通过检验属性和视觉通道之间是否存在线性映射或者类别性映射。在以下两种情况下该方法会失效:
%   2.1 如果不同的数据属性之间相互是线性相关的,那么该方法就无法真正确定是哪一种属性决定了视觉通道;
%   2.2 其检验的映射过于简单,复杂情形下无法得到相互关联的结论;

% 我们尝试从这几个方向增强该方法,并配合节点链接图的特性,帮助创作者更好的解释节点链接图中的编码方案。
% 我们发现,虽然 Deconstructing and Restyling D3 Visualizations 的方法只需要svg作为方法的输入,但它的svg是处于运行环境中的,也就是说,如果没有运行源代码,__data__属性也就没法绑定到对应的元素中。所以它隐式地依赖于可视化创作者的源代码。所以我们还可以使用其源代码用于分析。


%! 我们的方法的过程
% 我们对Deconstructing and Restyling D3 Visualizations进行了改进;主要从x方面进行:
% 作为该方法的补充,为了使数据绑定能够应用于非d3创建的代码,我们提出了一个检验方法,它将创建者创建的代码视作黑盒,通过修改输入的数据,检验输出的svg元素的变化来获取编码方案。
%? 这里假设了修改源数据,不会对代码造成影响,但实际上很多情况下,映射方式会根据数据本身进行更改,比如数据值的范围。可以放在discussion里面讨论一下。
% 下面的章节我们围绕着几个方面进行:
% 1. 如何为没有为没有进行数据绑定的svg,找到每个节点和链接对应的元素,以补充 Deconstructing and Restyling D3 Visualizations;
% 2. 新的视觉通道和属性之间的关联检验方式,防止出现多属性之间线性依赖以及复杂映射情形失效的情况;
% TODO下面都是围绕节点和链接展开,但我们的方法不仅仅能够解决节点链接图的可视化,也可以用用于其他类型的可视化形式。

\subsubsection{Data Binding}
%! 讲一下我们主要检测的基础视觉元素:

%! 1 首先对某个属性做出修改,但不修改数据分布,只调换数据顺序;查看哪个/些视觉通道发生了变化;找到那些不改变dom结构的属性(tagname和element数量);
%!  对嵌套结构(数组/对象)进行说明:没有关系,等详细说明映射方式的时候再决定;
%! 2 过滤掉会对svg element的tagName和数量产生影响的属性,交换A节点和B节点的其余数据,观察哪些svg的视觉通道发生了变化;然后恢复数据;再交换A节点和C节点的其余数据,观察通道变化;取两次的交集的svg element,它们绑定的数据就是A节点;