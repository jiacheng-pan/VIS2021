\section{Discussion}
\textbf{Failure Cases and Limitations}.
We observed several failure cases as we developed \ApproachName.
Our technique assumes the consistency of the diagram;
it cannot deal with complex cases where different parts of the graph are handled differently.
For example, if some nodes are connected by condition \textbf{C1} while other nodes are connected by condition \textbf{C2}, \ApproachName cannot generate such conditional descriptions.
Similarly, \ApproachName~cannot deal with node-link diagrams where different nodes have different encoding schemes or different areas of the diagram are positioned by different layouts.
Such failure cases are rare and can be mitigated by add classification strategies to distinguish different parts with disparate schemes.
For example, a set of data entities with the same linking condition or the same encoding scheme may have commonalities in some attributes.
Another failure case is that if an attribute's visual encoding relies on the order of the attribute's values, swapping attributes between two entities will destroy the order; thus, it can cause the change of elements that do not correspond to them.
To deal with different visual mappings, we plan to support more strategies of data modification in the data binding step.

\textbf{Scalability}.
\ApproachName~has three parts:
\textbf{interpreting linking conditions} must traverse all pairs of nodes to find potential conditions.
Thus, its time complexity is $O(N^2)$, where $N$ is the number of nodes.
\textbf{Interpreting visual encodings} swaps all pairs of nodes and links, and then traverses and compares all elements to obtain their differences.
The number of elements is linear in the number of data entities, so time complexity is $O(N^3 + M^3)$.
The time to detect visual mappings between attributes and channels is on the order of $O(N^3 + M^3)$ and is negligible.
\textbf{Interpreting layout meanings} requires the computation of two distance matrices.
Finding the shortest paths of all node pairs is $O(N^3)$.
The most time-consuming part is interpreting visual encodings:
a new SVG element must be generated for each swap and visual channels for each element must be computed.
We plan to implement our algorithm on the server side to improve performance.
The generation of SVG elements and the computation of visual channels can be accelerated.
For larger-scale node-link diagrams, sampling techniques can be employed for acceleration.

\textbf{Extensibility}.
Although our visual encoding extraction method is designed for node-link diagrams, it can be extended to more kinds of visualizations in SVG format.
The content of basic charts (without axes, legends, etc.) can be regarded as special types of node-link diagrams.
For example, a line chart can be regarded as a node-link diagram whose underlying graph is a path graph.
\ApproachName~can  also facilitate code debugging.
Creators can check the creating logic with descriptions generated by \ApproachName.