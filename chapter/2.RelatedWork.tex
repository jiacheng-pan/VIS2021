\section{Related Work}\label{sec:relatedwork}

Here, we have summarized prior researches that have covered the features of node-link diagrams, visualization information extraction and visualization comprehensibility improvement.

\subsection{Features of Node-link Diagram}
Investigating the features of node-link diagrams enables to provide practical insights for us to augment their comprehensibility. From the taxonomy of~\cite{DBLP:journals/cgf/NobreMSL19}, they can be divided into four categories.

\noindent \textbf{Node Encodings}. 
For general node encodings, they are typically presented as various visual metaphors, such as size, color, and shape~\cite{tvcg/RomatAP21}. When it comes to multiple attributes encoding, among numerous invented techniques, nesting complex glyphs into one node~\cite{DBLP:conf/infovis/AuberCJM03} plays a critical role. For instance, nodes visualized as line charts, box plots, and bar charts are commonly seen, especially in some specific domains~\cite{DBLP:conf/infovis/Jankun-KellyM03, DBLP:journals/bmcbi/JunkerKS06, DBLP:conf/iv/JusufiDK10}.

\noindent \textbf{Link Encodings}.
The links of a graph have less space for encoding than that of nodes, but they can equally employ visual channels, such as width~\cite{Katz_2015}, color~\cite{DBLP:journals/tvcg/Guo09} and dashes~\cite{DBLP:journals/bmcbi/JunkerKS06}.
In more complicated cases, links incorporate multiple visual elements, such as embedding bar charts into links~\cite{DBLP:conf/iv/SchoffelSE16}. Abyss-Explorer~\cite{DBLP:journals/tvcg/NielsenJBJ09} ``wiggles'' the links and encode attributes of a link with its length.

\noindent \textbf{Layouts}.
Outside of encoding attributes on data entities, layouts play a critical role in visualizing graphs as well. They are divided into two categories~\cite{DBLP:journals/cgf/NobreMSL19}. The first category is topology-driven layouts~\cite{DBLP:journals/spe/FruchtermanR91, DBLP:journals/cgf/KruigerRMKKT17, DBLP:journals/tvcg/GansnerHN13, DBLP:journals/tvcg/ZhuCHHLZ21}, which are commonly applied to reveal topology structures. The second category is attribute-driven layouts, which are more popular in real-use cases where the abscissa and ordinate are clearly defined~\cite{DBLP:journals/tvcg/ElzenW14, DBLP:journals/tvcg/Guo09}. 

% \noindent \textbf{Authoring Tools}. To help generate node-link diagrams, graph visualization tools such as Cytoscape~\cite{DBLP:journals/bioinformatics/FranzLHDSB16}, Gephi~\cite{DBLP:conf/icwsm/BastianHJ09}, NetV.js~\cite{HAN2021} can manage varying degrees of attribute-embedding in node-link diagrams. To enable more expressive designs, Graphies~\cite{tvcg/RomatAP21} was invented to generate node-link diagrams for multivariate graphs.
 
Multifarious information is always encoded within one graph in real-life scenarios, making the visual design complex and thus impeding users' comprehension, even with the help of assistant tools. Moreover, the variability of designs aggravates the situation. The existing visualizations and systems cannot accommodate dynamic visual encoding requirements. Thus, we propose a generator to encode complex elements and layouts in an easy-to-read and easy-to-understand way.

\subsection{Information Extraction of Visualization}

Extracting information from visualizations is an essential step to interpret them~\cite{DBLP:conf/sibgrapi/MayhuaNHP18, DBLP:conf/pkdd/ClicheRMY17, DBLP:conf/icdar/LeeYWH17}. 
We divided the extraction approaches into three categories according to their content.

% - 1) data retrieval approaches, 2) insight extraction approaches, and 3) visual encoding extraction approaches. The latter two commonly use data retrieved by the first category of approaches to form insights and visual encodings.

\noindent \textbf{Data retrieval approaches}. Numerous methods retrieve data to interpret visualizations. They mainly focus on raster images. Their data retrieval workflow can be roughly summarized into three steps - classifying charts into different types~\cite{DBLP:conf/icip/GaoZB12, DBLP:conf/chi/JungKSHLKS17, DBLP:conf/eccv/SiegelHLDF16, DBLP:journals/vlc/DaiWNZ18}, detecting textual information from captions, labels, and axes by Optical Character Recognition (OCR) techniques~\cite{DBLP:conf/icip/ZhouT00, DBLP:conf/doceng/HuangT07, DBLP:conf/grec/HuangTL03}, and extracting data by detecting graphical components of chart images and assigning them to corresponding values~\cite{8334241, 8226805, Shao2005231}.
% Moreover, two different approaches~\cite{DBLP:conf/vmv/HenkelKLG20, DBLP:conf/icdar/LeeYWH17} regarding to graphs retrieve tree-structured data from treemaps~\cite{DBLP:conf/vmv/HenkelKLG20} and dendrograms~\cite{DBLP:conf/icdar/LeeYWH17}.

% ! Insights
\noindent \textbf{Insight extraction approaches}. 
% To present findings on visualizations, different methods extract \textbf{insights} from visualizations~\cite{DBLP:conf/diagrams/WuCEC10, DBLP:journals/nrhm/DemirOSECMC10} and their underlying data~\cite{DBLP:conf/inlg/ObeidH20, DBLP:journals/ivs/CuiBYE19}. 
\textit{Insights} are defined as strong manifestations of data or visualization in Foresight~\cite{DBLP:journals/pvldb/DemiralpHPP17}. 
By employing data retrieval approaches, insight extraction approaches generate insights based on the extracted data~\cite{DBLP:conf/inlg/ObeidH20, DBLP:conf/apvis/LiuXHWY20, DBLP:journals/tvcg/WangSZCXMZ20, DBLP:journals/ivs/CuiBYE19} and significant visual features~\cite{DBLP:conf/diagrams/WuCEC10, DBLP:journals/nrhm/DemirOSECMC10}. 
For example, by recognizing effective visual features, systems~\cite{DBLP:conf/diagrams/WuCEC10, DBLP:journals/nrhm/DemirOSECMC10} can recognize the intention of line charts and bar charts they try to convey.

%! visual mapping
\noindent \textbf{Visual Encoding extraction approaches}. Retrieving visual encodings from visualization results is crucial to understand visual designs. 
For instance, multiple reverse-engineering approaches~\cite{DBLP:journals/cgf/PocoH17, DBLP:journals/tvcg/PocoMH18, DBLP:conf/sibgrapi/MayhuaNHP18} recover visual encodings from chart images by using OCR techniques to detect textual information.
Deep learning methods open the new possibility to retrieve visual encodings, such as improving the detection of color mapping without textual legends~\cite{DBLP:journals/corr/abs-2103-00741}. 
Web-based visualizations usually employ D3~\cite{DBLP:journals/tvcg/BostockOH11}'s data-binding feature, enabling different tools~\cite{DBLP:conf/uist/HarperA14, DBLP:journals/tvcg/HarperA18, DBLP:journals/tvcg/HoqueA20} to retrieve the visual encoding.


Information extracted by these techniques is typically expressed by some representations, such as fact sheets~\cite{DBLP:journals/tvcg/WangSZCXMZ20}, reusable style templates~\cite{DBLP:journals/tvcg/HarperA18}, and descriptions~\cite{DBLP:conf/apvis/LiuXHWY20, DBLP:conf/inlg/ObeidH20}.
Our description of generation follows the similar \textit{extract-then-express} procedure.
However, most existing works are not mainly designed for node-link diagrams. Different data structures and visual designs make most of them fail. To the best of our knowledge, few prior works explore how to extract information from programs and then interpret node-link diagrams automatically.

\subsection{Visualization Comprehensibility Enhancements}

The graphic enhancements, such as captions, legends, labels of axes, are decisive for visualization comprehensibility enhancement. They present and convey information to end-users in an understandable way. Based on the representations, there are three relevant enhancements.

\noindent \textbf{Descriptions}. Description, one of the most natural way to interpret visualization, have been studied more than two decades~\cite{DBLP:journals/coling/MittalMCR98, DBLP:journals/tochi/FerresLST13}. Currently, some researches focus on extracting insights to generate descriptions~\cite{DBLP:conf/inlg/ObeidH20, DBLP:conf/nips/VaswaniSPUJGKP17, DBLP:conf/apvis/LiuXHWY20}.For instance, Chart-to-Text~\cite{DBLP:conf/inlg/ObeidH20} generates natural language descriptions for charts based on the transformer model~\cite{DBLP:conf/nips/VaswaniSPUJGKP17}. Similarly, AutoCaption~\cite{DBLP:conf/apvis/LiuXHWY20} extracts insights to generate captions.
Besides, several approaches generate descriptions to answer questions about visualizations~\cite{DBLP:conf/chi/KimHA20, DBLP:conf/eccv/KembhaviSKSHF16, DBLP:conf/cvpr/KaflePCK18}. An algorithm~\cite{DBLP:conf/chi/KimHA20} extends the ability of Sempre~\cite{DBLP:conf/acl/PasupatL15, DBLP:conf/emnlp/ZhangPL17} to answer questions about charts and provide visual explanations.
Moreover, relevant to our work, \cite{DBLP:conf/vissym/LatifSB19}  generate interactive descriptions for graphs with a declarative syntax. However, the description generation is manual and designed for graph authors. It inspires us to generate interactive descriptions automatically to improve their comprehensibility.


% Description, as one of the most natural way to interpret visualization, have been studied more than two decades~\cite{DBLP:journals/coling/MittalMCR98, DBLP:journals/tochi/FerresLST13}. They normally generate captions to describe mappings between data and marks for the description. Besides, extracting insights to generate descriptions~\cite{DBLP:conf/inlg/ObeidH20, DBLP:conf/nips/VaswaniSPUJGKP17, DBLP:conf/apvis/LiuXHWY20} is another important research domain. For instance, Chart-to-Text~\cite{DBLP:conf/inlg/ObeidH20} generates natural language descriptions for charts based on transformer model~\cite{DBLP:conf/nips/VaswaniSPUJGKP17}. Similarly, AutoCaption~\cite{DBLP:conf/apvis/LiuXHWY20} extracts insights to generate captions. Besides, several approaches generate descriptions to answer questions about visualizations~\cite{DBLP:conf/chi/KimHA20, DBLP:conf/eccv/KembhaviSKSHF16, DBLP:conf/cvpr/KaflePCK18}. The DVQA dataset~\cite{DBLP:conf/cvpr/KaflePCK18} provides numerous image-question pairs about bar charts to support question answering. Improved algorithm~\cite{DBLP:conf/chi/KimHA20} extends the ability of Sempre~\cite{DBLP:conf/acl/PasupatL15, DBLP:conf/emnlp/ZhangPL17} to answer questions about charts and provide visual explanations.
Moreover, relevant to our work, \cite{DBLP:conf/vissym/LatifSB19} helps to generate interactive descriptions for graphs with a declarative syntax and requires little to no programming. However, the description generation is manual and designed for graph authors. It inspires us to generate interactive descriptions automatically to improve end-users' understanding.

\noindent \textbf{Legends}. Numerous research explores various aspects of legends, ranging from design guidelines~\cite{DBLP:journals/tvcg/DykesWS10} to the impact of cartographic legend positioning~\cite{doi:10.1080/00087041.2018.1533293}.
Recently, interactive legends~\cite{DBLP:conf/ACMse/TudoreanuH04} convey mappings with a more economical format and provide a widget for users to adjust visual encodings.

\noindent \textbf{Annotations}. Annotations are widely used to enhance the comprehensibility of visualizations. There are two main kinds of annotations. The first one is observational annotations~\cite{DBLP:conf/ieeevast/ChenBY10, DBLP:conf/ieeevast/Kandogan12, DBLP:journals/tvcg/BryanMW17}, such as Click2Annotate~\cite{DBLP:conf/ieeevast/ChenBY10}, that mainly relies on visual features. 
% For example, Click2Annotate~\cite{DBLP:conf/ieeevast/ChenBY10} supports generating annotations to show insights. 
The other is additive annotations~\cite{DBLP:conf/chi/HullmanDA13, 10.1145/1622176.1622219, ren2017chartaccent, DBLP:conf/icip/ZhouT00}, which focus on providing information of the data~\cite{DBLP:conf/chi/LaiLJH0Y20}.
For node-link diagrams, most annotation-related prior works focus on reducing the visual overlapping ~\cite{DBLP:journals/isci/DogrusozKMT07, DBLP:conf/gd/HagenK08} rather than improving comprehensibility. 

To sum up, few relevant enhancements are able to deal with node-link diagrams with flexible visual designs due to their simplicity. Inspired by these related works, we proposed~\ApproachName~to generate textual descriptions for node-link diagrams. Compared to prior work, our approach can describe more flexible node-link diagrams in a more easy-to-read manner.