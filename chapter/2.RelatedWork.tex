\section{Related Work}\label{sec:relatedwork}
\subsection{Node-link Diagrams}
\subsection{Extracting Information From Visualization}
% Reverse-Engineering Visualizations: Recovering Visual Encodings from Chart Images: 从 像素图 中识别文本信息(标题、坐标轴标签等),对文本进行分类和还原,并且用CNN网络来分类mark类型,通过文本和mark类型来恢复编码信息。

% ChartSense: Interactive Data Extraction from Chart Images: 先将图像进行分类,然后利用了一个mixed-initiative approach对图标蕴含的数据进行抽取。人首选指定检测范围和坐标轴等辅助信息,ChartSense算法检测其中包含的图形元素及其蕴含的数据,最后由人挑选检测正确的元素并修改其中错误的数据。

% Deconstructing and restyling D3 visualizations: 使用了d3数据绑定的特性,直接从svg元素的__data__属性中读取其绑定的数据,然后用线性回归等方法检验元素的视觉属性和数据属性之间的映射关系来获取编码。

% Converting basic D3 charts into reusable style templates 和 Searching the Visual Style and Structure of D3 Visualizations 对 Decons.. 进行了改进。前者围绕着图表中蕴含的label(数据点的label和坐标轴的label)进行了更加细致的分类;后者则为该方法增加了一些新的检测目标:比如对角度的检测,对非编码的视觉通道(background color, stroke-width等)的检测等。
