\section{Case Studies}
We employ the IMDb movie dataset\footnote{\small\url{https://www.kaggle.com/stefanoleone992/imdb-extensive-dataset}} to conduct our case studies that demonstrate the descriptions generated by \ApproachName.
They are available online\footnote{\small\url{https://graphdescriptor.github.io/}}.


\begin{figure*}[ht]
    \centering
    % \setlength{\belowcaptionskip}{-5pt}
    \includegraphics[width=2\columnwidth]{figures/BasicCases.eps}
    \caption{Two simple node-link diagrams with descriptions generated by our approach. 
    Nodes represent movies, and their links represent that two movies have common actors.
    The number of two movies' common actors is encoded by the link thickness in both diagrams.
    Their visual encodings and layouts differ:
    (a) encodes movies' publication seasons using different node colors; nodes are placed by a force-directed layout.  
    The nodes in (b) encode a movie's country, first genre, and its number of votes by shape, color, and size. 
    Nodes are placed with an attribute-based layout whose \textit{x}-coordinate encodes the movie's publication year, and \textit{y}-coordinate encodes movie duration.
    (c) and (d) are descriptions generated by our approach, each with three parts interpreting different steps in the creation process. }
  \label{fig:BasicCases}
\end{figure*}

\subsection{Case 1: Simple Nodes and Links}\label{sec:imdb_movies}
% 我们挑选了数据集中2019年在中国上映的电影。

We select movies in the dataset which are published in 2019, China and preserve seven movie attributes: \texttt{"title"}, \texttt{"date\_published"} (the publication date), \texttt{"actors"}, \texttt{"genre"}, \texttt{"director"}, \texttt{"writer"}, and \texttt{"country"}.

\noindent \textbf{1) Graph wrangling}. 
Two movies are connected if they have at least one same actor.
We generate an attribute, \texttt{"common\_actors"} representing the number of common actors for each link. 
We obtain 49 nodes and 99 links in total.

\noindent \textbf{2) Visual encoding}. 
Nodes and links are visualized as \texttt{<circle>}s and \texttt{<line>}s separately (Figure~\ref{fig:BasicCases} (a)).
The fill color of a \texttt{<circle>} encodes it publication season (e.g., when \texttt{"date\_published"} is \texttt{01}, \texttt{02}, or \texttt{03}, it means the publication season is spring and we fill the node with \texttt{steelblue}).
The width of \texttt{<line>} encodes the attribute \texttt{"common\_actors"}.

\noindent \textbf{3) Layout computing}. 
We pre-compute a layout with a spring layout algorithm~\cite{DBLP:journals/spe/FruchtermanR91} to reveal movies' relationships.

\textbf{Results}.
Descriptions are generated for the created node-link diagram (Figure~\ref{fig:BasicCases} (c)).
They can be divided into three parts:

\noindent \textbf{1) Interpreting Linking Conditions.} 
Several conditions (e.g. (\texttt{"C2"}, \texttt{"country"}, \texttt{"China"}) and (\texttt{"C1"}, \texttt{"year"}, \texttt{"2019"}), etc.) are detected but only the condition (\texttt{"C2"}, \texttt{"actors"}, \texttt{"[arbitrary value]"}) is preserved because other conditions also hold on unconnected node pairs.
Thus \ApproachName~describes the linking condition as \textit{``Two nodes are connected if their {\texttt{"actors"}} have common values''}.

\noindent \textbf{2) Interpreting Visual Encodings.} 
All visual encodings are detected by \ApproachName~as expected.
However, our technique does not detect that the color of a node encodes the publication season because ``season'' is an abstract concept that GraphDescriptor does not consider.
However, it generates the mapping of different color categories: 
\textit{``When the value of the attribute {\texttt{"date\_published[month]"}} is {\texttt{04}}, {\texttt{05}}, or {\texttt{06}}, 
its {\texttt{fill}} is changed into {\texttt{darkorange (\#ff7f0e)}}.$\ldots$''}, 
where \texttt{"date\_published[month]"} is the \texttt{"month"} field of the attribute publication date (\texttt{"date\_published"}).
Similar descriptions are generated to describe four different \texttt{fill} colors with four seasons of month, so that the audience learns that the fill color encodes the ``season''.

\noindent \textbf{3) Interpreting Layout Intention.} \ApproachName~detects how the nodes are placed in a topology-based layout. 
Detection of a specific layout type is not supported in our technique.
Thus, \ApproachName~only describes the meaning of the topology-based layout in Figure~\ref{fig:BasicCases} (c) (Layout Intention): \textit{``the farther the topology distance between two nodes, the farther the distance between them''}.

\subsection{Case 2: Nodes with Different Shapes}
This case demonstrates the generated descriptions where the shape of a node is used to encode a categorical attribute (Figure~\ref{fig:BasicCases} (b) and (d)).
We select movies directed by Jean-Pierre Melville. 

\noindent \textbf{1) Graph Wrangling}. 
Movies are connected if they have common actors. 
We finally obtain 14 nodes (movies) and 25 links.

\noindent \textbf{2) Visual Encodings}.
Movies directed by Melville were mostly published in France; through several appeared in Italy.
We use circular nodes to encode movies published in France and square nodes to encode movies published in Italy.
The size encodes the number of votes (\texttt{"votes"}) they have received.
The color encodes their primary genre (the first genre in their genre attribute \texttt{"genre"}).

\noindent \textbf{3) Layout Computing}.
The x and y coordinates encode the publication year (\texttt{"year"}) and the \texttt{"duration"} separately.

\textbf{Results}.
Descriptions are generated for the created node-link diagram (Figure~\ref{fig:BasicCases} (d)).
Unlike the former case, nodes are visualized with two different elements: \texttt{<circle>} and \texttt{<rect>}.
Their visual channels have differences, and so \ApproachName~describes them separately:
\textit{``Its \texttt{tagName} encodes the attribute country.
When the \texttt{"country"} is \texttt{France}, its \texttt{tagName} is changed into \texttt{<circle>}.
Its \texttt{radius} encodes the attribute \texttt{"votes"}.
When the \texttt{"country"} is \texttt{Italy}, its \texttt{tagName} is changed into \texttt{<rect>}.
Its \texttt{width} and \texttt{height} encode the attribute \texttt{"votes"}''}.

Another difference is that the layout is attribute-based.
Our technique detects the meanings of two coordinates by testing correlations between node positions and attributes: 
\textit{``The x-coordinate encodes the attributes \texttt{"year"}. 
The greater the \texttt{"year"}, the greater the x-coordinate.
The y-coordinate encodes the attributes \texttt{"duration"}.
The greater the \texttt{"duration"}, the greater the y-coordinate.''}.


\begin{figure*}[ht]
    \centering
    \includegraphics[width=2\columnwidth]{figures/LinkBarsCase.eps}
    \caption{ (a) A node-link diagram following the visual design proposed by Sch{\"{o}}ffel et al.~\cite{DBLP:conf/iv/SchoffelSE16}. Nodes represent movies directed by Christopher Nolan, and links are constructed if differences between two movies' publication years are less than 5. We employ the force-directed layout. Four bars on a link encode four attributes: the difference of their budgets (the red bar, \texttt{"budget\_diff"}), the difference of their duration (the yellow bar, \texttt{"duration\_diff"}), the difference of numbers of their votes (the blue bar, \texttt{"votes\_diff"}), and the difference of their average vote scores (the green bar, \texttt{"avg\_vote\_diff"}). (b) Descriptions generated by \ApproachName. Both the attribute-based layout and the topology-based layout are detected. Because we connect movies with similar publication years, movies are laid out from the upper-right corner to the lower-left corner according to the order of publication years. Their budgets also increase over time from the upper-right corner to the lower-left corner.}
    \label{fig:LinkBarsCase}
\end{figure*}

\subsection{Case 3: Line Charts Nested on Nodes}
With the IMDb movie dataset, we introduce another case that focuses on actors that demonstrates more complex situations in which each node has multiple elements (Figure~\ref{fig:NodeLineChartsCase} (b) and (f)).

\noindent \textbf{1) Graph Wrangling}. 
We first select movies in the dataset that are only released only in China from 2016 to 2021.
We select actors who have acted in more than five movies as nodes.
Each actor has five attributes: \texttt{"name"}, \texttt{"movies"} (movies s/he acted in from 2016 to 2021), \texttt{"avg\_vote"} (the average vote), \texttt{"votes"} (the number of votes s/he got), and \texttt{"number\_of\_movies\_by\_year"} (the number of movies in each year from 2016 to 2021).
Two actors are connected if they acted in at least one movie.
Finally, we obtain 17 nodes and 55 links.

\noindent \textbf{2) Visual Encodings}. 
Each node (actor) contains an attribute named \texttt{"number\_of\_movies\_by\_year"} that has five properties: 2016, 2017, 2018, 2019, and 2020.
Each property stores the number of movies the actor acted in that year.
Following the node-link diagram created by Junker et al.~\cite{DBLP:journals/bmcbi/JunkerKS06}, 
we nest a simple line chart into each node to show the number of movies the actor acted in from 2016 to 2021 (Figure~\ref{fig:NodeLineChartsCase} (a)). 
The width of the link's \texttt{<line>} encodes its attribute \texttt{"number\_of\_shared\_movies"}.

\noindent \textbf{3) Layout Computing}. We utilize the layout to show two attributes (\texttt{"avg\_vote"} and \texttt{"votes"}) of each node. The x-coordinate encodes the attribute \texttt{"votes"} and the y-coordinate encodes the \texttt{"avg\_vote"}.

\textbf{Results}. 
Unlike Section~\ref{sec:imdb_movies}, here each node has multiple elements in this case.
We bind elements to different entities and distinguish their roles.
We interpret the meaning of the line chart by describing its elements.
As described in the Visual Encodings part of Figure~\ref{fig:NodeLineChartsCase} (b), it contains five elements.
The first four elements are \texttt{<line>}s.
% 它们组合在一起编码了每个演员每年出演电影数量的趋势,
They are composed together to encode the actors' activeness, namely the number of movies they acted in each year.
Descriptions like \textit{``encode the actor' activeness''} are domain-knowledge based; our technique generates fine-grained descriptions to help audiences form complex connections.
% 于是我们的方法对它们的y值分别进行了描述:
We describe them separately: 
\textit{``The first element is a \texttt{<line>}. 
Its \texttt{y1} encodes the attribute \texttt{"number\_of\_movies\_by\_year[2019]"}. 
The greater the \texttt{"number\_of\_movies\_by\_year[2019]"}, the greater its \texttt{y1}. 
Its \texttt{y2} encodes the attribute \texttt{"number\_of\_movies\_by\_year[2020]"}.
The greater the \texttt{"number\_of\_movies\_by\_year[2020]"}, the greater its \texttt{y2}.''}
Then the description is repeated with different elements and attributes.
Such descriptions are fine-grained and can be composed to form complex tasks in the future.


\subsection{Case 4: Bars on Links}
Case 4 demonstrates the generated descriptions for links.
The design of the node-link diagram follows Sch{\"{o}}ffel et al.~\cite{DBLP:conf/iv/SchoffelSE16}, links are encoded with several bars to display link attributes.

\noindent \textbf{1) Graph Wrangling}.
We select movies directed by Christopher Nolan.
Attributes \texttt{"budget"}, \texttt{"duration"}, \texttt{"votes"}, \texttt{"avg\_vote"}, \texttt{"director"}, \texttt{"title"}, \texttt{"year"}, and \texttt{"date\_published"} are preserved on nodes.
Two nodes are connected if the difference between their \texttt{"year"}s is less than 5.
We calculate several attributes for links to show the difference of their end nodes, such as the difference in \texttt{"budget"}s, \texttt{"duration"}, \texttt{"votes"}, and \texttt{"avg\_vote"}.

\noindent \textbf{2) Visual Encodings}.
The radius of the node encodes its \texttt{"avg\_vote"}.
Widths of four bars on a link encode the difference of \texttt{"budget"}s (\texttt{"budget\_diff"}), \texttt{"duration"}s (\texttt{"duration\_diff"}), \texttt{"votes"}s (\texttt{"votes\_diff"}), and \texttt{"avg\_vote"}s (\texttt{"avg\_vote\_diff"}) respectively (Figure~\ref{fig:LinkBarsCase} (a)).
Their colors are used merely to distinguish them rather than encode attribute values.

\noindent \textbf{3) Layout Computing}.
A force-directed layout is employed.

\textbf{Results}.
As expected, four \texttt{<rect>}s are discussed separately (the Visual Encodings part in Figure~\ref{fig:LinkBarsCase} (b)):
\textit{``The first element is a \texttt{<rect>}. Its \texttt{width} encodes the attribute \texttt{"budget\_diff"}. The greater the attribute \texttt{"budget\_diff"} is, the greater its \texttt{width} is.''}
Descriptions of the other three \texttt{<rect>}s are similar.
Although we only compute a force-directed layout for the graph, our technique suggests that both the attribute-based layout and the topology-based layout are detected (Figure~\ref{fig:LinkBarsCase} (b)).
This is because we connect movies with similar years and the force-directed layout coincidentally places the earliest movie in the upper-right corner and the latest movie in the lower-left corner.
Earlier movies have smaller \texttt{"year"}s, \texttt{"date\_published[year]"}s (the year of publication), and \texttt{"budget"}s.
Thus, the layout is also attribute-based.
Our technique helps to obtain this impressive finding.