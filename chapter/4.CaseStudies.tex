\section{Case Studies}
We conduct four case studies to demonstrate descriptions generated by \ApproachName.
The IMDb movie dataset~\footnote{\url{https://www.imdb.com/}} has been used in plenty of works~\cite{DBLP:journals/tvcg/SrinivasanPEB18, DBLP:journals/pvldb/KrishnanWWFG16, DBLP:conf/ieeevast/BigelowNML19}.
We employ it to conduct our case studies.


\subsection{Case 1: Simple Nodes and Links}\label{sec:imdb_movies}
The result of Case 1 is available in Figure~\ref{fig:BasicCases} (a) and (c).
This case is aimed to show a basic example.
% 我们挑选了数据集中2019年在中国上映的电影。
\textbf{1) Graph Wrangling}. We select movies in the dataset which is published in 2019 and is released in China.
% 我们保留了其中标题、发布日期、演员、类别、
We preserve seven attributes for movies: \texttt{"title"}, the publish date (\texttt{"date\_published"}), \texttt{"actors"}, \texttt{"genre"}, \texttt{"director"}, \texttt{"writer"}, and \texttt{"country"}.
Two movies are connected if they share at least one same actor.
We generate an attribute, \texttt{"shared\_actors"}, for each link. 
It records the number of shared actors.
We totally obtain 49 nodes and 99 links.
\textbf{2) Visual Encodings}. 
Nodes are visualized as \texttt{<circle>}s and links are visualized as \texttt{<line>}s (Figure~\ref{fig:NodeLineChartsCase} (a)).
The fill colors of \texttt{<circle>}s encode their publish seasons (e.g., when \texttt{"date\_published"} is \texttt{01}, \texttt{02}, or \texttt{03}, it means the publish season is spring, we fill the node with \texttt{"steelblue"}).
The width of \texttt{<line>}s encode the \texttt{"shared\_actors"}.
\textbf{3) Layout Computing}. 
We pre-compute a layout with a spring layout algorithm~\cite{DBLP:journals/spe/FruchtermanR91}.

\textbf{Results}.
The descriptions are generated for the created node-link diagram (Figure~\ref{fig:NodeLineChartsCase} (e)).
They can be separated into three parts:

\textbf{1) Interpreting Linking Conditions.} Several conditions (e.g. (\texttt{"C2"}, \texttt{"country"}, \texttt{"China"}), (\texttt{"C1"}, \texttt{"year"}, \texttt{"2019"}), etc.) are detected but only the condition (\texttt{"C2"}, \texttt{"actors"}, \texttt{"[arbitrary value]"}) is preserved because other conditions also are held on unconnected node pairs.
Thus \ApproachName~describes the linking condition as \textit{``Two nodes are connected if their attributes {\texttt{actors}} have common values''}.

\textbf{2) Interpreting Visual Encodings.} All visual encodings are detected by \ApproachName~as expected.
Whereas, our technique does not detect that the color of each node encodes the publish season because ``season'' is an abstract concept that are not under our consideration.
However, it generates the mapping of different color categories: 
\textit{``When the value of the attribute {\texttt{date\_published[month]}} is {\texttt{04}}, {\texttt{06}}, or {\texttt{05}}, 
its {\texttt{fill}} turns to {\texttt{darkorange (\#ff7f0e)}}.$\ldots$''}, 
where \texttt{"date\_published[month]"} means the \texttt{"month"} field of the attribute publish date (\texttt{"date\_published"}).
Such descriptions are repeated generated to describe four different \texttt{fill} colors.
Such that audience can obtain that the fill color encodes the ``season''.

\textbf{3) Interpreting Layout Meanings.} \ApproachName~detects the nodes are placed with a topology-based layout. 
Detection of the specific type of the layout is not supported in our technique.
Thus, \ApproachName~only describes the meaning of the topology-based layout in Figure~\ref{fig:NodeLineChartsCase} (e) (Layout Meanings): \textit{``the greater the graph geodesic distance between two nodes is, the greater the euclidean distance between them is''}.

Such descriptions can also facilitate code debugging. 
For example, if the creator negligently encodes the \texttt{fill} color with its publish date,
our technique will detect it and generate descriptions as: \textit{``Its \texttt{fill} encodes the attribute \texttt{date\_published[day]}''}.

\begin{figure*}[ht]
    \centering
    \setlength{\belowcaptionskip}{-5pt}
    \includegraphics[width=2\columnwidth]{figures/NodeLineChartsCase.eps}
    \caption{xxx}
    \label{fig:NodeLineChartsCase}
\end{figure*}

\subsection{Case 2: Nodes with Different Shapes}
Shapes (\texttt{tagName}s) of primitives are often used to encode categorical attributes.
This case demonstrates what descriptions will be generated with \ApproachName~where the shape of a node is used to encode a categorical attribute (Figure~\ref{fig:BasicCases.eps} (b) and (d)).

\textbf{1) Graph Wrangling}. 
We select movies directed by Jean-Pierre Melville, a French director. 
Movies are connected if they have common actors. 
We finnally obtain 14 nodes (movies) and 25 links.

\textbf{2) Visual Encodings}.
Movies directed by Melville were most published in France, several of them were published in Italy.
We use square nodes to encode movies published in Italy and circular nodes to encode movies published in France.
The size encodes the number of votes (\texttt{"votes"}) they have received.
The color encodes their main genre (the first genre in their genre attribute \texttt{"genre"}).

\textbf{3) Layout Computing}.
The x and y coordinates encode the publish year (\texttt{"year"}) and the \texttt{"duration"} separately.

\textbf{Results}.
Different from the former case, nodes are visualized with two different primitives: \texttt{<circle>} and \texttt{<rect>}.
Their visual channels have differences.
Thus, \ApproachName~describes them separatly:
\textit{``Its \texttt{tagName} encodes the attribute country.
When the \texttt{"country"} is \texttt{France}, its \texttt{tagName} turns to \texttt{<circle>}.
When the \texttt{"country"} is \texttt{Italy}, its \texttt{tagName} turns to \texttt{<rect>}.
When it turns to \texttt{<circle>}, 
its \texttt{r} encodes the attribute \texttt{"votes"}.
When it turns to \texttt{<rect>}, its \texttt{width} and \texttt{height} encode the attribute \texttt{"votes"}.''}.

Another difference is that the layout is attribute-based.
Our technique detects the meanings of two coordinates by testing correlations between node positions and attributes: 
\textit{``The \texttt{x}-coordinate encodes the attributes \texttt{year}. 
The greater the \texttt{year} is, the greater the x-coordinate is.
The \texttt{y}-coordinate encodes the attributes \texttt{duration}.
The greater the \texttt{duration} is, the greater the \texttt{y}-coordinate is.''}.

\subsection{Case 3: Line Charts Embeded on Nodes}
With the IMDb movie dataset, we introduce another case which focuses on actors (Figure~\ref{fig:NodeLineChartsCase} (b) and (f)).
This case focuses on demonstrating more complex situations where each node consists of multiple primitives.

\textbf{1) Graph Wrangling}. We first select movies in the dataset that are only released in China from 2016 to 2021.
Their actors who have acted in more than five movies are regarded as nodes.
Each actor has five attributes: \texttt{"name"}, movies s/he acted in from 2016 to 2021 (\texttt{"movies"}), the average vote (\texttt{"avg\_vote"}), the number of votes s/he got (\texttt{"votes"}), the number of movies in each year from 2016 to 2021 (\texttt{"number\_of\_movies\_by\_year"}).
Two actors are connected if they acted in at least one movie from 2016 to 2021.
Finally we obtain 17 nodes and 55 links.

\textbf{2) Visual Encodings}. 
Each node (actor) contains an attribute named \texttt{"number\_of\_movies\_by\_year"}, which has five properties: 2016, 2017, 2018, 2019, and 2020.
Each property stores the number of movies the actor acted in that year.
Following the node-link diagram created by Junker et al.~\cite{DBLP:journals/bmcbi/JunkerKS06}, 
we embed a simple line chart for each node to show the number of movies s/he acted in from 2016 to 2021 (Figure~\ref{fig:NodeLineChartsCase} (a)). 
The width of the link's \texttt{<line>} encodes its attribute \texttt{"number\_of\_shared\_movies"}.

\textbf{3) Layout Computing}. We utilize the layout to show two attributes (\texttt{"avg\_vote"} and \texttt{"votes"}) of each node. The x-coordinate encodes the attribute \texttt{"votes"} and the y-coordinate encodes the \texttt{"avg\_vote"}.

Different with Section~\ref{sec:imdb_movies}, in this case, each node consists of several primitives.
We bind primitives to different entities and distinguish their roles.
We interpret the meaning of the line chart by describing its primitives.
As described in the Visual Encodings part in Figure~\ref{fig:NodeLineChartsCase} (b), it contains five primitives.
The first four primitives are \texttt{<line>}s.
% 它们组合在一起编码了每个演员每年出演电影数量的趋势,
They are composed together to encode the trend of the actors activeness, namely the number of movies they acted in each year.
Descriptions like \textit{``encode the trend of the actors activeness''} are domain-knowledge based,
% 我们的技术旨在通过生成一些细粒度的描述,帮助观众形成更复杂的认知。
our technique aims to generate fine-grained descriptions to help audiences to form complex cognitions.
% 于是我们的方法对它们的y值分别进行了描述:
It describes them separately: 
\textit{``The first primitive is a \texttt{<line>}. 
Its \texttt{y1} encodes the attribute \texttt{"number\_of\_movies\_by\_year[2019]"}. 
The greater the \texttt{"number\_of\_movies\_by\_year[2019]"} is, the greater its \texttt{y1} is. 
Its \texttt{y2} encodes the attribute \texttt{"number\_of\_movies\_by\_year[2020]"}.
The greater the \texttt{"number\_of\_movies\_by\_year[2020]"} is, the greater its \texttt{y2} is.''}
Then the description are repeated with different primitives and attribtues.
Such descriptions are fine-grained and can be composed to face complex tasks in the future.


\subsection{Case 4: Bars on Links}
Case 4 aims to demonstrate the generated descriptions for links.
The design of the node-link diagram follows Sch{\"{o}}ffel et al.~\cite{DBLP:conf/iv/SchoffelSE16}, links are encoded with several bars to display link attributes.

\begin{figure*}[ht]
    \centering
    \includegraphics[width=2\columnwidth]{figures/LinkBarsCase.eps}
    \caption{xxx}
    \label{fig:LinkBarsCase}
\end{figure*}

\textbf{1) Graph Wrangling}.
We select all movies directed by Christopher Nolan for this case.
Attributes such as \texttt{"budget"}, \texttt{"duration"}, \texttt{"votes"}, \texttt{"avg\_vote"}, \texttt{"director"}, \texttt{"title"}, \texttt{"year"}, and \texttt{"date\_published"} are preserved on nodes.
Two nodes are connected if the difference betwee their \texttt{"year"}s is less than 5.
We calculate several attributes for links to show the difference of their end nodes such as difference of \texttt{"budget"}s, \texttt{"duration"}, \texttt{"votes"}, and \texttt{"avg\_vote"}.

\textbf{2) Visual Encodings}.
The radius of the node encodes its \texttt{"avg\_vote"}.
Heights of four bars on a link encode the difference of \texttt{"budget"}s (\texttt{"budget\_diff"}), \texttt{"duration"}s (\texttt{"duration\_diff"}), \texttt{"votes"}s (\texttt{"votes\_diff"}), and \texttt{"avg\_vote"}s (\texttt{"avg\_vote\_diff"}) respectively (Figure~\ref{fig:LinkBarsCase} (a)).
Their colors are only used to distinguish them rather than encode attribute values.

\textbf{3) Layout Computing}.
The force-directed layout is emplpoyed.

As expected, four \texttt{<rect>}s are discussed separately (the Visual Encodings part in Figure~\ref{fig:LinkBarsCase} (b)):
\textit{``The first primitive is a \texttt{<rect>}. Its \texttt{width} encodes the attribute \texttt{"budget\_diff"}. The greater the attribute \texttt{"budget\_diff"} is, the greater its \texttt{width} is.
''}
Descriptions for the other three \texttt{<rect>}s are similar.
Although we only compute a force-directed layout for the graph, our technique suggests that both the attribute-based layout and the topology-based layout are detected (Figure~\ref{fig:LinkBarsCase} (b)).
It is because we connect movies with similar years and the force-directed layout coincidently places the earliest movie in the upper-right corner and positions the latest movie in the lower-left corner.
Earlier movies have smaller \texttt{"year"}s, \texttt{"date\_published[year]"}s (the year of publication), and \texttt{"budget"}s.
Thus, the layout is also attribute-based.
Our technique help us obtain such an interesting finding.