\firstsection{Introduction}
\maketitle

% 图是一种广泛使用的数据结构,在金融、社交、生物等领域都扮演着重要的角色。
Graphs, also called networks, are ubiquitous over numerous areas. For example, graphs can model financial transaction behaviors, social media contacts, biological processes such as protein-protein interactions, et cetera.
% 节点链接图被广泛应用于对图数据进行可视化,可以用来揭示数据实体之间的关系。
Node-link diagrams are widely used to visualize graphs to reveal the connections and relationships among data entities.
% 很多研究也会将图数据中的属性编码到节点链接图上,以展示属性相关的pattern;
Numerous researches encode attributes of the underlying graph into the node-link diagram (e.g., the shape, size, and color of a node encode three different attributes in Figure~\ref{fig:BasicCases} (b)) to reveal attribute-based patterns.
% 在这种复杂的情况下,为节点链接图生成标题来说明创作过程能够增加节点链接图的comprehensibility
This type of node-link diagrams is much more difficult to understand compared to those network topology visualizations. 
Thus, generating descriptions for node-link diagrams to demonstrate the creating process can improve its comprehensibility.
% Generating a caption to summarize a node-link diagram is crucial to enhance its accessibility.
% However, the automatic caption generation has not been explored yet.

% 为可视化生成标题,或者更宽泛地讲,语言描述,在其他类型的可视化研究领域已经被视作是一个重要的课题。
Generating textual descriptions is an important topic in visualizing basic charts such as line charts, bar charts, et cetera.
% 为可视化生成语言描述有很多好处,它的主要目的是为了帮助观众理解可视化所传达的意义。
Generating a description to summary a visualization facilitates its audience to understand the underlying meaning.
% 根据生成的描述,可以将这类工作分成两类,一类描述了可视化的基本构成(比如数据本身,可视映射等),另一类则关注于可视化中蕴含的insight。
Automatic description generation techniques can be roughly summarized into two categories~\cite{DBLP:conf/inlg/ObeidH20}: one describes constituents and visual encodings of a  visualization~\cite{DBLP:journals/coling/MittalMCR98, DBLP:journals/tochi/FerresLST13}, and the other describes high-level insights conveyed by the visualization~\cite{DBLP:conf/apvis/LiuXHWY20, DBLP:conf/inlg/ObeidH20}.
% 类比Hullman等人对annotation的分类,自动描述的这两类工作也可以被总结为: (observational) and (additive)
By analogy with the taxonomy of visualization annotation techniques~\cite{DBLP:conf/chi/HullmanDA13}, descriptions generated by the above two categories of techniques can also be categorized as observational descriptions and additive descriptions.
However, generating descriptions for node-link diagrams have not been explored yet.
To date both two kinds of techniques are unable to deal with node-link diagrams.
% We focus on the former category, which generates descriptions about node-link diagram constituents.

% 第一类描述可视化本身的工作,因为其处理的对象比较明确和简单。
Constituents such as axes, bars, legends, etc. of basic charts that the observational techniques target to deal with are usually well defined.
For example, the automatic caption generation technique proposed by Mittal et al.~\cite{DBLP:journals/coling/MittalMCR98} requires pre-defined visual mapping relationships.
% 一些工作依赖于提前定义好的mapping关系,更多的抽取信息的工作则需要检测文本信息来恢复数据和编码。
More techniques~\cite{DBLP:conf/icip/ZhouT00, DBLP:conf/doceng/HuangT07, DBLP:conf/grec/HuangTL03} that extract information to understand visualizations rely on detecting the text information from axes and legends.
% 节点链接图的复杂性更高,且往往不会带有坐标轴,因为visual clutter的问题,有时候甚至都不会生成相关的标签。
There are usually no axes or legends in node-link diagrams.
Due to the visual clutter caused by text labels, they are usually hidden from node-link diagrams.
% 一些节点链接图因为编码属性的原因,甚至引入了更加复杂的glyph设计,比如嵌入基础图表到节点链接图中。
In order to encode attributes, a plenty of researches introduce complex glyphs into node-link diagrams, such as embedding basic charts into nodes~\cite{gehlenborg2010visualization, DBLP:conf/iv/JusufiDK10} and customizing shapes of links~\cite{DBLP:conf/iv/SchoffelSE16, DBLP:journals/tvcg/NielsenJBJ09}.
% 这增加了对节点链接图进行描述的难度。
These obstacles increase the difficulty of describing node-link diagrams.


% 第二类工作对insight进行描述,它们的工作大多基于背后的表格型数据。
Techniques generate additive descriptions such as insights are mainly based on the underlying tabular data.
% 它们的insight一般是已经提前定义好的几类诸如最大值,均值,趋势等。
Their insights are usually pre-defined (e.g., the maximum value, the mean value, trends, et cetera)~\cite{DBLP:journals/pvldb/DemiralpHPP17, DBLP:journals/tvcg/WangSZCXMZ20} or related to the training data~\cite{DBLP:conf/apvis/LiuXHWY20, DBLP:conf/chi/KimHA20}.
% 本身节点链接图背后的图数据,相比于基础图表背后的表格型数据更具复杂性
Compared to the tabular data behind these charts, the underlying graph data behind node-link diagrams are usually more complex because it contains relationships (links) between table rows (nodes).
% 图数据蕴含的insight往往跟拓扑结构相关,其需要描述的insight也往往需要人的domain knowledge。
% 在不同领域,不同的拓扑结构有不一样的关注度。
Insights of graph data are mainly related to topologies.
Describing graph insights require the domain knowledge because audiences from different domains prefer different kinds of structures.
% 更好的解决方案是对可视化本身进行描述,辅助那些阅读可视化的观众理解可视化,基于他们自身的知识来形成个人不同的insight更有意义。
It is more meaningful to generate observational descriptions for node-link diagrams.
Observational descriptions only assist audiences to comprehend visualizations according to their individual knowledge rather than directly suggest pre-defined insights.


% % 其中第二类工作生成的insight大多是基于模板,或是跟训练数据相关的,其自由度会受到这些因素的限制。
% Insights generated by the latter are mostly template-based, or related to the training data, thus will be limited by such factors.
% Generating insights beyond templates and the training data is challenging.
% They focus on basic charts and statistical charts.

% % 所以第二类工作反而会导致生成的描述不够准确或者不够完备。
% Therefore, generating insights for node-link diagrams is more challenging and
% may be inaccurate or incomplete, which in turn impacts the analysis of graphs.
% % 第一类工作对于解释可视化本身很有意义,而insights则由用户自己决定,这会给予用户更多的分析自由度。
% However, the former category of techniques only describe the visualization itself, while insights are decided by audiences, thus reach a higher degree of analysis freedom.
% % 我们的方法只描述节点链接图的创作者所做的事情,而不去限制用户的思考和分析。
% We only focus on generating objective descriptions about node-link diagrams and do not limit audiences' thinking and analysis.


% 我们提出了xxxx,已解决上述挑战。
We contribute~\textit{\ApproachName}, an automatic description generator for node-link diagrams.
It aims to extract key information from the creation process of node-link diagrams to solve challenges mentioned above.
The general steps of creating a node-link diagram are summarized into three steps~\cite{DBLP:journals/cgf/SpritzerBDFF15, tvcg/RomatAP21}:
\nointent \textbf{1) Graph wrangling} transforms the original data from tabular format to graph format by defining relationships (links) between entities (nodes).
\nointent \textbf{2) Visual encoding} encodes node/link attributes with different visual channels to show data features. 
\nointent \textbf{3) Layout computing} positions nodes in a two-dimensional space to reveal attribute-based or topology-based patterns~\cite{DBLP:journals/cgf/NobreMSL19}.
\ApproachName~first checks the linking conditions between all pairs of nodes, and selects potential conditions to explain the process of \textbf{graph wrangling}.
Then it continuously modifies the data and checks changes of the visualization result to find encoding schemes for the explanation of the \textbf{visual encoding} step.
And \ApproachName~interprets the \textbf{layout computing} step by examining the influence factors of node positions to identifying the layout category.
Information is filled into several pre-defined templates to generate textual descriptions.
The textual descriptions are enhanced to be interactive by highlighting their corresponding parts of the diagram.
We implement an interface for creators to generate such interactive descriptions.
% 我们通过xxx对该方法的有效性进行了验证。
Four cases and a user study demonstrates the usability and effectiveness of our approach.


\deleted[id=pan]{
%! challenges
% 为节点链接图生成基于constituent的描述,需要对其创建过程进行解构。
Generating such constituent-based descriptions requires decomposing the creation process of node-link diagrams.
% 一个节点链接图的创作过程普遍会有以下X个步骤(其中第2步和第3步的顺序可以调换)[这里需要引一些文献]
% 1. 数据预处理;一般而言,真实世界收集到的源数据常常是表格型的多属性数据,而输入到节点链接图创作的图数据,往往是在源数据的基础上进行预处理,通过在源数据上定义实体之间的关系(链接),最终获得输入到节点链接图的图数据(包含节点和链接)
% 2. 设计可视编码;创作者还会把他想表达的数据特征编码到节点链接图上,比如利用节点的颜色来编码节点的类别等。
% 3. 计算图布局;节点链接图需要解决的一个关键问题是如何在二维空间中放置节点。链接则往往被绘制为节点之间的连接线,其位置往往由其两个端点决定。基于布局算法产生的节点位置,节点链接图才能进行绘制。
% 创建节点链接图的一般步骤被归纳为以下三步:
The general steps of creating a node-link diagram are summarized into three steps~\cite{DBLP:journals/cgf/SpritzerBDFF15, tvcg/RomatAP21}:

\nointent \textbf{1) Graph wrangling} transforms the original data from tabular format to graph format by defining relationships (links) between entities (nodes).
\nointent \textbf{2) Visual encoding} encodes node/link attributes with different visual channels to show data features. 
\nointent \textbf{3) Layout computing} positions nodes in a two-dimensional space to reveal attribute-based or topology-based patterns~\cite{DBLP:journals/cgf/NobreMSL19}.

% 三个步骤都包含了关键信息需要创建者向用户进行解释。
% 我们的方法旨在抽取这些信息生成相关描述来帮助用户克服对这三个步骤的理解障碍。
% 通过抽取信息的方式来生成描述会遇到以下挑战:
All three steps contain key information that needs to be explained to audiences.
We aim to extract information and generate relevant descriptions to help users overcome the obstacles of understanding the three steps.
Several challenges are encountered:

Numerous researches regard the construction of links as the crucial step of graph wrangling~\cite{DBLP:journals/tvcg/SrinivasanPEB18, DBLP:conf/ieeevast/BigelowNML19, DBLP:journals/ivs/HeerP14, DBLP:journals/ivs/LiuNS14}.
It is important to inform audiences how relationships are constructed to form links.
Although Graphiti~\cite{DBLP:journals/tvcg/SrinivasanPEB18} is designed to infer linking conditions from user demonstrations, it can not be employed directly to infer linking conditions in a graph with all links constructed.
% 需要一个新的方法,能够处理多条边之间的条件的逻辑关系,来得出完备的结论。
A new pipeline is needed to deal with multiple linking conditions to reach a complete conclusion.

\noindent \textbf{2)} % 从可视化结果中提取 visual mapping 的工作也有很多,但他们大多为了处理一些基础图表,缺少对于节点链接图的支持,又或者需要强有力的先验假设(比如需要d3将数据绑定到可视化元素上),使得工作难以拓展到更多领域。我们的方法将对已有方法进行补充,使他们能够适用于更广的范围,能够良好支持节点链接图中的视觉映射的抽取。
Numerous techniques are proposed to extract visual mappings from visualization~\cite{DBLP:conf/uist/HarperA14, DBLP:journals/tvcg/HoqueA20, DBLP:journals/corr/abs-2103-00741},  they mainly aim at basic charts such as line charts, bar charts, et cetera.
They rarely support node-link diagrams or are not robust enough to deal with complex conditions, such as embedding line charts into nodes~\cite{DBLP:journals/bmcbi/JunkerKS06} or visualizing links as bars~\cite{DBLP:conf/iv/SchoffelSE16}).

\noindent \textbf{3)} Describing a layout should determine the meaning of the layout.
% 虽然已经有很多布局相关的工作,但几乎没有工作能够推测某种布局结果的意义。
Although numerous layout algorithms are proposed~\cite{hachul2004drawing, DBLP:journals/spe/FruchtermanR91, DBLP:conf/gd/GansnerKN04, DBLP:conf/gd/BrandesP06, DBLP:journals/tvcg/ZhuCHHLZ21}, none of them can be used to infer the meaning of a specific layout.
}