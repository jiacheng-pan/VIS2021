\firstsection{Introduction}
\maketitle

% 图是一种广泛使用的数据结构,在金融、社交、生物等领域都扮演着重要的角色。
Graphs, also called networks, are ubiquitous over numerous areas. For example, graphs can model the financial transaction behaviors, social media contacts, biological processes such as protein-protein-interactions, and so on.
% 节点链接图被广泛应用于对图数据进行可视化,可以用来揭示数据实体之间的关系。
Node-link diagrams are widely used to visualize graphs to reveal the connections and relationships among data entities.
% 然而,如果没有任何的辅助信息,用户除了能知道拓扑结构以外,无法获得任何其它信息,比如节点链接图的编码,节点和节点之间如何联系等。
Whereas, without any description, audiences cannot obtain any information except plain topological patterns.
{\color{red}{Basic questions}?} about the diagram (e.g., how attributes are encoded, how nodes are connected, etc.) requires explanations.

% 提取信息,形成说明。
Numerous works are proposed to improve comprehensibility of visualizations.
%???

% 【解决方案和动机】
% 我们尝试从文本生成的角度,用标题等文本的形式,来帮助创作者解释其创建的节点链接图蕴含的意义。
% 已经有一些方法,帮助特定的可视化形式自动化生成文本解释。
% \underline{说明一下这些方法的做法。}
% 但针对帮助理解节点链接图的问题,还没有经过系统的探索和明确的定义。
% 本文探索了自动为节点链接图生成相应的解释性文本。

% 【我们的做法和存在的挑战】为了帮助创作者为其创作的节点链接图自动生成解释性文本,我们首先探索了观察节点链接图的观众会对节点链接图产生疑惑的原因。包括:
% \begin{enumerate}
%     \item 对背后的数据本身缺少了解。用户并不知道背后的数据中,节点表达的含义和链接表达的含义。
%     \item 对节点链接图的编码缺少了解。用户并不知道创作者将数据中的属性在节点链接图中,用何种视觉通道进行编码。
%     \item 虽然节点链接图已经能够直观展示节点之间的拓扑关系,但visual clutter使他会难以调查其中的拓扑结构。
%     \item 对布局算法的不了解。观众往往不知道算法的细节,无法体会为什么会形成如此布局。 %? 通过测试是否跟某个属性相关/或者是否跟拓扑距离相关来决定
% \end{enumerate}

% 【贡献】我们从以上几个角度出发,提出了一种\textbf{自动为节点链接图生成解释性文本的方法},它以创作者创作的节点链接图为输入,自动生成解释性文本,以帮助解决以上四个问题。\underline{我们的方法....其主要贡献可以被归纳为以下两点:}

%? 重要:需要提及我们的方法是template-based,重点不在于NLG,而是在于问题的formulation和信息的提取。