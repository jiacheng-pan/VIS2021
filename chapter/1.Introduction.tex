\firstsection{Introduction}
\maketitle

% 图是一种广泛使用的数据结构,在金融、社交、生物等领域都扮演着重要的角色。
Graphs, also called networks, are ubiquitous over numerous areas. For example, graphs can model financial transaction behaviors, social media contacts, biological processes such as protein-protein interactions, etc.
% 节点链接图被广泛应用于对图数据进行可视化,可以用来揭示数据实体之间的关系。
Node-link diagrams are widely used to visualize graphs to reveal the connections and relationships among data entities.
% 很多研究也会将图数据中的属性编码到节点链接图上,以展示属性相关的pattern;
Numerous researches encode attributes of the underlying graph into the node-link diagram to reveal attribute-based patterns.
% 在这种复杂的情况下,为节点链接图生成标题来说明创作过程能够增加节点链接图的accessibility
It is more complex than only visualizing the topology, thus generating descriptions for node-link diagrams to instruct the visualization can improve its comprehensibility.
% 然而,该项工作还缺少相关探索。
However, it has not been explored yet.
% Generating a caption to summarize a node-link diagram is crucial to enhance its accessibility.
% However, the automatic caption generation has not been explored yet.

% 为可视化生成标题,或者更宽泛地讲,语言描述,在其他类型的可视化研究领域已经被视作是一个重要的课题。
Generating textual descriptions has been regarded as an important topic in visualizing basic charts such as line charts, bar charts, etc..
% 为可视化生成语言描述有很多好处,它的主要目的是为了帮助观众理解可视化所传达的意义。
Generating a description to summary a visualization facilitates the audience to understand the underlying meaning.
% 根据生成的描述,可以将这类工作分成两类,一类描述了可视化的基本构成(比如数据本身,可视映射等),另一类则关注于可视化中蕴含的insight。
Automatic description generation techniques can be roughly summarized into two categories~\cite{DBLP:conf/inlg/ObeidH20}: one describes constituents and visual encodings of a visualization~\cite{DBLP:journals/coling/MittalMCR98, DBLP:journals/tochi/FerresLST13} and the other describes high-level insights conveyed by the visualization~\cite{DBLP:conf/apvis/LiuXHWY20, DBLP:conf/inlg/ObeidH20}.
%! 说明一下第二类工作的一些缺陷,以便引出我们目的是第一类工作。
% 对于图可视化而言,两类工作都还没有被探索过,都具有挑战性。
Generating descriptions for node-link diagrams have not been explored yet.
Both two categories of generation techniques are challenging.
We focus on the former category, which generates descriptions about node-link diagram constituents.
% % 其中第二类工作生成的insight大多是基于模板,或是跟训练数据相关的,其自由度会受到这些因素的限制。
% Insights generated by the latter are mostly template-based, or related to the training data, thus will be limited by such factors.
% Generating insights beyond templates and the training data is challenging.
% They focus on basic charts and statistical charts.
% % 本身节点链接图背后的图数据,相比于基础图表背后的表格型数据更具复杂性
% Compared to the tabular data behind these charts, the underlying data behind node-link diagrams are usually more complex.
% % 图数据不仅仅包含了表格型数据本身(节点往往蕴含了多个属性),并且还包括节点之间的拓扑关系。
% A graph not only contains tabular data (multi rows of nodes and multi columns of attributes), but also stores relationships (links) between table rows (nodes).
% % 所以第二类工作反而会导致生成的描述不够准确或者不够完备。
% Therefore, generating insights for node-link diagrams is more challenging and
% may be inaccurate or incomplete, which in turn impacts the analysis of graphs.
% % 第一类工作对于解释可视化本身很有意义,而insights则由用户自己决定,这会给予用户更多的分析自由度。
% However, the former category of techniques only describe the visualization itself, while insights are decided by audiences, thus reach a higher degree of analysis freedom.
% % 我们的方法只描述节点链接图的创作者所做的事情,而不去限制用户的思考和分析。
% We only focus on generating objective descriptions about node-link diagrams and do not limit audiences' thinking and analysis.


%! challenges
% 为节点链接图生成基于constituent的描述,需要对其创建过程进行解构。
Generating such constituent-based descriptions requires decomposing the creation process of node-link diagrams.
% 一个节点链接图的创作过程普遍会有以下X个步骤(其中第2步和第3步的顺序可以调换)[这里需要引一些文献]
% 1. 数据预处理;一般而言,真实世界收集到的源数据常常是表格型的多属性数据,而输入到节点链接图创作的图数据,往往是在源数据的基础上进行预处理,通过在源数据上定义实体之间的关系(链接),最终获得输入到节点链接图的图数据(包含节点和链接)
% 2. 设计可视编码;创作者还会把他想表达的数据特征编码到节点链接图上,比如利用节点的颜色来编码节点的类别等。
% 3. 计算图布局;节点链接图需要解决的一个关键问题是如何在二维空间中放置节点。链接则往往被绘制为节点之间的连接线,其位置往往由其两个端点决定。基于布局算法产生的节点位置,节点链接图才能进行绘制。
% 创建节点链接图的一般步骤被归纳为以下三步:
The general steps of creating a node-link diagram are summarized into three steps~\cite{DBLP:journals/cgf/SpritzerBDFF15, tvcg/RomatAP21}:
\textbf{1) Graph wrangling} transforms the original data from tabular format to graph format by defining relationships between entities~\cite{DBLP:journals/tvcg/SrinivasanPEB18, DBLP:conf/ieeevast/BigelowNML19, DBLP:journals/ivs/HeerP14, DBLP:journals/ivs/LiuNS14}.
% 用描述告知观众实体之间的联系是如何构建的 很关键
It is crucial to inform audiences how those relationships are constructed to form links.
% % 虽然Graphiti~\cite{DBLP:journals/tvcg/SrinivasanPEB18}从用户的demonstration中推测构建链接的条件,但它并不能直接被用于推测已经所有边已经构建完毕的图数据。
% Although Graphiti~\cite{DBLP:journals/tvcg/SrinivasanPEB18} is designed to infer linking conditions from user demonstrations, it can not be employed directly to infer linking conditions in a graph with all links constructed.
% % 需要一个新的方法,能够处理多条边之间的条件的逻辑关系,来得出完备的结论。
% A new pipeline is needed to deal with logical relationships among multiple linking conditions to obtain a complete conclusion.
\textbf{2) Visual encoding} encodes node/link attributes with different visual channels to show data features. 
\textbf{3) Layout computing} positions nodes in a two-dimensional space to reveal attribute-based or topology-based patterns~\cite{DBLP:journals/cgf/NobreMSL19}.

% 三个步骤都包含了关键信息需要创建者向用户进行解释。
% 我们的方法旨在抽取这些信息生成相关描述来帮助用户克服对这三个步骤的理解障碍。
% 通过抽取信息的方式来生成描述会遇到以下挑战:
All three steps contain key informations that needs to be explained to audiences.
We aim to extract these informations and generate relavent descriptions to help users overcome the obstacles of understanding these three steps.
Several challenges are encountered:

\noindent \textbf{1)} 
Describing how relationships are constructed needs to extract information about linking conditions.
% 虽然Graphiti~\cite{DBLP:journals/tvcg/SrinivasanPEB18}从用户的demonstration中推测构建链接的条件,但它并不能直接被用于推测已经所有边已经构建完毕的图数据。
Although Graphiti~\cite{DBLP:journals/tvcg/SrinivasanPEB18} is designed to infer linking conditions from user demonstrations, it is designed to infer linking conditions of a user defined link and can not be employed directly to infer linking conditions in a graph with all links constructed.
% 需要一个新的方法,能够处理多条边之间的条件的逻辑关系,来得出完备的结论。
A new pipeline is needed to deal with logical relationships among multiple linking conditions to obtain a complete conclusion.

\noindent \textbf{2)} % 从可视化结果中提取 visual mapping 的工作也有很多,但他们大多为了处理一些基础图表,缺少对于节点链接图的支持,又或者需要强有力的先验假设(比如需要d3将数据绑定到可视化元素上),使得工作难以拓展到更多领域。我们的方法将对已有方法进行补充,使他们能够适用于更广的范围,能够良好支持节点链接图中的视觉映射的抽取。
Numerous techniques are proposed to extract visual mappings from visualization~\cite{DBLP:conf/uist/HarperA14, DBLP:journals/tvcg/HoqueA20, DBLP:journals/corr/abs-2103-00741},  they mainly aim at basic charts such as line charts, bar charts, and so on.
They rarely support node-link diagrams or are not robust enough to deal with complex conditions (e.g., a node is visualized as a line chart~\cite{DBLP:journals/bmcbi/JunkerKS06}).

\noindent \textbf{3)} Describing a layout should determine the meaning of the layout.
% 虽然已经有很多布局相关的工作,但几乎没有工作能够推测某种布局结果的意义。
Although numerous layout algorithms are proposed~\cite{hachul2004drawing, DBLP:journals/spe/FruchtermanR91, DBLP:conf/gd/GansnerKN04, DBLP:conf/gd/BrandesP06, DBLP:journals/tvcg/ZhuCHHLZ21}, none of them can be used to infer the meaning of a specific layout.

% 我们提出了xxxx,已解决上述挑战。
We introduce~\textit{\ApproachName}, an automatic description generator for node-link diagrams.
It aims to extract key information from the three-steps creation process of node-link diagrams to solve challenges mentioned above.
\ApproachName~first checks the linking conditions between all pairs of nodes, and selects potential conditions to explain the process of \textbf{graph wrangling}.
Then it continuously modifies the source data and checks changes of the visualization result to find encoding schemes for the explanation of the \textbf{visual encoding} step.
And \ApproachName~interprets the meaning of the \textbf{layout computing} step by examining the influence factors of node positions to classifying the layout category.
We fill these informations into several pre-defined templates to generate textual descriptions.
Four case studies are proposed to demonstrate our results.
% 我们通过xxx对该方法的有效性进行了验证。
{\color{red} And we evaluate and discuss ?? of our technique with a user study.}