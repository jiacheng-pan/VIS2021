\firstsection{Introduction}
\maketitle

% 图是一种广泛使用的数据结构,在金融、社交、生物等领域都扮演着重要的角色。
Graphs, also called networks, are ubiquitous over numerous areas. For example, graphs can model the financial transaction behaviors, social media contacts, biological processes such as protein-protein-interactions, and so on.
% 节点链接图被广泛应用于对图数据进行可视化,可以用来揭示数据实体之间的关系。
Node-link diagrams are widely used to visualize graphs to reveal the connections and relationships among data entities.
Generating a caption to summarize a node-link diagram is crucial to enhance its accessibility.
However, the automatic caption generation has not been explored yet.

% 为可视化生成语言描述有很多好处,它的主要目的是为了帮助观众理解可视化所传达的意义。
Generating a caption, or a description, to summary a visualization has numerous benefits.
% 根据生成的描述,可以将这类工作分成两类,一类描述了可视化的基本构成(比如数据本身,可视映射等),另一类则关注于可视化中蕴含的insight。
According to the results, automatic description generation techniques can be roughly summarized into two categories~\cite{DBLP:conf/inlg/ObeidH20}: one describes constituents of a visualization objectively(e.g., visual mappings, underlying data, etc.)~\cite{DBLP:journals/coling/MittalMCR98, DBLP:journals/tochi/FerresLST13} and the other describes high-level insights conveyed by the visualization subjectively~\cite{DBLP:conf/apvis/LiuXHWY20, DBLP:conf/inlg/ObeidH20}.
% 这些工作,基本都是为基础可视化图表生成相关描述
They are mostly aimed at generating descriptions for basic charts such as line charts, bar charts, area charts, and so on.
% 其中第二类工作生成的insight大多是基于模板,或是跟训练数据相关的,其自由度会受到这些因素的限制。
Insights generated by the latter are most template-based, or related to the training data, thus will be limited by such factors.
% 本身节点链接图背后的图数据,相比于基础图表背后的表格型数据更具复杂性
Compared to basic charts, underlying graphs behind node-link diagrams are usually more complex, 
% 所以第二类工作反而会导致生成的描述不够准确或者不够完备。
insights generated may be inaccurate or incomplete, thus influence analysis.
% 第一类工作对于解释可视化本身很有意义,而insights则由用户自己决定,这会给予用户更多的分析自由度。
Whereas, the former category of techniques only describe the visualization itself, while insights are decided by audiences, thus reach a higher degree of analysis freedom.
% 我们的方法只描述节点链接图的创作者所做的事情,而不去限制用户的思考和分析。
We only focus on generating objective descriptions about node-link diagrams and do not limit audiences' thinking and analysis.


To generate such descriptions, we need to decompose the creation process of node-link diagrams to explore the challenges.
% 一个节点链接图的创作过程普遍会有以下X个步骤(其中第2步和第3步的顺序可以调换)[这里需要引一些文献]
% 1. 数据预处理;一般而言,真实世界收集到的源数据常常是表格型的多属性数据,而输入到节点链接图创作的图数据,往往是在源数据的基础上进行预处理,通过在源数据上定义实体之间的关系(链接),最终获得输入到节点链接图的图数据(包含节点和链接)
% 2. 设计可视编码;创作者还会把他想表达的数据特征编码到节点链接图上,比如利用节点的颜色来编码节点的类别等。
% 3. 计算图布局;节点链接图需要解决的一个关键问题是如何在二维空间中放置节点。链接则往往被绘制为节点之间的连接线,其位置往往由其两个端点决定。基于布局算法产生的节点位置,节点链接图才能进行绘制。
The creation of node-link diagrams commonly includes the following steps (the order of step 2 and step 3 can be reversed)~\cite{DBLP:journals/cgf/SpritzerBDFF15, tvcg/RomatAP21}:
\begin{compactenum}[\textbf{Step} 1)]
\item \textbf{Graph wrangling} transforms the original data from tabular format to graph format by defining relationships between entities~\cite{DBLP:journals/tvcg/SrinivasanPEB18, DBLP:conf/ieeevast/BigelowNML19, DBLP:journals/ivs/HeerP14, DBLP:journals/ivs/LiuNS14}.
% 虽然Graphiti~\cite{DBLP:journals/tvcg/SrinivasanPEB18}从用户的demonstration中推测构建链接的条件,但它并不能直接被用于推测已经所有边已经构建完毕的图数据。
Although Graphiti~\cite{DBLP:journals/tvcg/SrinivasanPEB18} is designed to infer linking conditions from user demonstrations, it can not be employed directly to infer linking conditions in a graph with all links constructed.
% 需要一个新的方法,能够处理多条边之间的条件的逻辑关系,来得出完备的结论。
A new pipeline is needed to deal with logical relationships among multiple linking conditions to obtain a complete conclusion.

\item \textbf{Layout computing} positions nodes in a two-dimensional space to reveal attribute-based or topology-based patterns~\cite{DBLP:journals/cgf/NobreMSL19}.
% 虽然已经有很多布局相关的工作,但几乎没有工作能够推测某种布局结果的意义。
Although numerous layout algorithms are proposed~\cite{}, but none of them can be used to infer the meaning of a specific layout.

\item \textbf{Visual mapping} encodes node/link attributes with different visual channels to show data features. 
% 从可视化结果中提取 visual mapping 的工作也有很多,但他们大多为了处理一些基础图表,缺少对于节点链接图的支持,又或者需要强有力的先验假设(比如需要d3将数据绑定到可视化元素上),使得工作难以拓展到更多领域。我们的方法将对已有方法进行补充,使他们能够适用于更广的范围,能够良好支持节点链接图中的视觉映射的抽取。
Numerous techniques are proposed to extract visual mappings from visualization~\cite{},  they mainly aim at basic charts such as line charts, bar charts, and so on.
They rarely support node-link diagrams or are not robust enough to deal with general conditions (e.g., without binding data to visual elements).
\end{compactenum}

% 我们提出了xxxx,已解决上述挑战。
We introduce~\textit{\ApproachName}, an automatic caption generator for node-link diagrams.
% xxxx 检验了创作者创作可视化的过程,从图数据,可视化构建源码中,提取解决上述挑战的关键信息。
It extracts key information from the creation process of node-link diagrams to solve challenges mentioned above.
% 最终将提取的信息注入到一些预定义的模板中生成文字描述形成标题。
And we fill the information into several pre-defined templates to generate textual descriptions as captions
% 我们通过xxx对该方法的有效性进行了验证。
{\color{red} We evaluate ...}
% 我们的贡献可以被归纳为:
Our contributions can be concluded as:
{\colorbox{text-highlight}TODO}

%? 重要:需要提及我们的方法是template-based,重点不在于NLG,而是在于问题的formulation和信息的提取。