\firstsection{Introduction}
\maketitle

Node-link diagrams, the most popular visualization for graph data exploring, depict relations among data entities and associated attributes. 
In addition to basic elements (e.g., nodes and links), numerous visual features and channels (e.g., position, shape, size, and color) can be employed to demonstrate and illustrate various data attributes~\cite{DBLP:conf/infovis/AuberCJM03, tvcg/RomatAP21, DBLP:conf/infovis/Jankun-KellyM03, DBLP:conf/iv/SchoffelSE16}.
Visually, encoding too many attributes may require users to exert heavy cognitive effort in recognizing patterns and perceiving data information (e.g., Figure~\ref{fig:teaser}a and Figure~\ref{fig:Movie-Year-Nolan}a).
As an alternative mean, leveraging natural language-based explanations for a visualization can effectively enhance the comprehensibility of the visualization~\cite{DBLP:conf/chi/KimHA20, DBLP:conf/eccv/KembhaviSKSHF16, DBLP:conf/cvpr/KaflePCK18, DBLP:journals/coling/MittalMCR98, DBLP:journals/tochi/FerresLST13}.


Hence, our main objective is to automatically generate a group of textual descriptions for a multi-attribute node-link diagram.
Note that, natural language-based descriptions such as captions and legends have been generated for a better understanding of basic charts (e.g., bar charts and line charts)~\cite{DBLP:conf/apvis/LiuXHWY20, DBLP:conf/nips/VaswaniSPUJGKP17, DBLP:conf/inlg/ObeidH20}. 
% To some extent, generating a summarized description for a statistical chart has been popular for characterizing the visualization's meaning. However, 
Unlike charts, node-link diagrams have much more complicated characteristics which: 1) emphasize entities and relationships rather than focusing only on individual data points, 2) use more complex encoding for bundles of nodes and links, and 3)  highlight local and global graph structures with layouts. This is a missed opportunity, as knowledge of these unique characteristics prevents other methods from being transferred and applied and requires a new generation method.



To explore the requirements of explaining node-link diagrams, we invited three experts and 12 end-users to explore how to enhance the comprehensibility of node-link diagrams. In conclusion, we identified eight requirements and formulated the goal of~\ApproachName~ to automatically extract relevant information and generate interactive descriptions of node-link diagrams.

Based on these findings, we contribute~\ApproachName, an automatic description generation approach for node-link diagrams in the format of Scalable Vector Graphics (SVG), which is widely used with the popularity of different SVG-based visualization libraries~\cite{DBLP:journals/tvcg/BostockOH11, sievert2017plotly, DBLP:journals/vi/WangBLDFPC21}. Based on SVG rather than the existing visualization library (e.g., D3 and Vega-Lite), our extraction approach is more adaptable to various node-link diagrams than prior works. In essence, it extracts three types of diagram information from the underlying graph and associated visualization program: 1) meanings of data entities, 2) visual encodings, and 3) layout types. Because of the variability of graph data and visualization programs, it is difficult to mention its salient features. Our solution for inferring this information is to correlate the visualization results (SVG representations) and the inputs (the programs and the data). For each type of feature, we designed an end-to-end scheme, which inferred the visualization process by leveraging the relation between the input and the output.

Generating natural language-based descriptions takes a template-based scheme: sentences are hierarchically created within the predefined templates and logic. Linkages between visual elements and their corresponding descriptions are kept for on-the-fly visual exploration. This scheme leverages the readability and uncertainty of natural language, benefiting users to search and understand required information quickly. We also provided an interactive version of the description to help users map the information between graphs and descriptions.
To verify the utility and effectiveness of our descriptions, we conducted three case studies and an in-lab user study. Then we concluded several interesting findings and lessons, which provide insights for descriptions design in the future.
The contributions of~\ApproachName~can be summarized as follows:
\begin{compactenum}
\item We conducted a pilot study to identify the requirements of generated interpretations for node-link diagrams.
\item We presented an efficient approach to extract three types of information and generate textual descriptions to improve the comprehensibility of node-link diagrams.
\item We conducted three case studies and a user study to verify the effectiveness of our approach. We report
quantitative and qualitative findings, based on which we discuss recommendations for future work on enhancing the comprehensibility of node-link diagrams.
\end{compactenum}














% Node-link diagrams, the most popular visualization for graph data exploring, depict relations among data entities and associated attributes. 
% % Geared at presenting and encoding various data attributes for easy-understanding, researchers designed and implemented
% In addition to basic elements (e.g., nodes and links), numerous visual features and channels (e.g., position, shape, size, and color) can be employed to demonstrate and illustrate various data attributes~\cite{DBLP:conf/infovis/AuberCJM03, tvcg/RomatAP21, DBLP:conf/infovis/Jankun-KellyM03, DBLP:conf/iv/SchoffelSE16}.
% % However, with the rise of data complexity and the granularity dimension of visual information, the visual elements are increasing dramatically for node-based attributes presentation. The excessive usage of 
% Visually, encoding too many attributes may require users to exert heavy cognitive effort in recognizing patterns and perceiving data information (e.g., Figure~\ref{fig:teaser}a and Figure~\ref{fig:Movie-Year-Nolan}a).
% % effectively, negatively impacting novice users with low visualization literacy. In this case, 
% As an alternative mean, leveraging natural language-based explanations for a visualization can effectively enhance the comprehensibility of the visualization~\cite{DBLP:conf/chi/KimHA20, DBLP:conf/eccv/KembhaviSKSHF16, DBLP:conf/cvpr/KaflePCK18, DBLP:journals/coling/MittalMCR98, DBLP:journals/tochi/FerresLST13}.