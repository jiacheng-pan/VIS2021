\firstsection{Introduction}
\maketitle

【背景】\underline{图数据的广泛存在。节点链接图在图可视化中的重要性。} 但仅仅有节点链接图仍然会让用户有疑问,比如节点链接图中的节点是如何被编码的?为什么会产生这样的布局结果?等等。这些问题有待创作者进行解释。

【解决方案和动机】
我们尝试从文本生成的角度,用标题等文本的形式,来帮助创作者解释其创建的节点链接图蕴含的意义。
已经有一些方法,帮助特定的可视化形式自动化生成文本解释。
\underline{说明一下这些方法的做法。}
但针对帮助理解节点链接图的问题,还没有经过系统的探索和明确的定义。
本文探索了自动为节点链接图生成相应的解释性文本。

【我们的做法和存在的挑战】为了帮助创作者为其创作的节点链接图自动生成解释性文本,我们首先探索了观察节点链接图的观众会对节点链接图产生疑惑的原因。包括:
\begin{enumerate}
    \item 对背后的数据本身缺少了解。用户并不知道背后的数据中,节点表达的含义和链接表达的含义。
    \item 对节点链接图的编码缺少了解。用户并不知道创作者将数据中的属性在节点链接图中,用何种视觉通道进行编码。
    \item 虽然节点链接图已经能够直观展示节点之间的拓扑关系,但visual clutter使他会难以调查其中的拓扑结构。
    \item 对布局算法的不了解。观众往往不知道算法的细节,无法体会为什么会形成如此布局。 %? 通过测试是否跟某个属性相关/或者是否跟拓扑距离相关来决定
\end{enumerate}

【贡献】我们从以上几个角度出发,提出了一种\textbf{自动为节点链接图生成解释性文本的方法},它以创作者创作的节点链接图为输入,自动生成解释性文本,以帮助解决以上四个问题。\underline{我们的方法....其主要贡献可以被归纳为以下两点:}