\section{Evaluation}
We demonstrate the utility of \ApproachName~by evaluating the quality of the generated descriptions.
\ApproachName~is tailored for generating descriptions of the creation process of node-link diagrams.
To the best of our knowledge, there is no other application can be utilized to achieve the same goal.
We believe a qualitative study is more meaningful to show the utility of \ApproachName.

\subsection{Study Design}
Case 2 and Case 4 were selected as evaluation datasets,
because they have different types of linking conditions, visual encodings, and layouts.
An expert familiar with the node-link diagram creation was invited to generate a quiz questionnaire which would be used to examine whether participants understood node-link diagrams in two cases.
We first explained how the two node-link diagrams are created.
After that we asked him to describe node-link diagrams regarding the three-steps creation process.
He removed crucial information about the descriptions to generate a cloze test for each case.
Two cloze tests contains 13 vacancies and 10 vacancies respectively.

% 验证生成的句子的质量
% 李克特量表,对于四个角度以及对节点连接图的理解程度
We recruited 12 participants in the study ({\color{red}x} males and {\color{red}x} females; aging from {\color{red}x} to {\color{red}x}).
All participants are students or researchers in computer science.
{\color{red}x} of them major in visualization.
We first introduce our interactive interface to the participants.
They are asked to read the generated descriptions interactively.
The source code and data are hidden from them.
After reading, participants are asked to finish the cloze test to check their understanding of node-link diagrams.
The expert was asked to mark cloze tests finished by participants with a full mark of five.
Participants also accomplished a questionnaire to rate the quality of the descriptions with the 5-point Likert scale regarding readability, utility, aesthetics, and attractiveness.


% 1. 针对两个case 找一个expert帮忙写一些描述,在这个描述上挖一些空,准备一些模糊的备选项,作为quiz;
% 2. 自动生成交互式描述;
% 3. 各找12个人观看描述;回答五分量表;(readability (e.g., is it easy to read and follow the logic?),
% utility (e.g., does it help you understand this visual design?), aesthetics
% (e.g., does it look pretty and pleasant?) and attractiveness (e.g., does it
% attract your interest?).) 让被试填入。