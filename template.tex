% \documentclass[journal]{vgtc}               % final (journal style)
\documentclass[review,journal]{vgtc}          % review (journal style)
%\documentclass[widereview]{vgtc}             % wide-spaced review
%\documentclass[preprint,journal]{vgtc}       % preprint (journal style)

\ifpdf%                                % if we use pdflatex
  \pdfoutput=1\relax                   % create PDFs from pdfLaTeX
  \pdfcompresslevel=9                  % PDF Compression
  \pdfoptionpdfminorversion=7          % create PDF 1.7
  \ExecuteOptions{pdftex}
  \usepackage{graphicx}                % allow us to embed graphics files
  \DeclareGraphicsExtensions{.pdf,.png,.jpg,.jpeg} % for pdflatex we expect .pdf, .png, or .jpg files
\else%                                 % else we use pure latex
  \ExecuteOptions{dvips}
  \usepackage{graphicx}                % allow us to embed graphics files
  \DeclareGraphicsExtensions{.eps}     % for pure latex we expect eps files
\fi%

%% it is recomended to use ``\autoref{sec:bla}'' instead of ``Fig.~\ref{sec:bla}''
\graphicspath{{figures/}{pictures/}{images/}{./}} % where to search for the images

\usepackage{epstopdf}
\usepackage{microtype}
\PassOptionsToPackage{warn}{textcomp}
\usepackage{textcomp}
\usepackage{mathptmx}
\usepackage{times}
\renewcommand*\ttdefault{txtt}
\usepackage{cite}
\usepackage{tabu}
\usepackage{booktabs}
\usepackage{wrapfig}
\usepackage{longfbox}
\usepackage{amssymb}
\usepackage{paralist}
\usepackage{xfrac}
\usepackage{url}
\usepackage{algorithm, algpseudocode}
\usepackage{algpseudocode}
\usepackage{amsmath}
\usepackage{framed}
\usepackage{multirow}
\usepackage{xcolor}
\usepackage{todonotes}
\usepackage[shortlabels]{enumitem}
% \usepackage[xcolor=dvipdf]{changes}
\usepackage[xcolor=dvipdf, final]{changes}
\usepackage{ulem}
\usepackage{color,soul}
\definecolor{mygreen}{rgb}{0.17, 0.55, 0.05}
\definecolor{myred}{rgb}{0.85, 0.17, 0.05}
\definecolor{text-highlight}{rgb}{0.25, 0.62, 0.29}
\sethlcolor{text-highlight}
\usepackage[T1]{fontenc} % solving problem: LaTeX Error: Command \k unavailable in encoding OT1.

\usepackage{CJKutf8} %! FOR CHINESE

\usepackage{dcolumn}
\newcolumntype{d}[1]{D{.}{.}{#1}}

\newif\ifrenderappendix
% render the appendix or not
\renderappendixfalse
% \renderappendixtrue

\definechangesauthor[name=JiachengPan, color=mygreen]{pan}
\definechangesauthor[name=XiaodongZhao, color=red]{zhao}
\definechangesauthor[name=Jian Chen, color=blue]{jc}
\setdeletedmarkup{\color{gray}{\sout{#1}}}
% \setdeletedmarkup{}
% \sethighlightmarkup{\color{red}{#1}}
\setaddedmarkup{\color{mygreen}{#1}}
\setauthormarkup{}
\renewcommand{\algorithmicrequire}{\textbf{Input:}}
\renewcommand{\algorithmicensure}{\textbf{Output:}}

\definecolor{lightgray}{rgb}{0.75,0.75,0.75}
 
\newenvironment{lightgrayleftbar}{%
  \def\FrameCommand{\textcolor{lightgray}{\vrule width 3pt} \hspace{3pt}}%
  \MakeFramed {\advance\hsize-\width \FrameRestore}}%
{\endMakeFramed}

\newcommand{\ApproachName}{GraphAugmentor}
\newcommand{\PaperTitle}{\ApproachName: Augmenting Node-Link Diagrams With Textual Descriptions} % todo


\onlineid{1396}

\vgtccategory{Research} % TODO
\vgtcpapertype{Representations \& Interaction} % TODO

\title{\PaperTitle}
\author{Jiacheng Pan, Zihan Yan, Zihan Zhou, Xiaodong Zhao, Shenghui Cheng, Dongming Han, Jian Chen, Mingliang Xu, Wei Chen}
\authorfooter{
    % \item Jiacheng Pan, Zihan Yan, Zihan Zhou, Xiaodong Zhao, Dongming Han, and Wei Chen are with the State Key Lab of CAD\&CG, Zhejiang University, China. E-mail: \{panjiacheng, zhouzihan,  zhaoxiaodong\}@zju.edu.cn, chenwei@cad.zju.edu.cn.\\
    % Wei Chen is the corresponding author.
    
    % \item Jian Chen is with Ohio State University, USA. E-mail: chen.8028@osu.edu.
}

\shortauthortitle{Pan \MakeLowercase{\textit{et al.}}: \PaperTitle}
    \abstract{
    Node-link diagrams are the most popular form for graph visualization.
    Yet, salient information and visual structures of a graph can not be fully depicted by solely presenting its node-link diagram.
    We propose to augment the comprehensibility of node-link diagrams by creating textual descriptions. 
    We conduct a pilot study to identify eight requirements of generated interpretations, including four items for connection extraction and four items for visual expression.
    Our solution, \textit{\ApproachName}, generating interactive descriptions for the widely-used Scalable Vector Graphics (SVG) format node-link diagrams, consists of two stages: feature extraction and description generation. 
    The first one identifies and extracts visual features, like node-link connections, visual designs, and types of graph layouts.
    The second stage creates a group of hierarchical sentences based on a pre-defined schema. 
    To the best of our knowledge, our approach is the first attempt to textually convey the visual design of node-link diagrams automatically. 
    Three use cases and in-lab user study confirm the superiority of our approach.
    }
    
    \keywords{Node-link diagram, graph visualization, visual encodings, visual representations}

%% ACM Computing Classification System (CCS). 
%% See <http://www.acm.org/class/1998/> for details.
%% The ``\CCScat'' command takes four arguments.

\CCScatlist{ % not used in journal version
 \CCScat{K.6.1}{Management of Computing and Information Systems}%
{Project and People Management}{Life Cycle};
 \CCScat{K.7.m}{The Computing Profession}{Miscellaneous}{Ethics}
}

\ifrenderappendix
    \teaser{}
\else
    \teaser{
      \centering
      \includegraphics[width=1\linewidth]{figures/teaser.eps}
        \caption{
            The pipeline of~\ApproachName.
            (a) The node-link diagram created with the Actor-China graph (17 nodes and 25 links). (b) The feature extraction modules summarize link conditions, visual encodings, and the layout type of the diagram. (c) The generator interprets the summarized features with interactive descriptions. (d) The generated descriptions are interactive. Hovering on descriptions will highlight their corresponding visual elements (e).
            % The interface of~\ApproachName.
            % (a) The node-link diagram created upon the Actor-Movie-2016-2021-China Dataset.
            % The graph consists of 17 nodes (actors in china who have acted in more than five movies from 2016 to 2021) and 55 links (two actors are connected if they acted in at least one movie). (b)~\ApproachName~generates descriptions to explain the graph automatically, including the meanings of nodes and links, visual encodings, and the layout meanings. Descriptions are linked to the visualization to make them more targeted. (c) Hovering on a certain description will toggle highlighting on the visual element it describes.
        }
        \label{fig:teaser}
    }
\fi

\vgtcinsertpkg
\begin{document}
\begin{CJK}{UTF8}{gbsn} %! FOR CHINESE

\ifrenderappendix
    \maketitle
\section{Node-Link Diagrams in the User Interview}

\begin{table}
\centering
\caption{\label{tab:pearson-correlation}The Pearson correlation coefficients testing correlations between Euclidean distances and graph geodesic distance with 15 graphs and 5 different layouts. Cells with coefficients less than $\theta=0.5$ are highlighted. $p$-values are all zero.}
\label{tab:pearson-correlation}
\begin{tabular}{llllll} 
\hline
Dataset       & FM$^3$ & F.R.           & S.M.  & PMDS  & R.T.            \\ 
\hline
dwt\_72       & 0.894  & 0.552          & 0.94  & 0.897 & 0.714           \\
lesmis        & 0.804  & 0.728          & 0.815 & 0.779 & 0.504           \\
bcsstk09      & 0.967  & 0.610          & 0.971 & 0.947 & 0.754           \\
cage8         & 0.658  & 0.678          & 0.725 & 0.701 & \textbf{0.307}  \\
can\_96       & 0.835  & 0.824          & 0.855 & 0.869 & \textbf{0.426}  \\
dwt\_1005     & 0.969  & 0.561          & 0.974 & 0.972 & 0.623           \\
dwt\_419      & 0.979  & \textbf{0.391} & 0.987 & 0.987 & 0.775           \\
grid17        & 0.968  & 0.551          & 0.976 & 0.964 & 0.738           \\
jazz          & 0.826  & 0.821          & 0.813 & 0.786 & \textbf{0.297}  \\
mesh3e1       & 0.99   & 0.502          & 0.996 & 0.995 & \textbf{0.465}  \\
netscience    & 0.901  & 0.573          & 0.93  & 0.919 & 0.752           \\
price\_1000   & 0.783  & \textbf{0.376} & 0.808 & 0.786 & 0.727           \\
rajat11       & 0.753  & 0.670          & 0.868 & 0.847 & 0.618           \\
soc-wiki-Vote & 0.798  & 0.775          & 0.821 & 0.757 & \textbf{0.316}  \\
visbrazil     & 0.937  & 0.828          & 0.941 & 0.958 & 0.758           \\
\hline
\end{tabular}
\end{table}
\else
    \firstsection{Introduction}
\maketitle

% 图是一种广泛使用的数据结构,在金融、社交、生物等领域都扮演着重要的角色。
Graphs, also called networks, are ubiquitous over numerous areas. For example, graphs can model the financial transaction behaviors, social media contacts, biological processes such as protein-protein-interactions, and so on.
% 节点链接图被广泛应用于对图数据进行可视化,可以用来揭示数据实体之间的关系。
Node-link diagrams are widely used to visualize graphs to reveal the connections and relationships among data entities.
Generating a caption to summarize a node-link diagram is crucial to enhance its accessibility.
However, the automatic caption generation has not been explored yet.

% 为可视化生成语言描述有很多好处,它的主要目的是为了帮助观众理解可视化所传达的意义。
Generating a caption, or a description, to summary a visualization has numerous benefits.
% 根据生成的描述,可以将这类工作分成两类,一类描述了可视化的基本构成(比如数据本身,可视映射等),另一类则关注于可视化中蕴含的insight。
According to the results, automatic description generation techniques can be roughly summarized into two categories~\cite{DBLP:conf/inlg/ObeidH20}: one describes constituents of a visualization objectively(e.g., visual mappings, underlying data, etc.)~\cite{DBLP:journals/coling/MittalMCR98, DBLP:journals/tochi/FerresLST13} and the other describes high-level insights conveyed by the visualization subjectively~\cite{DBLP:conf/apvis/LiuXHWY20, DBLP:conf/inlg/ObeidH20}.
% 这些工作,基本都是为基础可视化图表生成相关描述
They are mostly aimed at generating descriptions for basic charts such as line charts, bar charts, area charts, and so on.
% 其中第二类工作生成的insight大多是基于模板,或是跟训练数据相关的,其自由度会受到这些因素的限制。
Insights generated by the latter are most template-based, or related to the training data, thus will be limited by such factors.
% 本身节点链接图背后的图数据,相比于基础图表背后的表格型数据更具复杂性
Compared to basic charts, underlying graphs behind node-link diagrams are usually more complex, 
% 所以第二类工作反而会导致生成的描述不够准确或者不够完备。
insights generated may be inaccurate or incomplete, thus influence analysis.
% 第一类工作对于解释可视化本身很有意义,而insights则由用户自己决定,这会给予用户更多的分析自由度。
Whereas, the former category of techniques only describe the visualization itself, while insights are decided by audiences, thus reach a higher degree of analysis freedom.
% 我们的方法只描述节点链接图的创作者所做的事情,而不去限制用户的思考和分析。
We only focus on generating objective descriptions about node-link diagrams and do not limit audiences' thinking and analysis.


To generate such descriptions, we need to decompose the creation process of node-link diagrams to explore the challenges.
% 一个节点链接图的创作过程普遍会有以下X个步骤(其中第2步和第3步的顺序可以调换)[这里需要引一些文献]
% 1. 数据预处理;一般而言,真实世界收集到的源数据常常是表格型的多属性数据,而输入到节点链接图创作的图数据,往往是在源数据的基础上进行预处理,通过在源数据上定义实体之间的关系(链接),最终获得输入到节点链接图的图数据(包含节点和链接)
% 2. 设计可视编码;创作者还会把他想表达的数据特征编码到节点链接图上,比如利用节点的颜色来编码节点的类别等。
% 3. 计算图布局;节点链接图需要解决的一个关键问题是如何在二维空间中放置节点。链接则往往被绘制为节点之间的连接线,其位置往往由其两个端点决定。基于布局算法产生的节点位置,节点链接图才能进行绘制。
The creation of node-link diagrams commonly includes the following steps (the order of step 2 and step 3 can be reversed)~\cite{DBLP:journals/cgf/SpritzerBDFF15, tvcg/RomatAP21}:
\begin{compactenum}[\textbf{Step} 1)]
\item \textbf{Graph wrangling} transforms the original data from tabular format to graph format by defining relationships between entities~\cite{DBLP:journals/tvcg/SrinivasanPEB18, DBLP:conf/ieeevast/BigelowNML19, DBLP:journals/ivs/HeerP14, DBLP:journals/ivs/LiuNS14}.
% 虽然Graphiti~\cite{DBLP:journals/tvcg/SrinivasanPEB18}从用户的demonstration中推测构建链接的条件,但它并不能直接被用于推测已经所有边已经构建完毕的图数据。
Although Graphiti~\cite{DBLP:journals/tvcg/SrinivasanPEB18} is designed to infer linking conditions from user demonstrations, it can not be employed directly to infer linking conditions in a graph with all links constructed.
% 需要一个新的方法,能够处理多条边之间的条件的逻辑关系,来得出完备的结论。
A new pipeline is needed to deal with logical relationships among multiple linking conditions to obtain a complete conclusion.

\item \textbf{Layout computing} positions nodes in a two-dimensional space to reveal attribute-based or topology-based patterns~\cite{DBLP:journals/cgf/NobreMSL19}.
% 虽然已经有很多布局相关的工作,但几乎没有工作能够推测某种布局结果的意义。
Although numerous layout algorithms are proposed~\cite{}, but none of them can be used to infer the meaning of a specific layout.

\item \textbf{Visual mapping} encodes node/link attributes with different visual channels to show data features. 
% 从可视化结果中提取 visual mapping 的工作也有很多,但他们大多为了处理一些基础图表,缺少对于节点链接图的支持,又或者需要强有力的先验假设(比如需要d3将数据绑定到可视化元素上),使得工作难以拓展到更多领域。我们的方法将对已有方法进行补充,使他们能够适用于更广的范围,能够良好支持节点链接图中的视觉映射的抽取。
Numerous techniques are proposed to extract visual mappings from visualization~\cite{},  they mainly aim at basic charts such as line charts, bar charts, and so on.
They rarely support node-link diagrams or are not robust enough to deal with general conditions (e.g., without binding data to visual elements).
\end{compactenum}

% 我们提出了xxxx,已解决上述挑战。
We introduce~\textit{\ApproachName}, an automatic caption generator for node-link diagrams.
% xxxx 检验了创作者创作可视化的过程,从图数据,可视化构建源码中,提取解决上述挑战的关键信息。
It extracts key information from the creation process of node-link diagrams to solve challenges mentioned above.
% 最终将提取的信息注入到一些预定义的模板中生成文字描述形成标题。
And we fill the information into several pre-defined templates to generate textual descriptions as captions
% 我们通过xxx对该方法的有效性进行了验证。
{\color{red} We evaluate ...}
% 我们的贡献可以被归纳为:
Our contributions can be concluded as:
{\colorbox{text-highlight}TODO}

%? 重要:需要提及我们的方法是template-based,重点不在于NLG,而是在于问题的formulation和信息的提取。
    \section{Related Work}\label{sec:relatedwork}
\subsection{Node-link Diagrams}
\subsection{Extracting Information From Visualization}
% 抽取信息对于生成描述至关重要。
Extracting information from visualization is crutial for generating descriptions.
% 对可视化进行信息抽取的工作大概可以分成两类。
% 第一类工作聚焦从可视化结果中抽取insight。
% Chart-to-Text: Generating Natural Language Descriptions for Charts by Adapting the Transformer Model.
% Contextifier: Automatic generation of annotated stock visualizations.
% Automatic annotation synchronizing with textual description for visualization.
% Describing Complex Charts in Natural Language: A Caption Generation System

% 第二类工作聚焦于从各种可视化结果还原潜在数据。
Another category of methods focus on retrieve data from visualization results.
% 不同类型的chart需要不同的还原算法,所以每种方法往往聚焦在一个类型的chart上
Different types of charts require different retrieval algorithm.
Thus each method most focuses on one type of charts.
A Modified Probabilistic Hough Transform algorithm (MPHT) was developed to detect parallel lines clusters and it reconstrcts bar patterns from the clusters. It retrieves data from bar charts in raster images even hand-drawn images~\cite{DBLP:conf/icip/ZhouT00}. 
A system is developed to extract data from bar charts, pie charts, and line charts with different rules~\cite{DBLP:conf/doceng/HuangT07}. It also extracts textual information and text/graphics association to generate description for charts and their underlying data.

% TODO  View: Visual Information Extraction Widget for improving chart images accessibility

% TODO  Access to multimodal articles for individuals with sight impairments

% TODO  Getting computers to see information graphics so users do not have to


% ChartSense: Interactive Data Extraction from Chart Images: 先将图像进行分类,然后利用了一个mixed-initiative approach对图标蕴含的数据进行抽取。人首选指定检测范围和坐标轴等辅助信息,ChartSense算法检测其中包含的图形元素及其蕴含的数据,最后由人挑选检测正确的元素并修改其中错误的数据。
% Scatteract: Automated Extraction of Data from Scatter Plots: employs deep learning techniques to identify data from scatter plots with linear scales. It is the first approach that fully aumatic extracts data from scatter plots.
% 

% 第三类工作聚焦于从可视化结果中还原可视编码
% Reverse-Engineering Visualizations: Recovering Visual Encodings from Chart Images: 从 像素图 中识别文本信息(标题、坐标轴标签等),对文本进行分类和还原,并且用CNN网络来分类mark类型,通过文本和mark类型来恢复编码信息。

% Deconstructing and restyling D3 visualizations: 使用了d3数据绑定的特性,直接从svg元素的__data__属性中读取其绑定的数据,然后用线性回归等方法检验元素的视觉属性和数据属性之间的映射关系来获取编码。
% Converting basic D3 charts into reusable style templates 和 Searching the Visual Style and Structure of D3 Visualizations 对 Decons.. 进行了改进。前者围绕着图表中蕴含的label(数据点的label和坐标轴的label)进行了更加细致的分类;后者则为该方法增加了一些新的检测目标:比如对角度的检测,对非编码的视觉通道(background color, stroke-width等)的检测等。

% Visualizing for the Non-Visual: Enabling the Visually Impaired to Use Visualization :


% 这些方法或是只适用于检测基础的可视化图表如bar chart/line chart/area chart等,或是只适用于对d3等具有数据绑定的svg进行编码提取。
% 我们提出了对节点链接图进行信息提取的方法,该方法从多个角度出发,适用于通用的svg场景,而不需要数据绑定的前提。
    \section{Requirement and Overview}\label{sec:pilotstudy}

As surveyed in Section~\ref{sec:relatedwork}, there are two main steps to augment the node-link diagram: information extraction and description. Thus, we conducted a pilot study aiming to investigate the requirements for these two steps from the viewpoints of both 3 experts (P1-3) and 12 end-users (P4-15). We showed them some typical and various sample node-like diagrams with legends. Based on their relevant feedback, we analyzed the approach requirement and proposed an overview of our approach.

\subsection{Requirement Analysis}

Based on our analysis, we came up with four requirements [E1-4] for information extraction and three requirements [D1-3] for description generation.

\noindent {\bf Extracting Information.}

\begin{compactenum}[\textbf{E}1]
    \item {\bf Searching link condition.} The meanings of links should be demonstrated directly to end-users. Unlike nodes, users cannot find the meaning of a link on the title or caption of a graph. Hence, all experts suggested we summarize the meanings of links for users before they carefully analyze the graph. Experts also gave more refined suggestions pertaining to our approach of finding the correct link condition~\cite{DBLP:journals/ivs/LiuNS14, DBLP:journals/ivs/HeerP14, DBLP:journals/tvcg/SrinivasanPEB18} among nodes to help users understand the meaning of links automatically.
    
    \item {\bf Summarizing visual encodings.} Visual encoding is a necessary part for users to understand the graph. Although legends always represent it, they are too concise for users to understand. Furthermore, both P4 and P7 commented that despite their ability to comprehend visual encodings, it was wasteful to search attributes for each node and link according to their visual elements and legends. Thus, it is highly recommended to summarize all visual encodings, which can be demonstrated directly to users later.
    
    \item {\bf Identifing the layout type.} All experts agreed that the layout could help users understand the node-link diagram. Whereas, due to the lack of prior knowledge, most end-users ignored the layout. Experts recommended that we present only the essential information and type of layout rather than explain the algorithm details in detail to reduce end-users' understanding load. Expert P2  suggested only identifying whether the layout is topology-driven or attribute-driven, or neither, according to the taxonomy~\cite{DBLP:journals/cgf/NobreMSL19}. 
    
    \item {\bf Utilizing multi-source inputs}. Experts suggested our approach should utilize multi-source inputs, especially the original data and the visualization program. Expert P1 emphasized the importance of visualization programs, since they contain more information, such as how data attributes are encoded by visual channels, than raw data. Enriching the types of our inputs can improve the accuracy and the comprehensiveness of the collected information to avoid misleading users.
\end{compactenum}

\noindent {\bf Describing information. }
\begin{compactenum}[\textbf{D}1]
    \item {\bf Describing information with textual descriptions.} Most end-users prefer our method as it can use more detailed hints such as textual descriptions to provide more comprehensive illustrations. 
    % They mentioned that the common used legend is troublesome, first, they need to learn the mapping relationship, second, they should interpret each visual element one by one by searching the corresponding information provided by the legend. 
    Expert P2 suggested to use textual descriptions, which are more flexible to carry different content.
    The use of textual descriptions can reduce the learning cost, as well as help users focus more on analysis rather than information searching~\cite{DBLP:journals/tochi/FerresLST13, DBLP:conf/inlg/ObeidH20, DBLP:conf/apvis/LiuXHWY20}. 
    % The end-user P4 said that ``Language is the bridge of communication, and language descriptions of visualizations can bridge me to their authors.'' 
    
    \item {\bf Using templates to generate consistent descriptions.} One expert suggested that a template-based scheme is more suitable for our description generation, because it is stable and consistent to display structured descriptions in our scenario. He also suggested highlighting non-template information in the template to facilitate users' recognition.
    % End-users also prefer using template. They deemed that in tasks, they often need to find one attributes for multiple nodes. If the attributes showed on different place of descriptions, it is difficult for them to search the information and analyze.
    
    \item {\bf Organizing descriptions according to their relationship.} Because there may be several different types of information, expert P2 suggested organizing the corresponding descriptions according to their relationships. Specifically, he said that all descriptions of visual encodings of the same data entity should be displayed together without adding any additional descriptions, such as the meanings of links. It was agreed upon by all end-users that descriptions should be arranged according to the relationship as most daily document tools did, such as ``X-mind'', ``Google Doc'', ``Typora''and so on.
    
    \item {\bf Constructing mapping between graph and descriptions.} Building a linkage between our hints and the visualization can help users quickly locate and search for the target. Some end-users mentioned that the static hints (e.g., legend) obstructed their information tracking. They suggested providing an interactive mapping tool; for example, the relevant visual elements can be highlighted while hovering on their descriptions. The interactive mapping tool can make the hints more targeted and reduce mismatches with the end-users’ mental map.
\end{compactenum}

\subsection{\ApproachName~Overview} \label{sec:overview}

\begin{figure*}[t]
    \centering
    \includegraphics[width=2\columnwidth]{figures/workflow.eps}
    \caption{The pipeline of~\ApproachName. (a) Three information extraction modules search link conditions, summarize visual encodings, and identify layout types. (b) A description generator generates template-based descriptions with interactions. 
    }
    \label{fig:workflow}
\end{figure*}

Following the aforementioned requirements, we propose~\ApproachName~(Figure~\ref{fig:workflow}), an automatic description generation approach to augment the comprehensibility of node-link diagrams, which consists of three information extraction modules and an interactive description generator. They work together to extract information from node-link diagrams and generate corresponding descriptions for end-users.

\noindent {\bf Information Extraction Modules}  (Section~\ref{sec:approach}). To complete \textbf{E1-4}, we designed and implemented three information extraction modules (Figure~\ref{fig:workflow}a) with multi-source inputs (\textbf{E}) including the captured visualization programs and the original graph data.

\begin{compactenum}[\textbf{M}1]
    \item {\bf Link Condition Search Module.}
    To implement \textbf{E1}, we searched the link conditions. The module takes the original graph data as input and selects proper conditions between connected node pairs to represent the link meanings.
    
    \item {\bf Visual Encodings Summarization Module.}
    Following \textbf{E2}, our module forms the encoding scheme by summarizing mappings between data entities and visual elements and the correlations between data attributes and visual channels. It takes visualization programs and the original graph data as inputs and tests the relationship between the graph data and output visualizations to infer visual encodings.
    
    \item {\bf Layout Type Identification Module.} Based on \textbf{E3}, we proposed this module to determine whether the layout is attribute-based or topology-based. This module will capture the position of each node by computing bounding boxes, and determine the layout type by testing the Pearson correlation coefficient.
\end{compactenum}

\noindent {\bf Description Generator} (Section~\ref{sec:generator}).
According to \textbf{D1-4}, we propose a description generator
(Figure~\ref{fig:workflow}b) to express the extracted information. 
The description generation follows a manner of template-filling (\textbf{D1-2}). Although the template resembles the skeleton of the description, the spirit of the description is derived from the extracted information.
To demonstrate the relationship of different levels of descriptions (\textbf{D3}), we organized the generated descriptions in a pre-set structure. Besides, driven by \textbf{D4}, we linked them to their corresponding visual elements to reach an interactive scheme. When users hover over a description or visual element in this interactive scheme, the other element is highlighted.

    % \input{chapter/4.Overview}
    \section{Information Extraction Modules}\label{sec:approach}
In this section, we describe the implementation details of three information extraction modules. Compared with previous approaches~\cite{DBLP:journals/tvcg/HarperA18, DBLP:conf/uist/HarperA14}, our approach has highly significant advantages in adaptability. Specifically, our implementation only requires several accessible inputs by using existing network traffic investigation tools. By using a result-driven approach, our concept regards the creation of node-link diagrams as an invisible ``BlackBox''. It does not depend on any visualization library such as D3~\cite{DBLP:journals/tvcg/BostockOH11} and Vega-Lite~\cite{DBLP:journals/tvcg/SatyanarayanMWH17}, and thus can be utilized to process a diverse variety of visualization programs. Such design enables high adaptability to search different link conditions, summarize complicated visual encoding, and identify different layout results. To maintain consistency, we organize the three modules using the same structure as Section~\ref{sec:overview}.

\subsection {M1: Link Condition Search Module}

Unlike the existing techniques to construct links by selecting link conditions between data entities (nodes), our method infers link conditions based on already constructed links, which reverses the construction process. It extends the application of existing techniques to visualization deconstruction for comprehensibility enhancement.
To be specific, this module consists of three main steps listed below:
% 1) search candidate link conditions in every pair of nodes, 2) filter out the wrong conditions, and 3) store the remaining conditions with a priority, where a more detailed condition gets a higher ranking.

\noindent \textbf{M1.1 Search Candidate Link Conditions}. 
We first constructed conditions between all pairs of nodes. We chose the taxonomy of Graphiti~\cite{DBLP:journals/tvcg/SrinivasanPEB18} to categorize conditions into four types (Summarized in Table~\ref{tab:template}, Link Conditions C1-4). We formalized the conditions with three aspects to facilitate comparison, sorting and filtering.
One condition can be defined as:
\begin{equation}
    \abovedisplayskip=5pt
    \abovedisplayshortskip=5pt
    \belowdisplayskip=5pt
    \belowdisplayshortskip=5pt
    linking\text{ }condition := ( type, attribute, value ) \label{def:linkingcondition}
\end{equation}
where $type$ is the condition type, $attribute$ is the name of the attribute, $value$ is the value of the attribute when the condition holds.
For example, two movies sharing the same actors Alice and Bob are connected under the Condition \textbf{C2}, which could be represented as: ($type$=C2, $attribute$=actors, $value$=[Alice, Bob]). 

\noindent \textbf{M1.2 Filter Out Wrong Conditions}.
After searching candidate link conditions, current conditions are established on all node pairs despite no connection between two nodes.
Considering our goal is to detect conditions which only exist on node pairs, which are connected by links, we need to remove conditions held on node pairs without links. After that, only conditions with the least number of links will be selected from the remaining conditions.

\noindent \textbf{M1.3 Sort Remaining Conditions}.
After that, conditions are sorted with a priority, 
where the more detailed the condition, the higher its ranking.
For example, the condition ($type$=C2, $attribute$=actors, $value$=[Alice, Bob]) is implied by ($type$=C2, $attribute$=actors, $value$=arbitrary), where ($value$=arbitrary) means the condition does not assume the $actors$ should be of a certain value.
The former condition is more specific than the latter, and is thus ranked higher.
The highest-ranked condition is regarded as the most likely condition.

\begin{algorithm}[tp]
    \renewcommand\arraystretch{1.2}
    \caption{ Link Condition Search }
    \setlength{\belowcaptionskip}{-15pt}
    \label{alg:conditions}
    \begin{algorithmic}[1]
        \Require
            $G=(V=\{v_1, v_2, ..., v_n\}, E=\{e_1, e_2, ..., e_n\})$: a graph
        \Ensure
            $C$: the potential condition set
        \State Init conditions $C=\varnothing$, false conditions $FC=\varnothing$
        \For {each node pair $(v_i, v_j)$}
            \State $C_{ij} \gets$ all conditions that can connect $(v_i, v_j)$
            \If {$(v_i, v_j)$ is not a link}
                \State merge $FC$ with $C_{ij}$
            \Else
                \State merge $C$ with $C_{ij}$
            \EndIf
        \EndFor
        \State remove $FC$ from $C$
        \For {each condition $c$ in the condition set $C$}
            \If {the frequency of $c$ is less than $|E|$}
                \State remove $c$ from $C$
            \EndIf
        \EndFor
        \State sort $C$
        \State \Return $C$
    \end{algorithmic}
\end{algorithm}

\subsection{\textbf{M2: }Visual Encodings Summarization Module}\label{sec:visualencodings}
Our module summarizes visual encodings from node-link diagrams in the SVG format, which is widely used in the visualization area~\cite{DBLP:journals/tvcg/BostockOH11, sievert2017plotly, DBLP:journals/vi/WangBLDFPC21}.
To enlarge the usage scenario of this module, we did not make the algorithm rely on any library.

Attributes of nodes and links are usually encoded by visual channels to reveal attribute-based patterns.
Nodes and links contained in the underlying graph are denoted as \textit{data entities}, and each data entity consists of several \textit{attributes}.
For example, in the node-link diagram example (Figure~\ref{fig:VisualEncodings}), a node contains a categorical attribute (\textit{x}) and a numerical list attribute (\textit{y}). We encode the attribute \textit{y} with the height of two rectangles.
Then we encode the attribute \textit{x} with the two rectangles' fill color and the background rectangle's stroke color.
Visualization creators can write programs with the W3C DOM API to construct visualizations within SVG.
A SVG includes a root element \texttt{<svg>} and allows hierarchical grouping of sub-elements with group elements \texttt{<g>}.
Marks onscreen are generated by graphical elements, such as \texttt{<rect>}, \texttt{<circle>}, and \texttt{<ellipse>}.
We call these graphical elements \textit{visual elements}.
Their style attributes such as cx, cy, width, and height are denoted as \textit{visual channels}. We formulate the encoding scheme as:
\begin{equation}
    \abovedisplayskip=5pt
    \abovedisplayshortskip=5pt
    \belowdisplayskip=5pt
    \belowdisplayshortskip=5pt
    encoding := (entity, attribute, element, channel) \label{def:encoding}
\end{equation}


\begin{figure}[tp]
    \centering
    \includegraphics[width=1\columnwidth]{figures/VisualEncodings.eps}
    \setlength{\belowcaptionskip}{-10pt}
    \caption{How \ApproachName~summarizes visual encodings from a node-link diagram in the SVG format. (a) A node-link diagram consists of three nodes and three links (upper left corner). (b) Node elements are extracted and the data binding step (mapping visual elements to node entities)
    % maps them into different node entities. Then elements having the same role across different node entities are aligned into the same role class in the 2.
    the elements aligning step (aligning visual elements according to their roles),
    % Mappings among roles, visual channels, and attributes are detected by the 3.
    and the encoding mapping step (mapping data attributes to visual channels).}
    \label{fig:VisualEncodings}
\end{figure}

\begin{figure}[tp]
    \centering
    \setlength{\belowcaptionskip}{-10pt}
    \includegraphics[width=1\columnwidth]{figures/DataBinding.eps}
    \caption{ The Data binding is achieved by swapping entities. 
    % At first, visual mappings between original visual elements and data entities are unknown. Swapping attributes of entities B and C influences the appearance of elements 1 to 4. Thus, B and C correspond to 1 to 4. Them, swapping attributes of  A and B influences 2, 3, 7, and 8. Thus, A and B correspond to these elements. Swapping B with A and C changes elements 2 and 3 twice. Thus, we map entity B to elements 2 and 3.
    }
    \label{fig:DataBinding}
    % \setlength{\abovecaptionskip}{-100pt}
    % \setlength{\belowcaptionskip}{-100pt}
\end{figure}

\begin{figure}[tp]
    \centering
    \setlength{\belowcaptionskip}{-10pt}
    \includegraphics[width=1\columnwidth]{figures/ElementAligning.eps}
    \caption{The target of our visual encoding summarization module.
    (a) The target of the module is to find mappings among data entities, visual elements, data attributes, and visual channels. (b) To achieve the target, we utilize the data-binding step and the elements-aligning step to clarify the entity and the role of each visual element.}
    \label{fig:ElementAligning}
\end{figure}




{\bf M2.1 Data Binding.} \label{sec:databinding}
To detect mappings between \textit{data entities} and \textit{visual elements}, we modified attribute values of data entities and recorded corresponding visual element changes to construct mappings between them.
% For example, the node's \texttt{size} attribute is linearly mapped into the radius of the \texttt{<circle>} element encoding the node: 
% $radius_i = (size_i - min(size)) / (max(size) - min(size)) * radius_{max}$.
% Directly modifying $size_i$ may broaden the attribute range and changes the linear mapping defined by the attribute range.
% So that elements related to other data entities may also be changed.
% We prevent this by merely swapping attributes of the two entities rather than modifying them, so that no new data is introduced, and the distribution is preserved.
To prevent changing the distribution of attribute values, we merely swap attributes of the two entities rather than modifying them.
After swapping all two entities' attributes, visual elements that differ from the previous are regarded as two entities' corresponding elements.
For example, in Figure~\ref{fig:DataBinding} after swapping nodes B and C, elements 1 to 4 are changed.
All these changed elements correspond to nodes B and C because only B and C are modified. Then, after swapping the node B with A, elements 2 and 3 are changed twice. Thus they correspond to node B because only node B was swapped twice. Each node will be swapped with all other nodes to ensure detecting all elements belonging to it. After swapping all nodes, the entire node-to-element mapping is constructed. Thus, node entities are bound to their corresponding elements. The link-to-element mapping is constructed in the same way. Because swapping two nodes may influence their related links, node elements currently contains elements of their related links. We remove link elements from the node-to-element mappings. Two parts (\textit{entity} and \textit{element}) of the mapping relationship are solved.


{\bf M2.2 Elements Aligning.}
The data-binding step only binds visual elements into different data entities (the horizontal direction in Figure~\ref{fig:ElementAligning}b).
The roles of different elements are unknown.
The \textit{role} of an element is identified by its function in the visualization process.
We define the \textit{role} as a function that maps attributes to visual channels:
\begin{equation}
    \abovedisplayskip=5pt
    \abovedisplayshortskip=5pt
    \belowdisplayskip=5pt
    \belowdisplayshortskip=5pt
    role :=  \{attribute: value\} \mapsto \{channel: appearance\}
\end{equation}
Two elements have the same role if, given arbitrary, same inputs (all attributes of their corresponding entities) produce same outputs (all their visual channels).
For the example in Figure~\ref{fig:VisualEncodings}, 
the roles of all three left bars in node A, B, and C are the same,
because they encode same attributes with same visual channels.
% we swap all attributes of nodes A and B, node A's left rectangle before swapping will be the same as node B's left rectangle after swapping regarding all their visual channels such as x, y, fill, and height.
They are aligned to classify their roles (the vertical direction in Figure~\ref{fig:ElementAligning}b).
We determine the role of each element by swapping its data entity with others. Two elements of the same role behave the same after exchanging their data entities.
% Different elements' role identity can be determined by swapping their corresponding data entities along with the data binding step, because all attributes of one entity before swapping are the same to the counterpart of the other one after swapping.
% For instance, in Figure~\ref{fig:DataBinding}b, after swapping entity B with entity C, element 1 appears the same as element 2 before swapping in Figure~\ref{fig:DataBinding}a. Thus, elements 1 and 2 can be aligned into the same role.
We clarify the binding among visual elements, data attributes, and visual channels, which is conducive to the subsequent steps.


{\bf M2.3 Encoding mapping.}\label{sec:encodingmapping}
The previous two steps align elements according to two dimensions (the entity and the role) to clarify the relationship between data entities and visual elements.
However, correlations between visual channels and data attributes are not determined.
We detected related visual channels of an attribute by shuffling it of  all data entities and observing changes of their corresponding elements.
One correlation is defined as ``which \textit{attribute} changes which \textit{visual channel} of which \textit{element}''. We formalize it as:
\begin{equation}
    \abovedisplayskip=5pt
    \abovedisplayshortskip=5pt
    \belowdisplayskip=5pt
    \belowdisplayshortskip=5pt
    correlation := ( attribute, element, channel )
\end{equation}
We merge different correlations according to the role of elements.
Moreover, we identify the category of correlations to clarify correlations between visual channels and data attributes.
It requires the classification of attribute types and channel types.
We support numerical attributes and categorical attributes.
List attributes and dictionary attributes are separated into multiple numerical attributes or categorical attributes (e.g.,[1, 2, 3], \{"year":2021,"month":03\}).
We regard all visual channels as numerical (colors can be divided into RGB channels, which are numerical).
However, numerical data can be used as categorical data if there are only a few unique values (less than $\alpha\%$).
% For example, natural numbers are often used as categorical attributes such as labels, groups, and classes.
% Thus, for numerical data, we must compare the number of unique values and their entries to determine whether it is numerical or categorical.
% We set up a parameter $\alpha$ to make the distinction: if all values of the attribute are natural numbers and
% the number of an attribute's unique values is less than $\alpha$ a percent of the number of data entities, we regard the attribute as categorical.
We identify the type of correlation by attribute and channel types:

\noindent \textbf{Categorical correlation} -- the channel and attribute are both categorical or the channel is categorical while the attribute is numerical. 
To summarize the correlation of different channel values, we record the value range of the attribute for each channel values.
    
\noindent \textbf{Numerical correlation} -- the channel and attribute are both numerical. We compute the Pearson's Correlation Coefficient and test whether the correlation is positive, negative, or uncorrelated with the coefficient and the significance test's p-value.  The correlation is built when the absolute value of the coefficient is larger than $\theta$=.5, and the $p$-value is less than $\alpha$=.05. Both parameters can be adjusted on-demand.

We do not support the correlation type where the channel is numerical while the attribute is categorical because the former is continuous and the latter is discrete. It is counterintuitive.

\subsection{\textbf{M3}: Layout Type Identification Module}


\textbf{M3.1 Capturing the position of each node}.
Node positions are used to determine the layout type.
To capture each node's position, we compute a bounding box for elements detected in the \textbf{Data Binding} step.
It is the smallest rectangle that contains all corresponding elements.
We take its centroid as the position of the node.

\textbf{M3.2 Determining the layout type.} 
Topology-driven layouts prioritize the topology of a graph~\cite{DBLP:journals/cgf/NobreMSL19}, which means their graph geodesic distance influences the Euclidean distance between two nodes. To validate this assumption, we performed a study by using the Pearson test to check the correlation between the geodesic distance and the Euclidean distance in 15 graphs from~\cite{DBLP:journals/tvcg/ZhuCHHLZ21} layered by five topology-based layouts with default parameters (FM$^3$~\cite{hachul2004drawing}, Fruchterman-Reingold spring layout (F.R.)~\cite{DBLP:journals/spe/FruchtermanR91}, Stress Majorization (S.M.)~\cite{DBLP:conf/gd/GansnerKN04}, Pivot MDS (PMDS)~\cite{DBLP:conf/gd/BrandesP06}, and Radial Tree layout (R.T.)~\cite{DBLP:conf/infovis/Jankun-KellyM03}).
Over the total $15 \times 5 = 75$ trials, only coefficients of seven trials were less than $\theta = .5$ (see Supplementary Materials 1). And $p$-values were all zero for the five topology-based layouts in all 15 datasets.
The result indicates that in most cases, the Euclidean distance between two nodes reflects their graph geodesic distance with topology-based layouts.
Thus, it is feasible to study whether the layout is a topological layout using the Pearson correlation test.

The attribute-driven layout category consists of algorithms that map node attributes to the two dimensions ($x$ and $y$) of the Cartesian coordinate. 
We applied the Pearson correlation test to Section~\ref{sec:encodingmapping} similarly to test whether an attribute relates to the layout.

To determine the layout type, we utilize the Pearson correlation coefficient. If coefficients of the two layout types are both less than a threshold $\theta < .5$, we suggested there is no certain layout in the node-link diagram. Otherwise, the layout type with a larger coefficient is regarded as the actual layout. 
    
\begin{table*}[th]
\normalsize
\centering
\caption{The templates in the description generator. Placeholders are highlighted with a \textbf{\textit{bold italic}} font.}\label{tab:template}
\includegraphics[width=2\columnwidth]{figures/template.eps}
\vspace{-10pt}
\end{table*}

\section{Description Generator} \label{sec:generator}

\newfboxstyle{light-tight}{padding=1pt,margin=0pt,baseline-skip=false}
\fboxset{light-tight, rounded,border-style=dashed, border-radius=2pt,}%

We designed a set of templates to convert the encoded visual attributes into natural sentences (Figure~\ref{fig:workflow}b). The template is one of the most easy-to-understand ways to interpret visualization. Although many Natural Language Generation (NLG) techniques can generate more natural descriptions, the lack of training samples on node-link diagrams prevents them adapt to our scenarios. Moreover, the template is more controllable, which can be extended to different scenarios and keeps consistency.

\subsection{Description Template}

First, we present a sentence-level template by pre-setting several formats. Our template contains three parts -- 1) static texts, which bear a resemblance to the skeleton of the description and are basically consistent among different scenarios, 2) placeholders, which present the extracted information, thus are the core of our descriptions, and 3) corrections, which adapt our templates to different scenarios (e.g., replacing professional vocabulary with common words). All of our sentences which were included in the templates are listed in Table~\ref{tab:template}. For each template, we gave a sample to illustrate its input and output.

To form a logical format of these descriptions, we utilized a structure to organize them. To make the relationship among different information clear, the structures were logically organized. 

Users can obtain different levels of details in an orderly manner, where more general descriptions are shown with a higher priority.
The first part of the template is the link conditions, which help the users understand the relationship between two nodes.
The next part is the mapping between visual elements and descriptions to generate the interactive description automatically. Due to the automation characteristics, our generation approach can be promoted and adopted to lots of visualization graphs, widely benefiting millions of data news designers and stock analysts. 
The last part is the layout, because it gets the least attention from our end-users in Section~\ref{sec:pilotstudy}.
Because the visual encoding is hierarchically structured, its description generation follows a top-down scheme.
Data entities own visual elements, and visual elements have several visual channels. Some visual channels are used to encode data attributes with a certain correlation.
Thus, they are narrated hierarchically.

To reach an easy-to-understand description, professional vocabulary should be replaced. In our implementation, we replace ``tagName'' with ``shape'', ``r'' with ``radius'', ``rect'' with ``rectangle'', ``stroke\-width'' with ``thinkness'', et cetera. For colors, they are detected with their RGB values. It is obviously that describing the RGB value such as ``\#14dd39'' can not be understood by end-users (except professional designers). We build a vocabulary of 146 different color names with their values. RGB values can be replaced by uses' most similar colors in our vocabulary.


\subsection{Interactive Description}
To complete \textbf{R6}, we enable the \textit{hover-then-highlight} interaction to build mappings between descriptions and the node-link diagram. It makes descriptions more targeted.

The \textit{hover-then-highlight} is two-fold. Hovering on a description will highlight its target visual elements, and hovering on a visual element will highlight its related descriptions.
The highlight is implemented by fading irrelevant elements (Figure~\ref{fig:Movie-Actor-Jean-Pierre}b) or deepening the background color of descriptions (Figure~\ref{fig:Movie-Actor-Jean-Pierre}c).

In order to implement the interaction, we needed to clarify the visual elements related to descriptions.
In the data binding step in Section~\ref{sec:visualencodings}, we have built the relationship between data entities and visual elements. Thus, building mappings from link conditions and layout descriptions is easily achieved. Layout type descriptions highlight node visual elements, while links conditions emphasize link elements. For descriptions of visual encodings, elements to highlight are selected by a filtering mechanism. For instance, in Figure~\ref{fig:Movie-Actor-Jean-Pierre}c, the hovered description tells a color encoding to highlight only the elements that coincide to the description (steelblue nodes) (Figure~\ref{fig:Movie-Actor-Jean-Pierre}b).
 

In addition, to support the hierarchy, we implement a \textbf{collapse} interaction to clarify the ownership of different descriptions (Figure~\ref{fig:Movie-Actor-Jean-Pierre}c). Users can click \includegraphics[height=0.8\baselineskip]{figures/collapse-symbol-1.eps} and \includegraphics[height=0.8\baselineskip]{figures/collapse-symbol-2.eps} symbols at the beginning of the texts to toggle more detailed descriptions.



\begin{figure}[tp]
    \centering
    \setlength{\belowcaptionskip}{-10pt}
    \includegraphics[width=1\columnwidth]{figures/Movie-Actor-Jean-Pierre.eps}
    \caption{Case 1: Usage Scenario. (a) is the node-link diagram created with the Movie-J.P.M. graph (14 nodes and 25 links). (b) When the user hovers on a node, its related descriptions will be highlighted in (c).}
    \label{fig:Movie-Actor-Jean-Pierre}
\end{figure}
    \section{Case Studies} \label{sec:casestudy}

To show the usage scenarios and usability of our~\ApproachName, we designed and conducted three case studies with different link conditions, layout types, and flexible visual encodings.
Based on the IMDb dataset~\cite{IMDb-extensive-dataset} as others did~\cite{DBLP:conf/ieeevast/BigelowNML19, DBLP:journals/ivs/LiuNS14, DBLP:journals/tkde/HerschelNST12}, we generated three graphs and designed their corresponding node-link diagrams. Table~\ref{tab:case-graph} shows their detailed information.

\begin{table*}[t]
\normalsize
\centering
\caption{The proposed graphs used in case studies. }\label{tab:case-graph}
\includegraphics[width=2\columnwidth]{figures/case-graphs.eps}
\vspace{-10pt}
\end{table*}


\subsection{Case 1: Usage Scenario}
This case shows a basic usage scenario. 
Three kinds of information following E1-3 are generated: link conditions, visual encodings, and the layout type.

\noindent \textbf{Link Conditions}.~\ApproachName~recognizes the link condition ($type$=C2, $attribute$=actors, $value$=arbitrary) and thus generates a description in Figure~\ref{fig:Movie-Actor-Jean-Pierre}b line 1.

\noindent \textbf{Visual Encodings}. Nodes are visualized with two different shapes: circles and rectangles.
To distinguish the encoding scheme in different shapes,~\ApproachName~describes them separately.
First, the ``country'' attribute encoded on the shape is narrated (Figure~\ref{fig:Movie-Actor-Jean-Pierre}c, line 3, 4, and 6).
Details of the two shapes are described in line 5, and 7-8.

\noindent \textbf{Layout Type}.
The layout is attribute-based.~\ApproachName~detects the encoding attributes on two coordinates (year and duration) and generates descriptions in Figure~\ref{fig:Movie-Actor-Jean-Pierre}c, line 17-21.

\noindent \textbf{The Hover-then-highlight Interaction}.
The interaction enables users to obtain the targets of descriptions. For instance, hovering line 11 in Figure~\ref{fig:Movie-Actor-Jean-Pierre}c highlights the corresponding elements in Figure~\ref{fig:Movie-Actor-Jean-Pierre}b. So that users can clarify that the description target is the steelblue nodes.


\begin{figure}[tp]
    \centering
    \includegraphics[width=1\columnwidth]{figures/Movie-Year-Nolan.eps}
    \setlength{\abovecaptionskip}{-5pt}
    \setlength{\belowcaptionskip}{-10pt}
    \caption{Case 2: Glyph Design on Link. (a) is the node-link diagram based on the Movie-C.N. graph (11 nodes and 16 links). (b) Several nodes are highlighted when the user hovers on one of their related descriptions in (c).}\label{fig:Movie-Year-Nolan}
\end{figure}

\subsection{Case 2: Glyph Design on Links}
Case 2 (Figure~\ref{fig:Movie-Year-Nolan}) is designed to demonstrate~\ApproachName's ability of dealing with different link conditions and complex visual encodings on links. The design of the node-link diagram follows~\cite{DBLP:conf/iv/SchoffelSE16}. Links are encoded with several bars to display link attributes. It is hard to distinguish the roles of different visual elements in encoding attributes. Our~\ApproachName~gives an amazing performance in analyzing of such visual design.

\noindent \textbf{Link Conditions}. 
First,~\ApproachName~detects the link condition of the node-link diagram ($type$=C3, $attribute$=year, $value$=5) and describes it in Figure~\ref{fig:Movie-Year-Nolan}c, line 1.

\noindent \textbf{Visual Encodings}. 
Although each link has nine visual elements, only four rectangles encode attributes.
Thus, our descriptions only narrate the four rectangles with attributes encoded in Figure~\ref{fig:Movie-Year-Nolan}c, line 7-18.
End-users can focus on the informative visual elements and ignore the others without visual encodings.

\noindent \textbf{Layout Type}.~\ApproachName~detects the force-directed layout. We directly utilize the template to describe it (Figure~\ref{fig:Movie-Year-Nolan}c, line 19).

\noindent \textbf{The Hover-then-highlight Interaction}.
When hovering over certain visual elements, their correlated descriptions will be highlighted with a gray background. For example, in Figure~\ref{fig:Movie-Year-Nolan}(b and c), hovering the red rectangular on a link toggles descriptions in line 1 and 6-9.

\subsection{Case 3: Bar Charts Nested on Nodes}
In this case, we demonstrate~\ApproachName's ability to explain a complex node-link diagram with nested charts on nodes. It follows the design of~\cite{DBLP:journals/bmcbi/JunkerKS06}, where each node nests a chart (Figure~\ref{fig:teaser}). This case is used to illustrate visual encodings, and thus we do not discuss other descriptions here.

\noindent \textbf{Visual Encodings}
In each node, five <rect> (rectangle) elements are composed together to encode the actors' activeness, namely the number of movies they acted in each year.
To help end-users understand such an encoding scheme, our~\ApproachName~generates fine-grained descriptions to narrate each rectangle's encoding scheme (Figure~\ref{fig:teaser}d visual encodings).
One benefit of such descriptions is that users can understand the information encoded on each rectangle (numbers of movies the actor acted in from 2016 to 2020) rather than only an abstract description such as \textit{``encoding the actor's activeness''}.

    \section{User Study}

\begin{figure}[tp]
    \centering
    \setlength{\belowcaptionskip}{-25pt}
    \includegraphics[width=1\columnwidth]{figures/UserStudy.eps}
    \caption{The result of our user study with the mean values of $95\%$ confidence intervals. (a) The questionnaire score distributions of three techniques, (b) the average completion time distributions of three techniques, and (c) the distribution of ratings of different metrics of our interactive descriptions. }
    \label{fig:UserStudy}
\end{figure}


We conducted a qualitative user study to assess the effectiveness of~\ApproachName. The study aimed to evaluate whether our approach can enhance the comprehensibility of node-link diagrams. This study chooses one of the most classic scheme -- the legend, as our baseline. We evaluated our approach's effect on comprehensibility enhancement by comparing the effect of the legend, our generated descriptions, and descriptions without interactions.


\textbf{Participants.}
We recruited 12 participants in the study (P1-P12; 4 males; aged: 22-26 ($\mu$=23.58, $\sigma$=1.32)). All participants are students or researchers from the Computer Science department, three of them majored in Visualization with professional experience in analyzing node-like diagrams, and the others have the basic visualization knowledge by taking the relevant courses. The scale of expert participants is consistent with similar graph visualization research~\cite{DBLP:journals/tvcg/BehrischSP20, DBLP:journals/tvcg/YoghourdjianDKM18}. Each participant received a gift card worth \$15 at the beginning, independent of their performance.


\textbf{Task. }
We chose the three node-link diagrams used in Section~\ref{sec:casestudy} as our test material. We prepared three types of descriptions: $T_l$ is the legend, which serves as the control group, $T_d$ is our descriptions without interaction, and $T_{di}$ is our interactive descriptions. For the legend, there is no standard design to convey information about the layout type. Our legends design only displays the layout type with text (``topology'') and axes (attributes). For each diagram, there will be 
several detailed descriptions samples (ten in case 1, eight in case 2, and eight in case 3) with different types of information (link conditions, visual encodings, and layout type).  To sum up, a complete task contains: 3 \emph{descriptions conditions} $\times (10+8+8)$ \emph{samples} = 78 trials. 

\textbf{Procedure. }
The study began with an introduction (three minutes) of the study purpose, study materials and the study tasks. During the practice phase (five minutes), participants were equipped with a mouse to explore node-link diagrams, legends, and descriptions freely.

After the training, participants started the formal tasks, which lasted for around one hour. We recorded the exploration time and successfulness of each trial. Each trial ended with a post-study questionnaire (one minute) to evaluate the understanding degree of users, which reflects the comprehensibility of our approaches. The questions for the three diagrams are consistent with the three parts of information summarized in Section~\ref{sec:pilotstudy}. 
The total score of one participant was scaled to a five-point score.

After each task, participants were asked to rate four five-point Likert scale questionnaires regarding readability, aesthetics, and utility. Besides, they were encouraged to give suggestions for our techniques.

\subsection{Results, Findings and Lesson Learned}
We consider that the study assessed the system qualitatively rather than quantitatively due to the small sample size. 


\textbf{Comprehensibility}.
The higher the score, the better comprehensibility of node-link diagrams. Overall, the results of $T_d$ ($\mu$=4.72, $\sigma$=.94), $T_{di}$ ($\mu$=4.70, $\sigma$=.79)) and $T_l$ ($\mu$=3.13, $\sigma$=1.86) are shown in Figure~\ref{fig:UserStudy}a. For non-parametric statistical significance test results, the Friedman test suggests significance among three techniques ($p<.05$), and the Wilcoxon Signed-Rank test suggests $T_l$ gets lower scores than both $T_d$ and $T_{di}$ ($p<.05$). No significance is suggested between $T_d$ and $T_{di}$ ($p=.72$). 
We repeated the two tests on the scores of different description types (link conditions, visual encodings, and layout type), and found that $T_d$ and $T_{di}$ were significantly better than $T_l$ while no significance was detected between $T_d$ and $T_{di}$.
The results reported that our two techniques enhanced the comprehensibility of node-link diagrams compared with the legend.
Most of the participants commended our descriptions: \textit{``I do not need to speculate information when viewing descriptions. They are comprehensive and precise.''} 

\begin{itemize}[noitemsep,topsep=0pt,parsep=0pt,partopsep=0pt, leftmargin=20pt]
    \item {\bf Finding 1}: 
    The comprehensibility improvement made by our descriptions mainly comes from the completeness of needed information. The hover-then-highlight interaction helps little to the comprehensibility.
    \item {\bf Lesson Learned 1}: 
    To augment comprehensibility, we can explore other interaction methods and develop more comprehensive descriptions.
\end{itemize}

{\bf Efficiency.} Users reported that the relatively high information granularity of our descriptions makes them spend more time on exploration and relevant information selection. And the analysis on completing time supports their report. 
By employing the two non-parametric statistical significance tests, we find the time of using $T_d$ ($\mu$=44.25, $\sigma$=11.01) and $T_{di}$ ($\mu$=43.10, $\sigma$=13.40) is significantly higher ($p$<.05) than $T_l$ ($\mu$=24.51, $\sigma$=9.81).
We also find there is no significant difference between $T_d$ and $T_{di}$ ($p=.58$). Our interaction has little impact on the exploration time of the task.

\begin{itemize}[noitemsep,topsep=0pt,parsep=0pt,partopsep=0pt, leftmargin=20pt]
    \item {\bf Finding 2}: Exhaustive description costs people more time to explore and select information.
    \item {\bf Lesson Learned 2}: We need to balance the comprehensiveness and conciseness of information. We can highlight some partial information on the descriptions according to usage scenarios and user preferences.
\end{itemize}


\textbf{Utility}.
We found that our participants rated the utility of $T_{di}$ ($\mu$=4.42, $\sigma$=.67) significantly higher than $T_d$ ($\mu$=3.08, $\sigma$=.29), and the utility of $T_d$ is also significantly higher ($p$<.05) than $T_l$ ($\mu$=2.42, $\sigma$=.90).
It means our participants prefer interactive descriptions.
P10 said the interaction attracted his interest to read the description.
P3 mentioned that the spatial relationship between descriptions and the node-link diagram affected his information searching and mapping.
Two participants (P5 and P11) suggested generating tooltips to show the information of each data entity.

\begin{itemize}[noitemsep,topsep=0pt,parsep=0pt,partopsep=0pt, leftmargin=20pt]
    \item {\bf Finding 3}: The distance between descriptions to visualization affects the utility.
    \item {\bf Lesson Learned 3}: We need to consider the spatial relationship between visualization and descriptions. And the tooltip might be an alternative to the current hover-then-highlight interaction.
\end{itemize}

\textbf{Aesthetics}.
The Friedman significance test suggested there is no significance among our descriptions and the legend ($p$=.08).
P1 commented that \textit{``Text colors of descriptions are a little complex.''} She suggested reducing the design complexity by using fewer fonts and colors to highlight essential information.

\begin{itemize}[noitemsep,topsep=0pt,parsep=0pt,partopsep=0pt, leftmargin=20pt]
    \item {\bf Finding 4}: Over-considering the information does not improve readability, but also increases cognitive load for users.
    \item {\bf Lesson Learned 4}: we should rank the importance of different information, and only highlight the most important information.
\end{itemize}

\textbf{Readability}.
Two significance tests suggested that $T_{di}$ ($\mu$=4.50, $\sigma$=.80) is more readable than $T_d$ ($\mu$=3.33, $\sigma$=.89) and $T_l$ ($\mu$=2.67, $\sigma$=.65). And $T_d$ makes no difference to $T_l$.
Although our interaction helps to improve the comprehensibility a bit, it improves the readability of our descriptions.
Most of our participants praised our interaction design, such as \textit{``The interaction locates the target visual elements for me to track and solve questions.''} 
P10, who gave a low score for the readability, commented that descriptions of visual encodings were too trivial for him to find the required information. 
He suggested replacing trivial and repetitive text with symbols, such as icons and arrows.
% Another participant suggested us to generate descriptions with multilingual versions. It can help non-native English users to understand node-link diagrams.
 

\begin{itemize}[noitemsep,topsep=0pt,parsep=0pt,partopsep=0pt, leftmargin=20pt]
    \item {\bf Finding 5}: Trivial and repetitive templates affect the readability of descriptions.
    \item {\bf Lesson Learned 5}: We should explore more organizing forms of descriptions. And replacing repetitive texts with icons such as arrows and emojis may improve the readability.
\end{itemize}


    \section{Discussion}

\noindent {\bf Significance--using textual descriptions.}
\ApproachName~makes it possible for end-users to understand complicated node-link diagrams. Unlike traditional design only express information with simple representations, such as legends, our approach can provide more detailed and concise descriptions. We also found that interactive descriptions can ease users' understating compared with static ones. Moreover, it indicates opportunities for future visualization design and analytics, benefiting not only normal users but also experts. The generated description can help developers debug, modify and re-generate their node-link diagrams quickly, and help designers optimize visual representation logically.


\noindent {\bf Generalizability--extending to other visualization and scenarios.}
\ApproachName~can be applied to other visualization types because different types of visualization are somewhat related and similar. For instance, scatterplots can be viewed as graphs without any link. Thus, our approach can summarize their visual encodings. Moreover, the overall idea of deconstruction visualization can be extended to more scenarios. For instance, we can explain the scatterplot-based dimension reduction algorithm depicting the mainly correlated feature.

\noindent {\bf Comprehensibility--augmenting the design of descriptions}. 
Although our interactive description is practical, some users suggest exploring more information representation and interaction approaches, such as tooltip. One possible optimization is to balance the comprehensiveness and conciseness of the description. We propose three methods to reach this goal. First, Integrate different types of information and thus reduce the repetitive descriptions. Second, hybrid icons and emojis into descriptions to make them more pleasant and concise. Third, apply the shortest distance principle to the placement of the description. In this case, relevant descriptions can be placed near the target visual element, helping users understand the mapping.


\noindent {\bf Scalability-generating large-scale descriptions}.
\ApproachName~has three modules. We analyze the time's complexity of each, respectively. Link conditions searching modules (M1) must traverse all pairs of nodes to find potential conditions. Its time complexity is $O(N^2)$, where $N$ is the number of nodes.
Visual encodings summarizing (M2) need to swap all pairs of nodes and links, and then traverse and compare all elements to obtain their differences.
The number of elements is linear in the number of data entities, so time complexity is $O(N^3 + M^3)$, where $M$ is the number of links.
Detecting visual mappings between attributes and visual channels requires shuffling values of attributes and observing changes in visual channels.
Thus its time complexity is $O(N \times (K_{An} + K_{Cn}) + M \times  (K_{Al} + K_{Cl}))$, where $K_{An}$ and $K_{Al}$ are numbers of attributes on nodes and links, and $K_{Cn}$ and $K_{Cl}$ are numbers of visual channels on nodes and links.
Layout type Identifying M3 requires the computation of two distance matrices.
Finding the shortest paths of all node pairs is $O(N^3)$.
The most time-consuming part is interpreting visual encodings, because a new SVG element must be generated for each swap, and visual channels for each element must be computed.
By implementing our algorithm on the server-side,
the generation of SVG elements and the computation of visual channels can be accelerated.
Besides, for larger-scale node-link diagrams, sampling techniques can be employed for acceleration.

\noindent {\bf Limitations and future work.}
There are several limitations for~\ApproachName.
First, our approach builds upon one assumption that the diagram is consistent. If different parts of the graph are handled differently, for example, nodes are encoded by different encoding schemes,~\ApproachName~cannot extract all these features according to our expectations. 
However, these cases are infrequent and can be mitigated by adding classification strategies to distinguish different parts with disparate schemes.
Second, our approach is designed for web-based visualization. The web environment enables us to ignore the collection of data and visualization programs because network traffic investigation tools can easily capture them. And the SVG-based node-link diagrams make~\ApproachName~possible to deconstruct the visualization results.
Collecting data and programs and deconstructing non-SVG-based visualization are not considered yet.
Third, the types of link conditions and layouts are limited. We focus on specific types according to some existing research currently~\cite{DBLP:journals/tvcg/SrinivasanPEB18, DBLP:conf/ieeevast/BigelowNML19, DBLP:journals/cgf/NobreMSL19}, but we can explore more kinds of link conditions and layout types in the future.
    \section{Conclusion}
In this paper, we have introduced an automatic approach,~\ApproachName, to extract and express information with descriptions for node-link diagrams.
To understand how node-link diagrams are constructed and what should be explained, we have conducted a pilot study with three graph experts and 12 participants to explore the design of our approach.
We formulated two categories of requirements for~\ApproachName~including four extraction requirements and two expression requirements.
Accordingly, we separate the design of~\ApproachName~into two parts: 1) Three modules search link conditions, summarize visual encodings, and identify layout types. It follows the idea of deconstructing node-link diagrams by inferring the information from visualization results.
2) An description generator organizes template-based descriptions following a pre-set scheme. It constructs a linkage between the visualization and descriptions to enable interactions.
Three case studies and the results from an in-lab study show that~\ApproachName~can effectively express the node-link diagram and help end-users get clear understanding of the visualization.

\fi

%% if specified like this the section will be committed in review mode
\acknowledgments{
}

\bibliographystyle{abbrv}
%%use following if all content of bibtex file should be shown
\nocite{}
\bibliography{template}
\end{CJK} %! FOR CHINESE
\end{document}

