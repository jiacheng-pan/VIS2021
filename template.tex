% \documentclass[journal]{vgtc}                % final (journal style)
\documentclass[review,journal]{vgtc}         % review (journal style)
%\documentclass[widereview]{vgtc}             % wide-spaced review
%\documentclass[preprint,journal]{vgtc}       % preprint (journal style)

\ifpdf%                                % if we use pdflatex
  \pdfoutput=1\relax                   % create PDFs from pdfLaTeX
  \pdfcompresslevel=9                  % PDF Compression
  \pdfoptionpdfminorversion=7          % create PDF 1.7
  \ExecuteOptions{pdftex}
  \usepackage{graphicx}                % allow us to embed graphics files
  \DeclareGraphicsExtensions{.pdf,.png,.jpg,.jpeg} % for pdflatex we expect .pdf, .png, or .jpg files
\else%                                 % else we use pure latex
  \ExecuteOptions{dvips}
  \usepackage{graphicx}                % allow us to embed graphics files
  \DeclareGraphicsExtensions{.eps}     % for pure latex we expect eps files
\fi%

%% it is recomended to use ``\autoref{sec:bla}'' instead of ``Fig.~\ref{sec:bla}''
\graphicspath{{figures/}{pictures/}{images/}{./}} % where to search for the images

\usepackage{epstopdf}
\usepackage{microtype}
\PassOptionsToPackage{warn}{textcomp}
\usepackage{textcomp}
\usepackage{mathptmx}
\usepackage{times}
\renewcommand*\ttdefault{txtt}
\usepackage{cite}
\usepackage{tabu}
\usepackage{booktabs}
\usepackage{wrapfig}

\usepackage{amssymb}
\usepackage{paralist}
\usepackage{xfrac}
\usepackage{url}
\usepackage{algorithm}
\usepackage{algpseudocode}
\usepackage{amsmath}
\usepackage{framed}
\usepackage{multirow}
\usepackage{xcolor}
\usepackage{todonotes}
\usepackage[shortlabels]{enumitem}
% \usepackage[xcolor=dvipdf]{changes}
\usepackage[xcolor=dvipdf, final]{changes}
\usepackage{ulem}
\usepackage{color,soul}
\definecolor{mygreen}{rgb}{0.17, 0.55, 0.05}
\definecolor{myred}{rgb}{0.85, 0.17, 0.05}
\definecolor{text-highlight}{rgb}{0.25, 0.62, 0.29}
\sethlcolor{text-highlight}

\usepackage{CJKutf8} %! FOR CHINESE

\usepackage{dcolumn}
\newcolumntype{d}[1]{D{.}{.}{#1}}

\definechangesauthor[name=JiachengPan, color=mygreen]{pan}
\definechangesauthor[name=XiaodongZhao, color=red]{zhao}
\definechangesauthor[name=Jian Chen, color=blue]{jc}
\setdeletedmarkup{\color{gray}{\sout{#1}}}
% \setdeletedmarkup{}
% \sethighlightmarkup{\color{red}{#1}}
\setaddedmarkup{\color{mygreen}{#1}}
\setauthormarkup{}
\renewcommand{\algorithmicrequire}{\textbf{Input:}}
\renewcommand{\algorithmicensure}{\textbf{Output:}}

\definecolor{lightgray}{rgb}{0.75,0.75,0.75}
 
\newenvironment{lightgrayleftbar}{%
  \def\FrameCommand{\textcolor{lightgray}{\vrule width 3pt} \hspace{3pt}}%
  \MakeFramed {\advance\hsize-\width \FrameRestore}}%
{\endMakeFramed}

\newcommand{\ApproachName}{GraphDescriptor}
\newcommand{\PaperTitle}{\ApproachName: An Automatic Description Generator for Node-Link Diagrams} % todo

\onlineid{1561}

\vgtccategory{Research} % TODO
\vgtcpapertype{Representations \& Interaction} % TODO

\title{\PaperTitle}
\author{Jiacheng Pan, Wei Chen, Dongming Han, Xiaodong Zhao, Yingchaojie Feng, Zhuo Shi, Xiaonan Luo and Jian Chen} % TODO
\authorfooter{ % TODO
    \item Jiacheng Pan, Xiaodong Zhao, Yingchaojie Feng, Yuanzhe Hu, and Wei Chen are with the State Key Lab of CAD\&CG, Zhejiang University, China. E-mail: \{panjiacheng, zhaoxiaodong\}@zju.edu.cn, chenwei@cad.zju.edu.cn.\\
    Wei Chen is the corresponding author.
    
    \item Jian Chen is with Ohio State University, USA. E-mail: chen.8028@osu.edu.
}

\shortauthortitle{Pan \MakeLowercase{\textit{et al.}}: \PaperTitle}

\abstract{
  % 节点链接图被广泛用于对图数据进行可视化。我们设计并评估了一个新的为节点链接图自动生成描述的技术,该技术从节点链接图的创作的角度出发来生成描述,其核心思想是将图数据预处理,节点链接图编码,布局生成三个角度,抽取相应的信息并填充到文本模板中,为观众生成相应的描述。我们设计并开发了一个原型系统,为创作者自动生成描述。最后,我们发布了一个gallery来展示我们的方法对多种节点链接图生成的描述,以及进行了两个user study来对自动生成的结果进行评估。
  Node-link diagrams are widely used to visualize graph data. We design and evaluate a novel automatic description generation technique \ApproachName, to facilitate describing node-link diagrams from the perspective of creating such diagrams. The key idea is to extract relevant information from three node-link diagram creation steps: graph wrangling, visual encoding, and layout computing. Descriptions are generated by filling the extracted information into templates. We design and develop an interactive prototype system for demonstration. We present a gallery to exhibit the auto-generated descriptions for diverse node-link diagrams and conduct two user studies to evaluate the descriptions.
}

\keywords{Node-link diagram, graph visualization}

%% ACM Computing Classification System (CCS). 
%% See <http://www.acm.org/class/1998/> for details.
%% The ``\CCScat'' command takes four arguments.

\CCScatlist{ % not used in journal version
 \CCScat{K.6.1}{Management of Computing and Information Systems}%
{Project and People Management}{Life Cycle};
 \CCScat{K.7.m}{The Computing Profession}{Miscellaneous}{Ethics}
}

\teaser{
  \centering
%   \includegraphics[width=\linewidth]{CypressView}
  \includegraphics[width=1\linewidth]{figures/teaser.eps}
  \caption{In the Clouds: Vancouver from Cypress Mountain. Note that the teaser may not be wider than the abstract block.}
  \label{fig:teaser}
}

\vgtcinsertpkg

\begin{document}
\begin{CJK}{UTF8}{gbsn} %! FOR CHINESE

\firstsection{Introduction}
\maketitle

% 图是一种广泛使用的数据结构,在金融、社交、生物等领域都扮演着重要的角色。
Graphs, also called networks, are ubiquitous over numerous areas. For example, graphs can model the financial transaction behaviors, social media contacts, biological processes such as protein-protein-interactions, and so on.
% 节点链接图被广泛应用于对图数据进行可视化,可以用来揭示数据实体之间的关系。
Node-link diagrams are widely used to visualize graphs to reveal the connections and relationships among data entities.
Generating a caption to summarize a node-link diagram is crucial to enhance its accessibility.
However, the automatic caption generation has not been explored yet.

% 为可视化生成语言描述有很多好处,它的主要目的是为了帮助观众理解可视化所传达的意义。
Generating a caption, or a description, to summary a visualization has numerous benefits.
% 根据生成的描述,可以将这类工作分成两类,一类描述了可视化的基本构成(比如数据本身,可视映射等),另一类则关注于可视化中蕴含的insight。
According to the results, automatic description generation techniques can be roughly summarized into two categories~\cite{DBLP:conf/inlg/ObeidH20}: one describes constituents of a visualization objectively(e.g., visual mappings, underlying data, etc.)~\cite{DBLP:journals/coling/MittalMCR98, DBLP:journals/tochi/FerresLST13} and the other describes high-level insights conveyed by the visualization subjectively~\cite{DBLP:conf/apvis/LiuXHWY20, DBLP:conf/inlg/ObeidH20}.
% 这些工作,基本都是为基础可视化图表生成相关描述
They are mostly aimed at generating descriptions for basic charts such as line charts, bar charts, area charts, and so on.
% 其中第二类工作生成的insight大多是基于模板,或是跟训练数据相关的,其自由度会受到这些因素的限制。
Insights generated by the latter are most template-based, or related to the training data, thus will be limited by such factors.
% 本身节点链接图背后的图数据,相比于基础图表背后的表格型数据更具复杂性
Compared to basic charts, underlying graphs behind node-link diagrams are usually more complex, 
% 所以第二类工作反而会导致生成的描述不够准确或者不够完备。
insights generated may be inaccurate or incomplete, thus influence analysis.
% 第一类工作对于解释可视化本身很有意义,而insights则由用户自己决定,这会给予用户更多的分析自由度。
Whereas, the former category of techniques only describe the visualization itself, while insights are decided by audiences, thus reach a higher degree of analysis freedom.
% 我们的方法只描述节点链接图的创作者所做的事情,而不去限制用户的思考和分析。
We only focus on generating objective descriptions about node-link diagrams and do not limit audiences' thinking and analysis.


To generate such descriptions, we need to decompose the creation process of node-link diagrams to explore the challenges.
% 一个节点链接图的创作过程普遍会有以下X个步骤(其中第2步和第3步的顺序可以调换)[这里需要引一些文献]
% 1. 数据预处理;一般而言,真实世界收集到的源数据常常是表格型的多属性数据,而输入到节点链接图创作的图数据,往往是在源数据的基础上进行预处理,通过在源数据上定义实体之间的关系(链接),最终获得输入到节点链接图的图数据(包含节点和链接)
% 2. 设计可视编码;创作者还会把他想表达的数据特征编码到节点链接图上,比如利用节点的颜色来编码节点的类别等。
% 3. 计算图布局;节点链接图需要解决的一个关键问题是如何在二维空间中放置节点。链接则往往被绘制为节点之间的连接线,其位置往往由其两个端点决定。基于布局算法产生的节点位置,节点链接图才能进行绘制。
The creation of node-link diagrams commonly includes the following steps (the order of step 2 and step 3 can be reversed)~\cite{DBLP:journals/cgf/SpritzerBDFF15, tvcg/RomatAP21}:
\begin{compactenum}[\textbf{Step} 1)]
\item \textbf{Graph wrangling} transforms the original data from tabular format to graph format by defining relationships between entities~\cite{DBLP:journals/tvcg/SrinivasanPEB18, DBLP:conf/ieeevast/BigelowNML19, DBLP:journals/ivs/HeerP14, DBLP:journals/ivs/LiuNS14}.
% 虽然Graphiti~\cite{DBLP:journals/tvcg/SrinivasanPEB18}从用户的demonstration中推测构建链接的条件,但它并不能直接被用于推测已经所有边已经构建完毕的图数据。
Although Graphiti~\cite{DBLP:journals/tvcg/SrinivasanPEB18} is designed to infer linking conditions from user demonstrations, it can not be employed directly to infer linking conditions in a graph with all links constructed.
% 需要一个新的方法,能够处理多条边之间的条件的逻辑关系,来得出完备的结论。
A new pipeline is needed to deal with logical relationships among multiple linking conditions to obtain a complete conclusion.

\item \textbf{Layout computing} positions nodes in a two-dimensional space to reveal attribute-based or topology-based patterns~\cite{DBLP:journals/cgf/NobreMSL19}.
% 虽然已经有很多布局相关的工作,但几乎没有工作能够推测某种布局结果的意义。
Although numerous layout algorithms are proposed~\cite{}, but none of them can be used to infer the meaning of a specific layout.

\item \textbf{Visual mapping} encodes node/link attributes with different visual channels to show data features. 
% 从可视化结果中提取 visual mapping 的工作也有很多,但他们大多为了处理一些基础图表,缺少对于节点链接图的支持,又或者需要强有力的先验假设(比如需要d3将数据绑定到可视化元素上),使得工作难以拓展到更多领域。我们的方法将对已有方法进行补充,使他们能够适用于更广的范围,能够良好支持节点链接图中的视觉映射的抽取。
Numerous techniques are proposed to extract visual mappings from visualization~\cite{},  they mainly aim at basic charts such as line charts, bar charts, and so on.
They rarely support node-link diagrams or are not robust enough to deal with general conditions (e.g., without binding data to visual elements).
\end{compactenum}

% 我们提出了xxxx,已解决上述挑战。
We introduce~\textit{\ApproachName}, an automatic caption generator for node-link diagrams.
% xxxx 检验了创作者创作可视化的过程,从图数据,可视化构建源码中,提取解决上述挑战的关键信息。
It extracts key information from the creation process of node-link diagrams to solve challenges mentioned above.
% 最终将提取的信息注入到一些预定义的模板中生成文字描述形成标题。
And we fill the information into several pre-defined templates to generate textual descriptions as captions
% 我们通过xxx对该方法的有效性进行了验证。
{\color{red} We evaluate ...}
% 我们的贡献可以被归纳为:
Our contributions can be concluded as:
{\colorbox{text-highlight}TODO}

%? 重要:需要提及我们的方法是template-based,重点不在于NLG,而是在于问题的formulation和信息的提取。
\section{Related Work}\label{sec:relatedwork}
\subsection{Node-link Diagrams}
\subsection{Extracting Information From Visualization}
% 抽取信息对于生成描述至关重要。
Extracting information from visualization is crutial for generating descriptions.
% 对可视化进行信息抽取的工作大概可以分成两类。
% 第一类工作聚焦从可视化结果中抽取insight。
% Chart-to-Text: Generating Natural Language Descriptions for Charts by Adapting the Transformer Model.
% Contextifier: Automatic generation of annotated stock visualizations.
% Automatic annotation synchronizing with textual description for visualization.
% Describing Complex Charts in Natural Language: A Caption Generation System

% 第二类工作聚焦于从各种可视化结果还原潜在数据。
Another category of methods focus on retrieve data from visualization results.
% 不同类型的chart需要不同的还原算法,所以每种方法往往聚焦在一个类型的chart上
Different types of charts require different retrieval algorithm.
Thus each method most focuses on one type of charts.
A Modified Probabilistic Hough Transform algorithm (MPHT) was developed to detect parallel lines clusters and it reconstrcts bar patterns from the clusters. It retrieves data from bar charts in raster images even hand-drawn images~\cite{DBLP:conf/icip/ZhouT00}. 
A system is developed to extract data from bar charts, pie charts, and line charts with different rules~\cite{DBLP:conf/doceng/HuangT07}. It also extracts textual information and text/graphics association to generate description for charts and their underlying data.

% TODO  View: Visual Information Extraction Widget for improving chart images accessibility

% TODO  Access to multimodal articles for individuals with sight impairments

% TODO  Getting computers to see information graphics so users do not have to


% ChartSense: Interactive Data Extraction from Chart Images: 先将图像进行分类,然后利用了一个mixed-initiative approach对图标蕴含的数据进行抽取。人首选指定检测范围和坐标轴等辅助信息,ChartSense算法检测其中包含的图形元素及其蕴含的数据,最后由人挑选检测正确的元素并修改其中错误的数据。
% Scatteract: Automated Extraction of Data from Scatter Plots: employs deep learning techniques to identify data from scatter plots with linear scales. It is the first approach that fully aumatic extracts data from scatter plots.
% 

% 第三类工作聚焦于从可视化结果中还原可视编码
% Reverse-Engineering Visualizations: Recovering Visual Encodings from Chart Images: 从 像素图 中识别文本信息(标题、坐标轴标签等),对文本进行分类和还原,并且用CNN网络来分类mark类型,通过文本和mark类型来恢复编码信息。

% Deconstructing and restyling D3 visualizations: 使用了d3数据绑定的特性,直接从svg元素的__data__属性中读取其绑定的数据,然后用线性回归等方法检验元素的视觉属性和数据属性之间的映射关系来获取编码。
% Converting basic D3 charts into reusable style templates 和 Searching the Visual Style and Structure of D3 Visualizations 对 Decons.. 进行了改进。前者围绕着图表中蕴含的label(数据点的label和坐标轴的label)进行了更加细致的分类;后者则为该方法增加了一些新的检测目标:比如对角度的检测,对非编码的视觉通道(background color, stroke-width等)的检测等。

% Visualizing for the Non-Visual: Enabling the Visually Impaired to Use Visualization :


% 这些方法或是只适用于检测基础的可视化图表如bar chart/line chart/area chart等,或是只适用于对d3等具有数据绑定的svg进行编码提取。
% 我们提出了对节点链接图进行信息提取的方法,该方法从多个角度出发,适用于通用的svg场景,而不需要数据绑定的前提。
\section{Approach Workflow}
% 我们的方法目的是帮助节点链接图的创作者自动生成辅助性语句,帮助节点链接图的目标观众理解节点链接图所要表达的内容。
We present \ApproachName~to generate presentation-aided statements automatically for node-link diagrams creators. The generated statements are aimed to facilitate the comprehension of the created node-link diagrams for target audiences.
% 为了生成能够帮助观众理解的语句,我们需要通过分析节点链接图的构建过程,来寻找观众会对节点链接图产生疑惑的原因。
To generate such presentation-aided statements, we need to decompose the creation process of node-link diagrams to explore the reasons of the audiences' confusion.
% 一个节点链接图的创作过程普遍会有以下X个步骤(其中第2步和第3步的顺序可以调换)[这里需要引一些文献]
% 1. 数据预处理;一般而言,真实世界收集到的源数据常常是表格型的多属性数据,而输入到节点链接图创作的图数据,往往是在源数据的基础上进行预处理,通过在源数据上定义实体之间的关系(链接),最终获得输入到节点链接图的图数据(包含节点和链接)
% 2. 计算图布局;节点链接图需要解决的一个关键问题是如何在二维空间中放置节点。链接则往往被绘制为节点之间的连接线,其位置往往由其两个端点决定而无需计算。基于布局算法产生的节点位置,节点链接图才能进行绘制。
% 3. 设计可视编码;创作者还会把他想表达的数据特征编码到节点链接图上,比如利用节点的颜色来编码节点的类别等。
% 虽然在很多工作中,还支持交互式的修改图布局。但这并不在本文的讨论中,本文只讨论为静态的节点链接图生成辅助性语句。
The creation of node-link diagrams commonly includes the following steps~\cite{DBLP:journals/cgf/SpritzerBDFF15, tvcg/RomatAP21}:
\begin{compactenum}[\textbf{Step} 1.]
    \item Data wrangling. 
    \item Layout computing.
    \item Visual encoding.
\end{compactenum}
\section{Case Studies}
We employ the IMDb movie dataset\footnote{\small\url{https://www.kaggle.com/stefanoleone992/imdb-extensive-dataset}} to conduct our case studies that demonstrate the descriptions generated by \ApproachName.
They are available online\footnote{\small\url{https://graphdescriptor.github.io/}}.


\begin{figure*}[ht]
    \centering
    % \setlength{\belowcaptionskip}{-5pt}
    \includegraphics[width=2\columnwidth]{figures/BasicCases.eps}
    \caption{Two simple node-link diagrams with descriptions generated by our approach. 
    Nodes represent movies, and their links represent that two movies have common actors.
    The number of two movies' common actors is encoded by the link thickness in both diagrams.
    Their visual encodings and layouts differ:
    (a) encodes movies' publication seasons using different node colors; nodes are placed by a force-directed layout.  
    The nodes in (b) encode a movie's country, first genre, and its number of votes by shape, color, and size. 
    Nodes are placed with an attribute-based layout whose \textit{x}-coordinate encodes the movie's publication year, and \textit{y}-coordinate encodes movie duration.
    (c) and (d) are descriptions generated by our approach, each with three parts interpreting different steps in the creation process. }
  \label{fig:BasicCases}
\end{figure*}

\subsection{Case 1: Simple Nodes and Links}\label{sec:imdb_movies}
% 我们挑选了数据集中2019年在中国上映的电影。

We select movies in the dataset which are published in 2019, China and preserve seven movie attributes: \texttt{"title"}, \texttt{"date\_published"} (the publication date), \texttt{"actors"}, \texttt{"genre"}, \texttt{"director"}, \texttt{"writer"}, and \texttt{"country"}.

\noindent \textbf{1) Graph wrangling}. 
Two movies are connected if they have at least one same actor.
We generate an attribute, \texttt{"common\_actors"} representing the number of common actors for each link. 
We obtain 49 nodes and 99 links in total.

\noindent \textbf{2) Visual encoding}. 
Nodes and links are visualized as \texttt{<circle>}s and \texttt{<line>}s separately (Figure~\ref{fig:BasicCases} (a)).
The fill color of a \texttt{<circle>} encodes it publication season (e.g., when \texttt{"date\_published"} is \texttt{01}, \texttt{02}, or \texttt{03}, it means the publication season is spring and we fill the node with \texttt{steelblue}).
The width of \texttt{<line>} encodes the attribute \texttt{"common\_actors"}.

\noindent \textbf{3) Layout computing}. 
We pre-compute a layout with a spring layout algorithm~\cite{DBLP:journals/spe/FruchtermanR91} to reveal movies' relationships.

\textbf{Results}.
Descriptions are generated for the created node-link diagram (Figure~\ref{fig:BasicCases} (c)).
They can be divided into three parts:

\noindent \textbf{1) Interpreting Linking Conditions.} 
Several conditions (e.g. (\texttt{"C2"}, \texttt{"country"}, \texttt{"China"}) and (\texttt{"C1"}, \texttt{"year"}, \texttt{"2019"}), etc.) are detected but only the condition (\texttt{"C2"}, \texttt{"actors"}, \texttt{"[arbitrary value]"}) is preserved because other conditions also hold on unconnected node pairs.
Thus \ApproachName~describes the linking condition as \textit{``Two nodes are connected if their {\texttt{"actors"}} have common values''}.

\noindent \textbf{2) Interpreting Visual Encodings.} 
All visual encodings are detected by \ApproachName~as expected.
However, our technique does not detect that the color of a node encodes the publication season because ``season'' is an abstract concept that GraphDescriptor does not consider.
However, it generates the mapping of different color categories: 
\textit{``When the value of the attribute {\texttt{"date\_published[month]"}} is {\texttt{04}}, {\texttt{05}}, or {\texttt{06}}, 
its {\texttt{fill}} is changed into {\texttt{darkorange (\#ff7f0e)}}.$\ldots$''}, 
where \texttt{"date\_published[month]"} is the \texttt{"month"} field of the attribute publication date (\texttt{"date\_published"}).
Similar descriptions are generated to describe four different \texttt{fill} colors with four seasons of month, so that the audience learns that the fill color encodes the ``season''.

\noindent \textbf{3) Interpreting Layout Intention.} \ApproachName~detects how the nodes are placed in a topology-based layout. 
Detection of a specific layout type is not supported in our technique.
Thus, \ApproachName~only describes the meaning of the topology-based layout in Figure~\ref{fig:BasicCases} (c) (Layout Intention): \textit{``the farther the topology distance between two nodes, the farther the distance between them''}.

\subsection{Case 2: Nodes with Different Shapes}
This case demonstrates the generated descriptions where the shape of a node is used to encode a categorical attribute (Figure~\ref{fig:BasicCases} (b) and (d)).
We select movies directed by Jean-Pierre Melville. 

\noindent \textbf{1) Graph Wrangling}. 
Movies are connected if they have common actors. 
We finally obtain 14 nodes (movies) and 25 links.

\noindent \textbf{2) Visual Encodings}.
Movies directed by Melville were mostly published in France; through several appeared in Italy.
We use circular nodes to encode movies published in France and square nodes to encode movies published in Italy.
The size encodes the number of votes (\texttt{"votes"}) they have received.
The color encodes their primary genre (the first genre in their genre attribute \texttt{"genre"}).

\noindent \textbf{3) Layout Computing}.
The x and y coordinates encode the publication year (\texttt{"year"}) and the \texttt{"duration"} separately.

\textbf{Results}.
Descriptions are generated for the created node-link diagram (Figure~\ref{fig:BasicCases} (d)).
Unlike the former case, nodes are visualized with two different elements: \texttt{<circle>} and \texttt{<rect>}.
Their visual channels have differences, and so \ApproachName~describes them separately:
\textit{``Its \texttt{tagName} encodes the attribute country.
When the \texttt{"country"} is \texttt{France}, its \texttt{tagName} is changed into \texttt{<circle>}.
Its \texttt{radius} encodes the attribute \texttt{"votes"}.
When the \texttt{"country"} is \texttt{Italy}, its \texttt{tagName} is changed into \texttt{<rect>}.
Its \texttt{width} and \texttt{height} encode the attribute \texttt{"votes"}''}.

Another difference is that the layout is attribute-based.
Our technique detects the meanings of two coordinates by testing correlations between node positions and attributes: 
\textit{``The x-coordinate encodes the attributes \texttt{"year"}. 
The greater the \texttt{"year"}, the greater the x-coordinate.
The y-coordinate encodes the attributes \texttt{"duration"}.
The greater the \texttt{"duration"}, the greater the y-coordinate.''}.


\begin{figure*}[ht]
    \centering
    \includegraphics[width=2\columnwidth]{figures/LinkBarsCase.eps}
    \caption{ (a) A node-link diagram following the visual design proposed by Sch{\"{o}}ffel et al.~\cite{DBLP:conf/iv/SchoffelSE16}. Nodes represent movies directed by Christopher Nolan, and links are constructed if differences between two movies' publication years are less than 5. We employ the force-directed layout. Four bars on a link encode four attributes: the difference of their budgets (the red bar, \texttt{"budget\_diff"}), the difference of their duration (the yellow bar, \texttt{"duration\_diff"}), the difference of numbers of their votes (the blue bar, \texttt{"votes\_diff"}), and the difference of their average vote scores (the green bar, \texttt{"avg\_vote\_diff"}). (b) Descriptions generated by \ApproachName. Both the attribute-based layout and the topology-based layout are detected. Because we connect movies with similar publication years, movies are laid out from the upper-right corner to the lower-left corner according to the order of publication years. Their budgets also increase over time from the upper-right corner to the lower-left corner.}
    \label{fig:LinkBarsCase}
\end{figure*}

\subsection{Case 3: Line Charts Nested on Nodes}
With the IMDb movie dataset, we introduce another case that focuses on actors that demonstrates more complex situations in which each node has multiple elements (Figure~\ref{fig:NodeLineChartsCase} (b) and (f)).

\noindent \textbf{1) Graph Wrangling}. 
We first select movies in the dataset that are only released only in China from 2016 to 2021.
We select actors who have acted in more than five movies as nodes.
Each actor has five attributes: \texttt{"name"}, \texttt{"movies"} (movies s/he acted in from 2016 to 2021), \texttt{"avg\_vote"} (the average vote), \texttt{"votes"} (the number of votes s/he got), and \texttt{"number\_of\_movies\_by\_year"} (the number of movies in each year from 2016 to 2021).
Two actors are connected if they acted in at least one movie.
Finally, we obtain 17 nodes and 55 links.

\noindent \textbf{2) Visual Encodings}. 
Each node (actor) contains an attribute named \texttt{"number\_of\_movies\_by\_year"} that has five properties: 2016, 2017, 2018, 2019, and 2020.
Each property stores the number of movies the actor acted in that year.
Following the node-link diagram created by Junker et al.~\cite{DBLP:journals/bmcbi/JunkerKS06}, 
we nest a simple line chart into each node to show the number of movies the actor acted in from 2016 to 2021 (Figure~\ref{fig:NodeLineChartsCase} (a)). 
The width of the link's \texttt{<line>} encodes its attribute \texttt{"number\_of\_shared\_movies"}.

\noindent \textbf{3) Layout Computing}. We utilize the layout to show two attributes (\texttt{"avg\_vote"} and \texttt{"votes"}) of each node. The x-coordinate encodes the attribute \texttt{"votes"} and the y-coordinate encodes the \texttt{"avg\_vote"}.

\textbf{Results}. 
Unlike Section~\ref{sec:imdb_movies}, here each node has multiple elements in this case.
We bind elements to different entities and distinguish their roles.
We interpret the meaning of the line chart by describing its elements.
As described in the Visual Encodings part of Figure~\ref{fig:NodeLineChartsCase} (b), it contains five elements.
The first four elements are \texttt{<line>}s.
% 它们组合在一起编码了每个演员每年出演电影数量的趋势,
They are composed together to encode the actors' activeness, namely the number of movies they acted in each year.
Descriptions like \textit{``encode the actor' activeness''} are domain-knowledge based; our technique generates fine-grained descriptions to help audiences form complex connections.
% 于是我们的方法对它们的y值分别进行了描述:
We describe them separately: 
\textit{``The first element is a \texttt{<line>}. 
Its \texttt{y1} encodes the attribute \texttt{"number\_of\_movies\_by\_year[2019]"}. 
The greater the \texttt{"number\_of\_movies\_by\_year[2019]"}, the greater its \texttt{y1}. 
Its \texttt{y2} encodes the attribute \texttt{"number\_of\_movies\_by\_year[2020]"}.
The greater the \texttt{"number\_of\_movies\_by\_year[2020]"}, the greater its \texttt{y2}.''}
Then the description is repeated with different elements and attributes.
Such descriptions are fine-grained and can be composed to form complex tasks in the future.


\subsection{Case 4: Bars on Links}
Case 4 demonstrates the generated descriptions for links.
The design of the node-link diagram follows Sch{\"{o}}ffel et al.~\cite{DBLP:conf/iv/SchoffelSE16}, links are encoded with several bars to display link attributes.

\noindent \textbf{1) Graph Wrangling}.
We select movies directed by Christopher Nolan.
Attributes \texttt{"budget"}, \texttt{"duration"}, \texttt{"votes"}, \texttt{"avg\_vote"}, \texttt{"director"}, \texttt{"title"}, \texttt{"year"}, and \texttt{"date\_published"} are preserved on nodes.
Two nodes are connected if the difference between their \texttt{"year"}s is less than 5.
We calculate several attributes for links to show the difference of their end nodes, such as the difference in \texttt{"budget"}s, \texttt{"duration"}, \texttt{"votes"}, and \texttt{"avg\_vote"}.

\noindent \textbf{2) Visual Encodings}.
The radius of the node encodes its \texttt{"avg\_vote"}.
Widths of four bars on a link encode the difference of \texttt{"budget"}s (\texttt{"budget\_diff"}), \texttt{"duration"}s (\texttt{"duration\_diff"}), \texttt{"votes"}s (\texttt{"votes\_diff"}), and \texttt{"avg\_vote"}s (\texttt{"avg\_vote\_diff"}) respectively (Figure~\ref{fig:LinkBarsCase} (a)).
Their colors are used merely to distinguish them rather than encode attribute values.

\noindent \textbf{3) Layout Computing}.
A force-directed layout is employed.

\textbf{Results}.
As expected, four \texttt{<rect>}s are discussed separately (the Visual Encodings part in Figure~\ref{fig:LinkBarsCase} (b)):
\textit{``The first element is a \texttt{<rect>}. Its \texttt{width} encodes the attribute \texttt{"budget\_diff"}. The greater the attribute \texttt{"budget\_diff"} is, the greater its \texttt{width} is.''}
Descriptions of the other three \texttt{<rect>}s are similar.
Although we only compute a force-directed layout for the graph, our technique suggests that both the attribute-based layout and the topology-based layout are detected (Figure~\ref{fig:LinkBarsCase} (b)).
This is because we connect movies with similar years and the force-directed layout coincidentally places the earliest movie in the upper-right corner and the latest movie in the lower-left corner.
Earlier movies have smaller \texttt{"year"}s, \texttt{"date\_published[year]"}s (the year of publication), and \texttt{"budget"}s.
Thus, the layout is also attribute-based.
Our technique helps to obtain this impressive finding.
\section{User Study}
1. 针对两个case 找一个expert帮忙写一些描述,定义相关交互,我们帮助它生成相应的交互式描述;
2. 自动生成交互式描述;让expert做一些评价。
3. 各找12个人观看描述;回答五分量表;(readability (e.g., is it easy to read and follow the logic?),
utility (e.g., does it help you understand this visual design?), aesthetics
(e.g., does it look pretty and pleasant?) and attractiveness (e.g., does it
attract your interest?).) 在expert的描述上挖一些空,让被试填入;
\section{Discussion}


\subsection{Failure Cases and Limitations}
\textbf{Failure Cases}.
We observed several failure cases during the development of our technique.
Our technique assumes the consistency of the diagram.
It is not able to deal with complex cases where different parts of the graph are dealt differently.
For example, a part of nodes are connected if they have common values on one attribute (condition 2) while the other connected nodes are because of the closeness of another numerical attribute (condition 3).
Similarly, \ApproachName~cannot deal with node-link diagrams where different nodes have different encoding schemes and where different areas of the diagram are positioned by different layouts.
For example, our visual encoding extraction technique in Section~\ref{sec:visualencodings} is not able to deal with nodes visualized by different numbers of primitives.
Such failure cases are rare in common situations and can be mitigated by add classification strategies to distinguish different parts with diverse schemes.
For example, we can detect potential priori conditions for different schemes.
A set of data entities with the same linking condition or the same encoding scheme may have commonalities in some attributes.

\textbf{Scalability}.
\ApproachName~consists of three parts:
\textbf{interpreting linking conditions} need to tranverse all pairs of nodes to find potential conditions.
Thus, its time complexity is $O(N^2)$, where $N$ is the number of nodes.
\textbf{Interpreting visual encodings} swaps all pairs of nodes and links, and then traverse and compare all primitives to obtain their differences.
The number of primitives are linear to the number of data entities.
Its time complexity is $O(N^3 + M^3)$.
Time of detecting visual mappings between attributes and channels can be ignored compares to $O(N^3 + M^3)$.
\textbf{Interpreting layout meanings} requires the computation of two distance matrices.
Finding the shortest paths of all node pairs is $O(N^3)$.
The most time consuming part is interpreting visual encodings.
It needs to generate a new SVG element for each swapping and compute visual channels for each primitive.
We plan to implement our algorithm in the server side to improve its performance.
The generation of SVG elements and the computation of visual channels can be accelerated.
\section{Conclusion}
\replaced{We designed and evaluated}{In this work, we present} an exemplar-based \added{graph} layout fine-tuning approach that reduces human labor by transferring modifications made on an exemplar to other substructures. 
A user interface is developed to enable fine-tuning of graph layouts.
A quantitative comparison of two datasets with ground truth indicates that our approach can reach more accurate correspondences. Three case studies show that our approach works well on different datasets and layouts. A user study shows that our approach significantly reduces or even eliminates laborious interactions.

%% if specified like this the section will be committed in review mode
\acknowledgments{ % TODO
We wish to thank all the anonymous reviewers for their thorough and constructive comments. We also thank the participants for their time and efforts. 
This work is partially supported by National Natural Science Foundation of China (61772456, 61761136020), NSFC (61761136020), NSFC-Zhejiang Joint Fund for the Integration of Industrialization and Informatization (U1609217), and Zhejiang Provincial Natural Science Foundation (LR18F020001). J. Chen is partially supported by National Science Foundation NSF OAC-1945347, NSF DBI-1260795, NSF IIS-1302755, CNS-1531491, and NIST MSE-70NANB13H181. Any opinions, findings, and conclusions or recommendations expressed in this material are those of the authors and do not necessarily reflect the views of  the National Science Foundation of China (NSFC), National Institute of Standards and Technology (NIST) or the National Science Foundation (NSF).
}

\bibliographystyle{abbrv}
%%use following if all content of bibtex file should be shown
\nocite{*}
\bibliography{template}
\end{CJK} %! FOR CHINESE
\end{document}

